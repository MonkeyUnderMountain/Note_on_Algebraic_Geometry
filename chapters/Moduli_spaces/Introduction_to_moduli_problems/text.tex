\section{Introduction to moduli problems}


\subsection{Moduli problem by representable functors}
    
    Moduli space is a geometric space whose points represent isomorphism classes of certain geometric objects.
    For example, fix a field \(\kkk\), all elliptic curves over \(\kkk\) can be classified by the \(j\)-invariant, which gives a bijection between isomorphism classes of elliptic curves and elements of \(\kkk\).
    However, this classification is just a set-theoretic one.
    We would like to have a geometric ``parameter space'' such that we can ``deform'' the objects continuously.
    This is the initial motivation for moduli spaces.

    In algebraic geometry, ``deforming objects continuously'' is usually described by flat families.
    Hence the most perfect object to represent this moduli problem is such a flat family \(\calX \to M\), where \(M\) is a variety parameterizing all elliptic curves, and the fiber over each point \(m \in M(\kkk)\) is the elliptic curve corresponding to \(m\).
    Furthermore, we hope that this family is universal, i.e., any other flat family \(\calY \to B\) of elliptic curves is obtained by pulling back \(\calX \to M\) along a unique morphism \(B \to M\).
    (Despite in this example, such a perfect family does not exist; we will discuss this later.)
    It can be described by the language of functors.
    Consider the functor
    \[ \calM: \Var_{\kkk}^{\op} \to \Set,\quad X \mapsto \{\text{flat families of elliptic curves over } X\}/\sim, \]
    where \(\sim\) is the equivalence relation given by isomorphisms of families over \(X\).
    If the family \(\calX \to M\) above exists, then it represents the functor \(\calM\).
    In this case, we say that the moduli problem \(\calM\) is \emph{representable}, \(M\) is called a \emph{fine moduli space} and \(\calX \to M\) is called a \emph{universal object}.


    
    % Moduli space is a geometric space whose points represent isomorphism classes of certain geometric objects.
    % For example, fix a field \(\kkk\), we want to parameterize all line on the projective plane \(\bbP_{\kkk}^2\).
    % Each line \(L \subset \bbP_{\kkk}^2\) can be described by a linear equation
    % \[ a x + b y + c z = 0, \]
    % where \([a:b:c] \in \bbP_{\kkk}^2\).
    % Hence, all lines on \(\bbP_{\kkk}^2\) can be parameterized by the dual projective plane \((\bbP_{\kkk}^2)^{\vee}\).
    % However, this parameterization is just a set-theoretic one.
    % We would like to have a geometric ``parameter space'' such that we can ``deform'' the objects continuously.
    % This is the initial motivation for moduli spaces.

    % In algebraic geometry, ``deforming objects continuously'' is usually described by flat families.
    % Hence the best object to represent this moduli problem is such a flat family \(\calU \to M\), where \(M\) is a variety parameterizing all lines on \(\bbP_{\kkk}^2\), and the fiber over each point \(m \in M(\kkk)\) is the line corresponding to \(m\).
    % Furthermore, we hope that this family is universal, i.e., any other flat family \(\calY \to B\) of lines on \(\bbP_{\kkk}^2\) is obtained by pulling back \(\calU \to M\) along a unique morphism \(B \to M\).
    % It can be described by the language of representable functors.
    % Consider the functor
    % \[ \calM: \Sch_{\kkk}^{\op} \to \Set,\quad X \mapsto \{L \subset \bbP^2 \times X | L \text{ is flat over } X, L_x \text{ is a line in } \bbP^2 \text{ for all } x \in X(\kkk)\}. \]




    % Hence the most perfect object to represent this moduli problem is such a flat family \(\calX \to M\), where \(M\) is a variety parameterizing all elliptic curves, and the fiber over each point \(m \in M(\kkk)\) is the elliptic curve corresponding to \(m\).
    % Furthermore, we hope that this family is universal, i.e., any other flat family \(\calY \to B\) of elliptic curves is obtained by pulling back \(\calX \to M\) along a unique morphism \(B \to M\).
    % (Despite in this example, such a perfect family does not exist; we will discuss this later.)
    % It can be described by the language of functors.
    % Consider the functor
    % \[ \calM: \Var_{\kkk}^{\op} \to \Set,\quad X \mapsto \{\text{flat families of elliptic curves over } X\}/\sim, \]
    % where \(\sim\) is the equivalence relation given by isomorphisms of families over \(X\).
    % If the family \(\calX \to M\) above exists, then it represents the functor \(\calM\).
    % In this case, we say th

    Hence, to study a moduli problem, we follow the steps:
    \begin{enumerate}[label=\arabic*.]
        \item Define the moduli functor \(\calM\).
        \item Check whether \(\calM\) is representable.
    \end{enumerate}

    \begin{slogan}
        To solve a moduli problem is to find a representable functor which describes ``continuous deformations''.
    \end{slogan}

    \begin{definition}\label{def:fine_moduli_space}
        Let \(S\) be a noetherian scheme and \(\calM: \Sch_{S}^{\op} \to \Set\) be a functor.
        If \(\calM\) is representable by a scheme \(M\) of finite type over \(S\), then we say that \(M\) is a \emph{fine moduli space} for the moduli problem \(\calM\).
        The object \(\calU \in \calM(M)\) corresponding to the identity morphism \(\id_M \in \Hom_{\Sch_{S}}(M, M)\) is called a \emph{universal object} over \(M\).
    \end{definition}

    We give a simple example of a fine moduli space.

    \begin{example}\label{eg:moduli_of_lines_in_P2}
        Consider the moduli problem of lines in the projective plane \(\bbP_{\kkk}^2\) over a field \(\kkk\).
        Define the moduli functor
        \[
            \calG: \Sch_{\kkk}^{\op} \to \Set,\quad X \mapsto \{L \subset \bbP^2 \times X | L \text{ is flat over } X, L_x \text{ is a line in } \bbP^2 \text{ for all } x \in X(\kkk)\}.
        \]

        We claim that the dual projective plane \(G = \bbP_{\kkk}^2\) is a fine moduli space for the moduli problem \(\calG\).
        The universal object is given by 
        \[
            U = \{([x:y:z], [a:b:c]) \in \bbP_{\kkk}^2 \times G | ax + by + cz = 0\} \subset \bbP_{\kkk}^2 \times G.
        \]
        For every \(X \in \Sch_{\kkk}\) and \(f \in G(X)\), we can give a family of lines \(L = U \times_G X \in \calG(X)\) by pulling back the universal object \(U\) along \(\id_{\bbP_{\kkk}^2} \times f: \bbP_{\kkk}^2 \times X \to \bbP_{\kkk}^2 \times G\).
        The difficult part is to construct the inverse map, i.e., given a family of lines \(L \in \calG(X)\), we need to construct a morphism \(f: X \to G\) such that \(L\) is obtained by pulling back \(U\) along \(\id_{\bbP_{\kkk}^2} \times f\).

        We need a more ``functorial'' way to describe the dual projective plane \(G\).
        Set \(V = H^0(\calO_{\bbP_{\kkk}^2}(1))\), let \(G = \Proj (\Sym^{\bullet} V^{\vee})\) be the dual projective plane.
        To give a morphism \(f:X \to G\), it is equivalent to giving a line bundle \(\calL\) on \(X\) and a surjective morphism \(\calO_X \ten_\kkk V^{\vee} \surjmap \calL\) by \Yang{To be added.}.
        
        Let \(\calI_L\) be the ideal sheaf of \(L\) in \(\bbP_X^2 = \bbP^2 \times X\).
        Consider the short exact sequence on \(\bbP_X^2\):
        \[ 0 \to \calI_L(1) \to \calO_{\bbP_{X}^2}(1) \to \calO_{L}(1) \to 0. \]
        Since \(L\) is a family of lines, we have \(\calI_L(1)|_{\bbP^2 \times \{x\}} \cong \calO_{\bbP^2}\) for all \(x \in X(\kkk)\).
        By Theorem of Cohomology and Base Change (\Yang{To be added.}), we have \(\calL^\vee \coloneqq (\pr_X)_*(\calI_L(1))\) is a line bundle on \(X\) and \(\sfR^1(\pr_X)_*\calI_L(1) = 0\).
        Then, pushing forward the above short exact sequence along the projection \(\pr_X: \bbP_X^2 \to X\) gives a short exact sequence
        \[ 0 \to \calL^{\vee} \to \calO_X \ten_{\kkk} V \to (\pr_X)_*\calO_L(1) \to 0. \]
        Dualizing it gives a surjective morphism \(\calO_X \ten_{\kkk} V^{\vee} \surjmap \calL\).
        In particular, if \(X = G\) and \(L = U\) is the universal family, we get the line bundle \(\calU = \calO_G(1)\) on \(G\) and the surjective morphism \(\calO_G \ten_{\kkk} V^{\vee} \surjmap \calO_G(1)\) corresponding to the identity morphism \(\id_G\).

        Let \(f: X \to G\) be the morphism induced by the surjective morphism \(\calO_X \ten_{\kkk} V^{\vee} \surjmap \calL\).
        Then we have the following commutative diagram:(\Yang{Why?})
        \[ \begin{tikzcd}
            f^* (\calO_{G} \ten_\kkk V^\vee) \arrow[d,"\cong"] \arrow{r}    & f^*\calU \arrow[d,"\cong"] \arrow{r} & 0 \\
            \calO_{X}\ten_\kkk V^\vee \arrow[r]                             & \calL \arrow{r} & 0.
        \end{tikzcd} \]
        Take duals and pull back along \(\pr_X: \bbP_X^2 \to X\), we get a commutative diagram: (\Yang{To be added.})
        \[ \begin{tikzcd}
            0   \arrow{r}   & \pr_X^*f^* \calU^\vee \arrow[d,"\cong"] \arrow{r} & \pr_X^*f^* \calO_{G} \ten_\kkk V \arrow[d,"\cong"] \\
            0   \arrow[r]   & \pr_X^* \calL^\vee \arrow{r} & \pr_X^* \calO_{X}\ten_\kkk V.
        \end{tikzcd} \]
        Note that \(\calL^\vee = (\pr_X)_*(\calI_L(1))\) and \(\calU^\vee = (\pr_G)_*(\calI_U(1))\).
        Hence, by cohomology commuting with flat base change \Yang{To be added.}, we have
        \[ \begin{tikzcd}
            0   \arrow{r}   & \pr_X^* (\pr_X)_* \Big((\id_{\bbP^2} \times f)^* \calI_U(1)\Big) \arrow[d,"\cong"] \arrow{r}  & \pr_X^* \Big(f^* \calO_{G} \ten_\kkk V\Big) \arrow[d,"\cong"] \\
            0   \arrow[r]   & \pr_X^* (\pr_X)_* \Big(\calI_L(1) \Big) \arrow{r}                                             & \pr_X^* \Big(\calO_{X}\ten_\kkk V\Big).
        \end{tikzcd} \]
        Note that \((\pr_X)_* \calI_L(1)\) is a line bundle and hence we have \(\pr_X^*(\pr_X)_* (\calI_L(1) ) \cong \calI_L(1)\).
        Together with the natural surjective homomorphism \(\pr_X^* (\calO_{X}\ten_\kkk V) \to \calO_{\bbP_X^2}(1)\), we have 
        \[ \begin{tikzcd}
                        &                                                                               & \pr_X^* \Big(f^* \calO_{G} \ten_\kkk V\Big) \arrow[dr, two heads] & \\
            0 \arrow[r] & (\id_{\bbP^2} \times f)^* \calI_U(1) \arrow[d,"\cong"] \arrow[rr] \arrow[ur]  &                                                                   & \calO_{\bbP_X^2}(1) \arrow[d,"\cong"]  \\
            0 \arrow[r] & \calI_L(1) \arrow[rr] \arrow[dr]                                              &                                                                   & \calO_{\bbP_X^2}(1). \\
                        &                                                                               & \pr_X^* \Big(\calO_{X}\ten_\kkk V\Big)      \arrow[ur,two heads]  &   \\
        \end{tikzcd} \]
        \Yang{Why this homomorphism injective and why it is the natural inclusion?}
        After identifying the last vertical isomorphism, we have the equality of sheaf ideals \((\id_{\bbP^2} \times f)^* \calI_U = \calI_L\).
        It follows that \(L = U \times_G X\) and we are done.
        % \Yang{To be completed.}
    \end{example}

    


\subsection{Coarse moduli space}