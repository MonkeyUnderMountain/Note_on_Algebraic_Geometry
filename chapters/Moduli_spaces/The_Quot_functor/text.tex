\section{The Quot functor}

\subsection{Definitions and examples}

    \begin{definition}\label{def:Quot_functor}
        Let \(S\) be a noetherian scheme, \((X,\calO(1))\) a projective scheme over \(S\), and \(\calE\) a vector bundle on \(X\).
        For a polynomial \(P \in \bbQ[\lambda]\),
        The \emph{Quot functor} \(\frakQuot_{\calE/X/S,P}:\Sch_S^\op \to \Set\) is defined as
        \[ T \mapsto \left\{ \calE_{X_T} \to \calQ \to 0 \text{ on } X_T \ \middle|\ \calQ \text{ is flat over } T,\text{ and } P_{\calQ|_{X_\xi}} = P,\ \forall \xi \in T \right\} / \sim, \]
        where two quotients \(\calE_{X_T} \to \calQ \to 0\) and \(\calE_{X_T} \to \calQ' \to 0\) are equivalent if there is an isomorphism \(\calQ \cong \calQ'\) making the diagram
        \[ \begin{tikzcd}
            \calE_{X_T} \arrow[r] \arrow[rd] & \calQ \arrow[d, "\cong"] \\
            & \calQ'
        \end{tikzcd} \]
        commute.
    \end{definition}

    The main goal of this section is to prove the following representability theorem.

    \begin{theorem}\label{thm:Quot_is_representable}
        The Quot functor \(\frakQuot_{\calE/X/S,P}\) is representable by a projective \(S\)-scheme \(\Quot_{\calE/X/S,P}\) and a universal quotient \(p_X^*\calE \to \calQ \to 0\) on \(X \times_S \Quot_{\calE/X/S,P}\).
        \Yang{To be checked.}
    \end{theorem}

    Many important moduli spaces can be realized as special cases of the Quot scheme.
    
    \paragraph{Grassmannian scheme} The first example is the Grassmannian scheme.

    \begin{definition}\label{def:Grassmannian_functor}
        Let \(S\) be a noetherian scheme and \(\calE\) a vector bundle of rank \(n\) on \(S\).
        The \emph{Grassmannian functor} \(\frakGrass_{\calE,r}:\Sch_S^\op \to \Set\) is defined as 
        \[ T \mapsto \left\{\calE_T \to \calQ \to 0 \ \middle|\ \calQ \text{ locally free of rank } r \text{ on } T \right\} / \sim, \]
        where two quotients \(\calE_T \to \calQ \to 0\) and \(\calE_T \to \calQ' \to 0\) are equivalent if there is an isomorphism \(\calQ \cong \calQ'\) making the diagram
        \[ \begin{tikzcd}
            \calE_T \arrow[r] \arrow[rd] & \calQ \arrow[d, "\cong"] \\
            & \calQ'
        \end{tikzcd} \]
        commute.
    \end{definition}

    Let \(i:\xi \to S\) be a point of \(S\).
    Then the fiber \(\calQ|_{\xi} = i^*\calQ\) is a vector space over the residue field \(\kappa(\xi)\) of dimension \(r\).
    By taking \((X,\calO(1)) = (S, \calO_{S})\) and \(P(\lambda) = r\), the Grassmannian functor \(\frakGrass_{\calE,r}\) is a special case of the Quot functor \(\frakQuot_{\calE/X/S,P}\).

    Let us further specialize to the case where \(S = \Spec \kk\) for a field \(\kk\) and \(\calE = V = \kk^{\oplus n}\) is a finite-dimensional \(\kk\)-vector space.
    Note that \(V \to W \to 0\) is equivalent to \(V \to W' \to 0\) if and only if \(\ker(V \to W) = \ker(V \to W')\).
    Hence in this case, the Grassmannian functor \(\frakGrass_{\calE,r}\) becomes the classical Grassmannian variety \(\Gr(n-r,V)\) parameterizing \(n-r\)-dimensional subspaces of \(\kk^{\oplus n}\).

    More specially, when \(r = 1\) or \(r = n-1\), the Grassmannian variety \(\Gr(n-r,V)\) is the projective space \(\bbP^{n-1}_\kk\).
    However, although the space is the same, the universal object is different.
    When \(r=1\), i.e., \(\bbP_\kk^{n-1}\) parameterizes quotients \(V \to W \to 0\) where \(W\) is a one-dimensional vector space, the universal object is 
    \[ \bigoplus_{i=1}^n \calO_{\bbP_\kk^{n-1}} \cdot e_i \to \calO_{\bbP_\kk^{n-1}}(1) \to 0, \quad e_i \mapsto x_i, \]
    where \(x_1,\ldots,x_n\) are the homogeneous coordinates on \(\bbP_\kk^{n-1}\).
    When \(r=n-1\), i.e., \(\bbP_\kk^{n-1}\) parameterizes quotients \(V \to W \to 0\) where \(W\) is an \((n-1)\)-dimensional vector space, the universal object is
    \[ 0 \to \calO_{\bbP_\kk^{n-1}}(-1) \xrightarrow{\varphi} \bigoplus_{i=1}^n \calO_{\bbP_\kk^{n-1}} \cdot e_i \to \calQ \to 0, \quad \varphi(1/x_i) = e_i, \]
    where we view \(\calO_{\bbP_\kk^{n-1}}(-1)\) locally generated by \(1/x_i\) on the chart \(x_i \neq 0\).

    When we say \(\bbP(V)\), we usually mean the case \(r=1\).
    This is also called the projectivization of the vector space \(V\) of hyperplanes in \(V\) or in the sense of Grothendieck.
    Under this convention, the functor \(\bbP: \vect_\kk \to \Sch_\kk\) sending a finite-dimensional vector space \(V\) to the projective space \(\bbP(V)\) is contravariant.
    Hence one should view \(V\) as the space of linear functions rather than points.

    \Yang{To be continued, describe \(\bbP^n_S\) for general \(S\).}

    \paragraph{Hilbert scheme} Another important example is the Hilbert scheme.

    \begin{definition}\label{def:Hilbert_functor}
        Let \(S\) be a noetherian scheme and \(X\) a projective scheme over \(S\).
        For a polynomial \(P \in \bbQ[\lambda]\), the \emph{Hilbert functor} \(\frakHilb_{X/S,P}:\Sch_S^\op \to \Set\) is defined as
        \[ T \mapsto \left\{ Y \subseteq X_T \ \middle|\ Y \to T \text{ is flat and for all } \xi \in T,\ P_{\calO_{Y_\xi}} = P \right\}, \]
        where \(Y_\xi\) is the fiber of \(Y\) over the point \(\xi\) and \(P_{\calO_{Y_\xi}}\) is the Hilbert polynomial of \(\calO_{Y_\xi}\) with respect to \(\calO_X(1)|_{X_\xi}\).
    \end{definition}

    By taking \(\calE = \calO_X\) and noting that a quotient \(\calO_{X_T} \to \calQ \to 0\) corresponds to a closed subscheme \(Y = \calSpec_{X_T} \calQ \subseteq X_T\), 
    we see that the Hilbert functor \(\frakHilb_{X/S,P}\) is a special case of the Quot functor \(\frakQuot_{\calE/X/S,P}\).

    \Yang{To be continued}


    \paragraph{Morphisms space} Yet another example is the morphisms space.

    \begin{definition}\label{def:functor_of_morphism_space}
        Let \(S\) be a noetherian scheme, \(X,Y\) projective schemes over \(S\), and \(f:X \to Y\) a morphism over \(S\).
        The \emph{functor of morphism space through \(f\)} \(\frakMor_{(X,Y)/S,f}:\Sch_S^\op \to \Set\) is defined as
        \[ T \mapsto \left\{ g_T:X_T \to Y_T \text{ over } T \middle| \Gamma_g \text{ is flat over } T \text{ and } P_{\Gamma_{g_\xi}} = P_{\Gamma_{f_\xi}} \text{ for all } \xi \in T \right\}, \]
        where \(X_T = X \times_S T\) and \(Y_T = Y \times_S T\).
        \Yang{To be revised.}
    \end{definition}

% \subsection{Quot functor and scheme}



\subsection{Castelnuovo-Mumford regularity}



\subsection{Construction of Quot scheme}
