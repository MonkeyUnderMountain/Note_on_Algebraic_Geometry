\section{Definitions and examples}

    Let \(\kkk\) be an algebraically closed field of characteristic zero. 
    Let \(G\) be an algebraic group over \(\kkk\) acting on a variety \(X\) over \(\kkk\).

    % \begin{definition}\label{def:G-invariant_morphism}
    %     Let \(X\) and \(Y\) be varieties with a \(G\)-action.
    %     A morphism \(\varphi: X \to Y\) is called \emph{\(G\)-invariant} if the following diagram
    %     \[
    %     \begin{tikzcd}
    %         G \times X \arrow[r, "\sigma_X"] \arrow[d, "id_G \times \varphi"'] & X \arrow[d, "\varphi"] \\
    %         G \times Y \arrow[r, "\sigma_Y"] & Y
    %     \end{tikzcd}
    %     \]
    %     commutes, where \(\sigma_X\) and \(\sigma_Y\) denote the actions of \(G\) on \(X\) and \(Y\) respectively.
    % \end{definition}
    
    \begin{definition}\label{def:category_quotient}
        A \emph{categorical quotient} of \(X\) by \(G\) is a variety \(Y\) with trivial \(G\)-action together with a \(G\)-invariant morphism \(\pi: X \to Y\) 
        such that for any \(G\)-invariant morphism \(\varphi: X \to Z\) to a variety \(Z\) with trivial \(G\)-action, 
        there exists a unique morphism \(\psi: Y \to Z\) making the following diagram commute:
        \[
        \begin{tikzcd}
            X \arrow[r, "\pi"] \arrow[rd, "\varphi"'] & Y \arrow[d,gray, "\psi"] \\
            & Z
        \end{tikzcd}
        \]
    \end{definition}


    \begin{definition}\label{def:geometric_quotient}
        A \emph{geometric quotient} of \(X\) by \(G\) is a variety \(Y\) together with a \(G\)-invariant morphism \(\pi: X \to Y\) satisfying the following conditions:
        \begin{enumerate}
            \item The morphism \(\pi\) is surjective, and \((\sigma,p_2):G \times X \to X \times X\) factors through the fiber product \(X \times_Y X \to X \times X\) and \(G \times X \to X \times_Y X\) is surjective;
            \item The topology on \(Y\) is the quotient topology induced by \(\pi\), i.e., a subset \(U \subseteq Y\) is open if and only if \(\pi^{-1}(U)\) is open in \(X\);
            \item The structure sheaf \(\mathcal{O}_Y\) is given by the sheaf of \(G\)-invariant regular functions on \(X\), i.e., for any open subset \(U \subseteq Y\),
            \[
                \mathcal{O}_Y(U) = \mathcal{O}_X(\pi^{-1}(U))^G.
            \]
        \end{enumerate}
    \end{definition}

    \Yang{geometric quotient must be categorical quotient}
    \begin{proposition}\label{prop:geometric_quotient_is_categorical_quotient}
        A geometric quotient is a categorical quotient.
    \end{proposition}

    \begin{example}\label{eg:G_m_acting_on_A^1_has_no_geometric_quotient}
        Let \(G = \bbG_m\) and \(X = \bbA^1\) with the action of \(G\) on \(X\) given by multiplication.
        \Yang{To be continous}
    \end{example}

    \Yang{Scheme and universal and uniform quotient.}


    \begin{proposition}\label{prop:descend_properties_to_categorical_quotient}
        Let \(\pi: X \to Y\) be a categorical quotient of \(X\) by \(G\).
        \begin{enumerate}
            \item If \(X\) is separated, then \(Y\) is separated.
            \item If \(X\) is normal, then \(Y\) is normal.
            \item If \(X\) is smooth, then \(Y\) is smooth.
        \end{enumerate}
        \Yang{To be revised}
    \end{proposition}

    \begin{proposition}\label{prop:fiber_is_single_orbit_implies_geometric_quotient}
        Let \(\pi: X \to Y\) be a categorical quotient of \(X\) by \(G\).
        If for every \(y \in Y\), the fiber \(\pi^{-1}(y)\) is a single \(G\)-orbit, then \(\pi: X \to Y\) is a geometric quotient.
        \Yang{To be checked.}
    \end{proposition}