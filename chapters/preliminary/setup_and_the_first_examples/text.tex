\section{Setup and the first examples}

    All schemes are assumed to be separated.
    For a ``scheme'' which is not separated, we will use the term ``prescheme''.
 
    Let $S$ be $\spec \kk$, $\spec \calo_K$ or an algebraic variety.
    An $S$-variety is an integral scheme $X$ which is of finite type and flat over $S$.
    For an algebraic variety, we mean a $\kk$-variety.

    We will use $\kk,\KK$ to denote fields, and $\kkk,\KKK$ to denote their algebraically closure relatively.

    Let $X$ be an integral scheme.
    We denote by $\fkk(X)$ the function field of $X$.
    For a closed point $x \in X$, we denote by $\rkk(x)$ the residue field of $x$.

    We denote the category of $S$-varieties by $\Var_S$.
    We denote by $X(T)$ the set of $T$-points of $X$, that is, the set of morphisms $T \to X$.

    Let $X$ be an algebraic variety over $\kk$.
    A geometrical point is referred a morphism $\spec \kkk \to X$.

    When refer a point (may not be closed) in a scheme, we will use the notation $\xi \in X$.
    We 
    When we talk about a closed point on an algebraic variety, we will use the notation $x \in X(\kkk)$.


\begin{example}
    Let $\kkk$ be an algebraically closed field and $A$ the localization of $\kkk[x]$ at $(x)$.
    Let $S = \Spec A$ and $X = \Spec A[y]$. 
    There are three types of points in $X$:
    \begin{enumerate}[label=(\roman*)]
        \item closed points with residue field $\kkk$, like $p = (x,y-a)$;
        \item closed points with residue field $\kkk(y)$, like $P = (xy-1)$;
        \item non-closed points, like $\eta_1 = (x),\eta_2 = (y),\eta_3 = (x-y)$.
    \end{enumerate}

\end{example}

    \subsection{Preparation in commutative algebra}

    \begin{definition}[Associated prime ideals]
        Let $A$ be a noetherian ring and $M$ a finitely generated $A$-module.
        The \textit{associated prime ideals} of $M$ are the prime ideals $\frakp$ of form $\Ann(x)$ for some $x \in M$.
        The set of associated prime ideals of $M$ is denoted by $\Ass(M)$.
    \end{definition}


    \begin{definition}
        Let $A$ be a noetherian ring and $M$ a finitely generated $A$-module.
        The \textit{support} of $M$ is the set of prime ideals $\frakp$ of $A$ such that $M_\frakp \neq 0$.
    \end{definition}


