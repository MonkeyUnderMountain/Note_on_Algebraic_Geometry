\section{Setup and the first examples}

    Let $S$ be a locally noetherian and separated scheme.
    An $S$-variety is a separated scheme $X$ which is of finite type and flat over $S$.

    We will use $\mathsf{k},\mathsf{K}$ to denote fields, and $\mathbf{k},\mathbf{K}$ to denote their algebraically closure relatively.
    Let $X$ be an integral scheme.
    We denote by $\KK(X)$ the function field of $X$.
    For a closed point $x \in X$, we denote by $\mathsf{k}(x)$ the residue field of $x$.
    We denote the category of $S$-varieties by $\Var_S$.
    We denote by $X(T)$ the set of $T$-points of $X$, that is, the set of morphisms $T \to X$.


\begin{example}
    Let $\kkk$ be an algebraically closed field and $A$ the localization of $\kkk[x]$ at $(x)$.
    Let $S = \Spec A$ and $X = \Spec A[y]$. 
    There are three types of points in $X$:
    \begin{enumerate}[label=(\roman*)]
        \item closed points with residue field $\kkk$, like $p = (x,y-a)$;
        \item closed points with residue field $\kkk(y)$, like $P = (xy-1)$;
        \item non-closed points, like $\eta_1 = (x),\eta_2 = (y),\eta_3 = (x-y)$.
    \end{enumerate}

\end{example}
