\section{Setup and the first examples}

    % Let $S$ be a locally noetherian, separated and integral scheme.
    Let $S$ be $\spec \kk$, $\spec \calo_K$ or an algebraic variety.
    An $S$-variety is a separated and integral scheme $X$ which is of finite type and flat over $S$.

    We will use $\kk,\KK$ to denote fields, and $\kkk,\KKK$ to denote their algebraically closure relatively.

    Let $X$ be an integral scheme.
    We denote by $\fkk(X)$ the function field of $X$.
    For a closed point $x \in X$, we denote by $\rkk(x)$ the residue field of $x$.

    We denote the category of $S$-varieties by $\Var_S$.
    We denote by $X(T)$ the set of $T$-points of $X$, that is, the set of morphisms $T \to X$.

    Let $X$ be an algebraic variety over $\kk$.
    A geometrical point is referred a morphism $\spec \kkk \to X$.

    When refer a point (may not be closed) in a scheme, we will use the notation $\xi \in X$.
    We 
    When we talk about a closed point on an algebraic variety, we will use the notation $x \in X(\kkk)$.


\begin{example}
    Let $\kkk$ be an algebraically closed field and $A$ the localization of $\kkk[x]$ at $(x)$.
    Let $S = \Spec A$ and $X = \Spec A[y]$. 
    There are three types of points in $X$:
    \begin{enumerate}[label=(\roman*)]
        \item closed points with residue field $\kkk$, like $p = (x,y-a)$;
        \item closed points with residue field $\kkk(y)$, like $P = (xy-1)$;
        \item non-closed points, like $\eta_1 = (x),\eta_2 = (y),\eta_3 = (x-y)$.
    \end{enumerate}

\end{example}
