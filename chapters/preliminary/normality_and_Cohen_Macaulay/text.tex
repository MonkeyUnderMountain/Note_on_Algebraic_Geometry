\section{Normal, Cohen-Macaulay and regular schemes}

\subsection{Height, Depth and Dimension \Yang{To be completed}}

    \paragraph{Krull dimension and height of prime ideals}
    Algebraically, we have the following definitions.
    \begin{definition}\label{def: height of ideals}
        Let $A$ be a noetherian ring.
        The \textit{height of a prime ideal} $\frakp$ in $A$ is defined as the maximum length of chains of prime ideals contained in $\frakp$, that is, 
        \[ \idealht(\frakp) := \sup\{ n \mid \exists \text{ a chain of prime ideals } \frakp_0 \subsetneq \frakp_1 \subsetneq \cdots \subsetneq \frakp_n = \frakp\}. \] 
        The \textit{Krull dimension} of $A$ is defined as 
        \[ \dim A := \max_{\frakp \in \Spec A}\idealht(\frakp). \]
    \end{definition}

    Geometrically, we have the corresponding definition.
    \begin{definition}\label{def: dimension of schemes}
        Let $X$ be a noetherian scheme.
        The \textit{codimension of an irreducible subscheme} $Y$ in $X$ is defined as the length of the longest chain of irreducible closed subsets containing $Y$, that is, 
        \[ \codim_X(Y) := \sup\{ n \mid \exists \text{ a chain of irreducible closed subsets } Y = Y_0 \subsetneq Y_1 \subsetneq \cdots \subsetneq Y_n\}. \] 
        The \textit{dimension} of $X$ is defined as
        \[ \dim X := \max_{\xi \in X}\codim_X Z_\xi. \]
    \end{definition}

    For an affine scheme $X = \Spec A$, above two definitions coincide by the correspondence of prime ideals and irreducible closed subsets.

    \begin{proposition}\label{prop: dimension of localization, height and codimension}
        Let $A$ be a noetherian ring and $\frakp \in \Spec A$.
        Then 
        \[ \idealht(\frakp) = \codim_{\Spec A} V(\frakp) = \dim A_{\frakp}. \]
    \end{proposition}

    \begin{lemma}\label{lem: finite inclusion induces surjective morphism}
        Let $A \subset B$ be noetherian rings such that $B$ is finite over $A$.
        Then the induced morphism $\Spec B \to \Spec A$ is surjective.
    \end{lemma}
    \begin{proof}
        For $\frakp \in \Spec A$, let $S:= A-\frakp$ and denote $S^{-1}B$ by $B_\frakp$.
        Then we have $A_\frakp \injmap B_\frakp$ and $B_\frakp$ is finite over $A_\frakp$.
        Let $\frakP B_\frakp$ be a maximal ideal of $B_\frakp$.
        We claim that $\frakP B_\frakp \cap A_\frakp$ is maximal.
        Indeed, consider $A_\frakp/(\frakP \cap A_\frakp) \injmap B_\frakp/\frakP B_\frakp$, the latter is finite over the former.
        This enforces $A_\frakp/(\frakP B_\frakp \cap A_\frakp)$ be a field.
        Hence $\frakP B_\frakp \cap A_\frakp = \frakp A_\frakp$, and then $\frakP \cap A = \frakp$.
    \end{proof}

    \begin{proposition}\label{prop: finite morphisms preserve dimension}
        Let $A \subset B$ be noetherian rings such that $B$ is finite over $A$.
        Then $\dim A = \dim B$.
    \end{proposition}
    \begin{proof}
        If we have a sequence $\frakP_1 \subsetneq \frakP_2$ of prime ideals in $B$, then there exists $f \in \frakP_2 \setminus \frakP_1$.
        Since $B$ is finite over $A$, there exist $a_1,\cdots,a_n \in A$ such that 
        \[ f^n + a_1f^{n-1} + \cdots + a_n = 0.\]

        Then $a_n \in \frakP_2 \cap A$.
        If $a_n \in \frakP_1$, $f^{n-1} + \cdots + a_{n_1} \in \frakP_1$ since $f \notin \frakP_1$.
        Then $a_{n-1} \in \frakP_2$.
        Repeat the process, it will terminate, whence $\frakP_1 \cap A \subsetneq \frakP_2 \cap A$.
        Otherwise, we have $f^n \in a_1 B+ \cdots + a_nB \subset \frakP_1$.

        Conversely, suppose we have $\frakp_1,\frakp_2 \in \Spec A$ with $\frakp_1 \subsetneq \frakp_2$.
        Choose $\frakP_1 \in \Spec B$ such that $\frakP_1 \cap A = \frakp_1$, then we have $A/\frakp_1 \subset B/\frakP_1$.
        Let $\frakP_2$ be the preimage of the prime ideal in $B/\frakP_1$ which is over image of $\frakp_2$ in $A/\frakp_1$.
        Proposition \ref{lem: finite inclusion induces surjective morphism} guarantees that such $\frakP_2$ exists.
        Then we get $\frakP_1 \subsetneq \frakP_2$.
        Repeat this progress, we get $\dim B \geq \dim A$.
    \end{proof}

    \begin{theorem}[Krull's Principal Ideal Theorem]\label{thm: Krull's principal ideal theorem}
        Let $A$ be a noetherian ring.
        Suppose $f \in A$ is not a unit.
        Let $\frakp$ be a minimal prime ideal among those containing $f$.
        Then $\idealht(\frakp)\leq 1$.
        % Moreover, if $A$ is local or $\dim A_\frakm$ is constant for all $\frakm$, then the equality holds.
    \end{theorem}
    \begin{proof}
        By replacing $A$ by $A_\frakp$, we may assume $A$ is local with maximal ideal $\frakp$.
        Note that $A/(f)$ is artinian since it has only one prime ideal $\frakp/(f)$.

        Let $\frakq \subsetneq \frakp$.
        Consider the sequence $\frakq^{(1)} \supset \frakq^{(2)} \supset \cdots$, its image in $A/(f)$ is stationary.
        Then there exists $n \in \bbz_{\geq 0}$ such that $\frakq^{(n)} + (f) = \frakq^{(n+1)} +(f)$.
        For $x \in \frakq^{(n)}$, we may write $x = y + af$ for $y \in \frakq^{(n+1)}$.
        Then $af \in \frakq^{(n)}$.
        Since $\frakq^{(n)}$ is $\frakq$-primary and $f \notin \frakq$, $a \in \frakq^{(n)}$.
        Then we get $\frakq^{(n)} = \frakq^{(n+1)} + f \frakq^{(n)}$.
        That is, $\frakq^{(n)}/\frakq^{(n+1)} = f \frakq^{(n)}/\frakq^{(n+1)}$.
        Note that $f \in \frakp$, by Nakayama's Lemma, $\frakq^{(n)} = \frakq^{(n+1)}$.
        That is, $\frakq^n A_\frakq = \frakq^{n+1} A_\frakq$.
        By Nakayama's Lemma again, $\frakq^n A_\frakq = 0$.
        It follows that $\frakq A_\frakq$ is minimal, whence $A_\frakq$ is artinian.
        Therefore, $\frakq$ is minimal in $A$.
    \end{proof}

    \begin{corollary}\label{cor: geometric form of principal ideal theorem}
        Let $A$ be a noetherian local ring.
        Suppose $f \in A$ is not a unit.
        Then $\dim A/(f) \geq \dim A - 1$.
        If $f$ is not contained in a minimal prime ideal, the equality holds.
    \end{corollary}
    \begin{proof}
        Let $\frakp_0 \subsetneq \cdots \subsetneq \frakp_n$ be a sequence of prime ideals.
        By assumption, $f \in \frakp_n$.
        If $f \in \frakp_0$, we get a sequence of prime ideals in $A/(f)$ of length $n$.
        Now we suppose $f \notin \frakp_0$.
        Then there exists $k \geq 0$ such that $f \in \frakp_{k+1} \setminus \frakp_{k}$. 
            
        Choose $\frakq$ be a minimal prime ideal among those containing $(\frakp_{k-1},f)$ and contained in $\frakp_{k+1}$.
        Then by Krull's Principal Ideal Theorem \ref{thm: Krull's principal ideal theorem}, $\frakq_k \subsetneq \frakp_{k+1}$.
        Replace $\frakp_k$ by $\frakq_k$, we have $f \in \frakq_{k} \setminus \frakp_{k-1}$

        Repeat this process, we get a sequence $\frakp_0' \subsetneq \cdots \subsetneq \frakp_n'$ such that $f \in \frakp_1'$.
        This gives a sequence $\frakp_1' \subsetneq \cdots \subsetneq \frakp_n'$ in $A/(f)$.
        Hence we get $\dim A/(f) \geq \dim A - 1$.
 
        % Then there exist a prime ideal $\frakp_0 \subsetneq (f)$.
        Since $f$ is not contained in minimal prime ideal, preimage of a minimal prime ideal in $A/(f)$ has height $1$.
        Hence a sequence of prime ideals in $A/fA$ can be extended by a minimal prime ideal in $A$.
        It follows that $\dim A/(f) + 1 \leq \dim A$.
    \end{proof}

    For varieties, the Krull dimension behaves well by follows.

    \begin{lemma}\label{lem: dimension of algebraic varieties}
        Let $X$ be an algebraic variety over $\kk$.
        Then for every closed point $x \in X(\kkk)$, we have 
        \[ \dim X = \dim \calo_{X,x} = \trdeg(\fkk(X)/\kk). \] 
    \end{lemma}
    \begin{proof}
        Since $X$ is irreducible, we may assume that $X = \Spec A$ is affine.
        Let $d = \trdeg(\fkk(X)/\kk)$.

        By Noether's Normalization Lemma \ref{thm: Noether's Normalization Lemma}, there is an injective and finite homomorphism $A_0 = \kk[T_1,\cdots,T_d] \injmap A$.
        Let $\frakM$ be the corresponding maximal ideal of $x$ in $A$ and $\frakm = \frakM \cap \kk[T_1,\cdots,T_d]$.
        Denote the image of $T_i$ in $\exk:=A_0/\frakm$ by $t_i$.
        The extension $\exk/\kk$ is finite by Nullstellensatz \ref{thm: Nullstellensatz}.
        Let $f_i\in \kk[T]$ be the minimal polynomial of $t_i$ and $g_i:= f_i(T_i) \in A_0$.
        Then $g_i \in \frakm$ and $\frakm = g_1A_0 +\cdots,g_d A_0$.
        In particular, $g_1,\cdots,g_d \in \frakM$.

        We have $A/g_1A+\cdots+g_dA$ is finite over $A_0/\frakm$, whence it is artinian.
        This implies that $A_\frakM/g_1A_\frakM + \cdots + g_dA_\frakM$ is also artinian.
        Since $g_{k+1}$ is not a zero divisor in $A_0/g_1A_0+\cdots+g_kA_0$, $g_{k+1}$ is not contained in any minimal prime ideal of $A_0/g_1A_0+\cdots+g_kA_0$.
        Then $g_{k+1}$ is also not contained in any minimal prime ideal of $A/g_1A+\cdots+g_kA$.
        By Corollary \ref{cor: geometric form of principal ideal theorem}, $\dim A_\frakM = \dim (A_\frakM/g_1A_\frakM + \cdots + g_dA_\frakM) + d = d$.
    \end{proof}

    \begin{theorem}\label{thm: dimension of varieties}
        Let $S$ be spectrum of a field $\kk$ or an algebraic integer ring $\calo_K$ and $X$ an integral $S$-variety.
        Then we have the follows:
        \begin{enumerate}[label=(\roman*)]
            \item For every point $\xi \in X$, $\dim X = \dim \calo_{X,\xi} + \codim Z_\xi$.
            \item For every non-empty open subset $U \subset X$, $\dim U = \dim X$.
            \item $\dim X = \trdeg(\fkk(X)/\fkk(S)) + \dim S$.
        \end{enumerate}
    \end{theorem}
    \begin{proof}
        \Yang{To be continued.}
    \end{proof}

    \begin{example}
        For general noetherian schemes, Theorem \ref{thm: dimension of varieties} may not hold.
        Let $A = \kk[t]$, $\frakm = (t)$, $B=A_\frakm[x]$ and $X = \Spec B$.
        Then we have $\dim X = 2$ since 
        \Yang{To be added.}
    \end{example}


    \paragraph{Depth}
    For a noetherian local ring $(A,\frakm)$, we can define the depth of an $A$-module $M$.
    Somehow the Krull dimension is ``homological'' and the depth is ``cohomological''.

    \begin{definition}\label{def: regular sequence}
        Let $A$ be a noetherian ring, $I \subset A$ an ideal and $M$ a finitely generated $A$-module.
        A sequence $t_1,\cdots,t_n\in \frakm$ is called an \textit{$M$-regular sequence in $I$} if $t_i$ is not a zero divisor on $M/(t_1,\cdots,t_{i-1})M$ for all $i$. 
    \end{definition}

    \begin{example}
        Let $A = \kk[x,y]/(x^2,xy)$ and $I = (x,y)$.
        Then $\depth_I A = 0$.
    \end{example}

    % \begin{proposition}\label{lem: permutation of regular sequence is still regular}
    %     Suppose $A$ is local and $I = \frakm$ is the maximal ideal of $A$.
    %     Then any permutation of an $M$-regular sequence is $M$-regular.
    % \end{proposition}
    % \begin{proof}

    %     \Yang{To be completed.}
    % \end{proof}

    \begin{definition}
        The \textit{$I$-depth} of $M$ is defined as the maximum length of $M$-regular sequences in $I$, denoted by $\depth_I M$. 
        When $A$ is a local ring with maximal ideal $\frakm$, we write $\depth M$ for $\depth_{\frakm} M$.
    \end{definition}

    \paragraph{Regular and Serre's conditions}
    Up to now, there are three numbers measuring the ``size'' of a local ring $(A,\frakm)$:
    \begin{itemize}
        \item $\dim A$: the Krull dimension of $A$.
        \item $\depth A$: the depth of $A$.
        \item $\dim_{\rkk(\frakm)} T_{A,\frakm}$: the dimension of Zariski tangent space $T_{A,\frakm} := (\frakm/\frakm^2)^\lor$ as a ${\rkk(\frakm)}$-vector space.
    \end{itemize}

    These three numbers are related by the following inequalities.
    \begin{proposition}\label{prop: inequality of depth, dimension and dimension of tangent space}
        Let $(A,\frakm)$ be a local noetherian ring with residue field $\kk$.
        Then the following inequalities hold:
        \[ \depth A \leq \dim A \leq \dim_\kk T_{A,\frakm}. \]
    \end{proposition}
    \begin{proof}
        The first inequality is a direct corollary of Corollary \ref{cor: geometric form of principal ideal theorem}.

        Let $t_1,\cdots,t_n$ be a $\rkk(\frakm)$-basis of $\frakm/\frakm^2$.
        Then we have $\frakm/(t_1,\cdots,t_n)+\frakm^2 = 0$, whence $\frakm/(t_1,\cdots,t_n) = \frakm(\frakm/(t_1,\cdots,t_n))$.
        It follows that $\frakm = (t_1,\cdots,t_n)$ by Nakayama's Lemma.
        By Corollary \ref{cor: geometric form of principal ideal theorem}, 
        \[ n + \dim A/(t_1,\cdots,t_n) \geq n-1 + \dim A/(t_1,\cdots,t_{n-1}) \geq \cdots \geq 1 + \dim A/(t_1) \geq \dim A. \]  
        We conclude the result.
    \end{proof}

    % To see these, we need the following well-known theorem.

    \begin{definition}\label{def: regular and Serre's conditions}
        Let $X$ be a locally noetherian scheme and $k \in \bbz_{\geq 0}$.
        We say that \textit{$X$ verifies property $(R_k)$} or \textit{is regular in codimension $k$} if $\forall \xi \in X$ with $\codim Z_\xi \leq k$, 
        \[ \dim_{\rkk(\xi)} T_{X,\xi} = \dim \calo_{X,\xi}. \]
        We say that \textit{$X$ verifies property $(S_k)$} if $\forall \xi \in X$ with $\depth \calo_{X,\xi} < k$,
        \[ \depth \calo_{X,\xi} = \dim \calo_{X,\xi}. \]        
    \end{definition}

    \begin{lemma}\label{lem: trick lemma: subideal in union of prime belongs one}
        Let $A$ be a ring and $\fraka \subset \bigcup_i \frakp_i$.
        Then $\fraka \subset \frakp_i$ for some $i$.
    \end{lemma}
    \begin{proof}
        \Yang{To be completed.}
    \end{proof}

    \begin{example}\label{eg: S_1 is equivalent to A has no embedded point}
        Let $A$ be a noetherian ring.
        Then $A$ verifies $(S_1)$ iff $A$ has no embedded point.
        
        Suppose $A$ verifies $(S_1)$.
        If $\frakp \in Ass A$, every element in $\frakp$ is a zero divisor.
        Then $\depth A_{\frakp} = 0$.
        It follows that $\dim A_\frakp = 0$ and then $\frakp$ is minimal. 

        Suppose $A$ has no embedded point.
        Let $\frakp \in \Spec A$ with $\depth A_\frakp = 0$.
        This means every element in $\frakp A_\frakp$ is a zero divisor.
        Then 
        \[ \frakp \subset \{\text{zero divisors in } A\} = \bigcup_{\text{minimal prime ideals}} \frakq. \]
        By Lemma \ref{lem: trick lemma: subideal in union of prime belongs one}, $\frakp = \frakq$ for some minimal $\frakq$, whence $\dim A_\frakp = 0$. 
    \end{example}

    \begin{example}\label{eg: S_2 is equivalent to A/fA has no embedded point}
        Let $A$ be a noetherian ring verifies $(S_1)$.
        Then $A$ verifies $(S_2)$ iff for any nonzero divisor $f \in A$, $\Ass_A A/fA$ has no embedded point.

        Suppose $A$ verifies $(S_2)$.
        Let $f \in A$ be a nonzero divisor and $\frakp \in \Ass_A A/fA$.
        There exist $g \in A \setminus fA$ such that $\frakp = (f:g)$.
        For any $t_1,t_2 \in \frakp$, there exist $s_1,s_2$ with $s_i \notin (t_i)$ and $t_i g = f s_i$.
        Then $t_1t_2g = fs_1t_2 = fs_2t_1$.
        Since $f$ is not a zero divisor, $s_1t_2 = s_2t_1$.
        Then $t_2$ is a zero divisor in $A_\frakp/t_1 A_\frakp$ since $s_1 \notin (t_1)$.
        Since $f \in \frakp$, $\depth A_\frakp = 1$ and then $\idealht \frakp = 1$.
        This show that $\frakp$ is not embedded in $\Ass_A A/fA$.
        
        Conversely, suppose $\Ass_A A/fA$ has no embedded point.
        Let $\frakp \in \Spec A$ with $\depth A_{\frakp} = 1$.
        Then there exists $f \in A_{\frakp}$ which is not a zero divisor.
        We have $\depth A_{\frakp}/fA_\frakp = 0$ and $\Ass_A A/fA$ has no embedded point, whence $\frakp$ is minimal in $A/fA$.
        Then $\idealht \frakp = 1$ by Krull's Principal Ideal Theorem \ref{thm: Krull's principal ideal theorem} and the fact $f$ is not a zero divisor.
    \end{example}

    \begin{example}\label{eg: R-S criterion of reducedness}
        Let $X$ be a locally noetherian scheme.
        Then $X$ is reduced iff it verifies $(R_0)$ and $(S_1)$.

        The properties are local, whence we can assume $X=\Spec A$.
        Suppose $A$ is reduced.
        Let $\frakp_1,\cdots,\frakp_n$ be all minimal prime ideals of $A$.
        We have $\bigcap \frakp_i = \frakN = (0)$, where $\frakN$ is the nilradical of $A$.
        Hence $A$ has no embedded point.
        Since $A_\frakp$ is artinian, local and reduced, $A_\frakp$ is a field and hence regular.
        
        Conversely, let $\Ass A$ be equal to $\{\frakp_1,\cdots,\frakp_n\}$.
        Then every $\frakp_i$ is minimal by $(S_1)$.
        Let $f$ be in $\frakN$.
        Then the image of $f$ in $A_{\frakp_i}$ is $0$ since by $(R_0)$, $A_{\frakp_i}$ is a field.
        It follows that $f \in \frakq_i$, where $\frakq_i$ is the $\frakp_i$ component of $(0)$ in $A$.
        Hence $f \in \bigcap \frakq_i = (0)$.
        That is, $A$ is reduced.
    \end{example}




\subsection{Normal schemes \Yang{To be completed}}

    % Fix a noetherian local ring $(A,\frakm)$ with residue field $\kk$.

    \begin{definition}\label{def: normal domain}
        An integral domain $A$ is called \textit{normal} if it is integrally closed in its field of fractions $\Frac(A)$.
    \end{definition}

    \begin{lemma}\label{lem: integral closed under localization}
        Let $A \subset C$ be rings and $B$ the integral closure of $A$ in $C$, $S$ a multiplicatively closed subset of $A$.
        Then the integral closure of  $S^{-1}A$ in $S^{-1}C$ is $S^{-1}B$.
    \end{lemma}
    \begin{proof}
            For every $b \in B$ and $\forall s \in S$, there exists $a_i \in A$ s.t. 
            \[ 
                b^n + a_1 b^{n-1} + \cdots + a_n = 0.
            \] 
            Then 
            \[ 
                \left( \frac{b}{s} \right)^n + \frac{a_1}{s^1} \left( \frac{b}{s} \right)^{n-1} + \cdots + \frac{a_n}{s^n} = 0.
            \] 
            Hence $b/s$ is integral over $S^{-1}A$, $S^{-1}B$ is integral over $S^{-1}A$.

            If $c/s \in S^{-1}C$ is integral over $S^{-1}A$, then $\exists a_i \in S^{-1}A$ s.t.
            \[ 
                \left( \frac{c}{s} \right)^n + a_1 \left( \frac{c}{s} \right)^{n-1} + \cdots + a_n = 0.
            \]
            Then 
            \[ 
                c^n + a_1 s c^{n-1} + \cdots + a_n s^n = 0 \in S^{-1}C
            \] 
            Then $\exists t \in S$ s.t. 
            \[
                t (c^n + a_1 s c^{n-1} + \cdots + a_n s^n) = 0 \in C.
            \] 
            Then 
            \[ 
                (ct)^n + a_1 s t (ct)^{n-1} + \cdots + a_n s^n t^n  = t^n (c^n + a_1 s c^{n-1} + \cdots + a_n s^n) = 0.
            \] 
            Hence $ct$ is integral over $A$, then $ct \in B$.
            Then $c/s = (ct)/(st) \in S^{-1}B$.
            This completes the proof.
    \end{proof}

    \begin{proposition}\label{prop: normality is a local property}
        Normality is a local property. 
        That is, for an integral domain $A$, TFAE:
        \begin{enumerate}[label=(\roman*)]
            \item $A$ is normal.
            \item For any prime ideal $\frakp \in \Spec A$, the localization $A_\frakp$ is normal.
            \item For any maximal ideal $\frakm \in \mSpec A$, the localization $A_\frakm$ is normal.
        \end{enumerate}
    \end{proposition}
    \begin{proof}
        When $A$ is normal, $A_\frakp$ is normal by Lemma \ref{lem: integral closed under localization}.

        Assume that $A_\frakm$ is normal for every $\frakm \in \mSpec A$.
        If $A$ is not normal, let $\tilde{A}$ be the integral closure of $A$ in $\Frac A$, $\tilde{A}/A$ is a nonzero $A$-module.
        Suppose $\frakp \in \Supp \tilde{A}/A$ and $\frakp \subset \frakm$.
        We have $\tilde{A}_\frakm/A_\frakm = 0$ and $\tilde{A}_\frakp/A_\frakp = (\tilde{A}_\frakm/A_\frakm)_\frakp \neq 0$.
        This is a contradiction.
    \end{proof}

    % \begin{proposition}
    %     Let $A$ be a normal ring.
    %     Then $A[X]$ is also normal.
    % \end{proposition}

    \begin{definition}\label{def: normal of scheme and general ring}
        A scheme $X$ is called \textit{normal} if the local ring $\calo_{X,\xi}$ is normal for any point $\xi \in X$.
        A ring $A$ is called \textit{normal} if $\Spec A$ is normal.
    \end{definition}

    \begin{remark}\label{rmk: total ring of fractions and normality of general reduced ring}
        For a general ring $A$, let $S := A\setminus (\bigcup_{\frakp \in \Ass A} \frakp) = \bigcap_{\frakp \in \Ass A}A\setminus \frakp$.
        Then $S$ is a multiplicative set.
        The localization $S^{-1}A$ is called \textit{the total ring of fractions} of $A$.
        
        Suppose $A$ is reduced and $\Ass A = \{\frakp_1,\cdots,\frakp_n\}$.
        Denote its total ring of fractions by $Q$.
        Note that elements in $Q$ are either unit or zero divisor.
        Hence any maximal ideal $\frakm$ is contained in $\bigcup \frakp_iQ$, whence contained in some $\frakp_i Q$.
        Thus $\frakp_i Q$ are maximal ideals.
        And we have $\bigcap \frakp_i Q = 0$. 
        By the Chinese Remainder Theorem, we have $Q = \prod Q/{\frakp_i}Q = \prod A_{\frakp_i}$.
        
        Let $A$ be a reduced ring with total ring of fractions $Q$.
        Then $A$ is normal iff $A$ is integral closed in $Q$.
        If $A$ is normal, then for every $\frakp \in \Spec A$, $A_\frakp$ is integral.
        Then there is unique minimal prime ideal $\frakp_i \subset \frakp$.
        In particular, any two minimal prime ideal are relatively prime.
        By the Chinese Remainder Theorem, $A = \prod A/\frakp_i$.
        Just need to check $A/\frakp_i$ is integral closed in $A_{\frakp_i}$.
        This is clear by check pointwise.

        Conversely, suppose $A$ is integral closed in $Q$.
        Let $e_i$ be the unit element of $A_{\frakp_i}$.
        It belongs to $A$ since $e_i^2 - e_i = 0$.
        Since $1 = e_1 + \cdots + e_n$ and $e_ie_j = \delta_{ij}$, we have $A = \prod Ae_i$.
        Since $Ae_i$ is integral closed in $A_{\frakp_i}$, it is normal.
        Hence $A$ is normal.
    \end{remark}

    \begin{definition}
        Let $X$ be a scheme.
        The \textit{normalization} of $X$ is an $X$-scheme $X^\nu$ with the following universal property:
        for any normal $X$-scheme $Y$ with dominant structure morphism, its structure morphism $Y \to X$ factors through $X^\nu$.
    \end{definition}

    \begin{proposition}
        The normalization $X^\nu$ of $X$ exists.
        Moreover, if $X$ is reduced, $X^\nu \to X$ is birational.
    \end{proposition}
    \begin{proof}
        Suppose there is a dominant morphism $Y \to X$ with $Y$ normal.
        Since $Y$ is normal, it is reduced.
        Then it factors through $X_{red}$.
        Hence we can assume that $X$ is reduced by replacing $X$ by $X_{red}$.

        Suppose $X = \Spec A$ is affine.
        Let $A^\nu$ be the integral closure of $A$ in it total ring of fractions and $X^\nu := \Spec A^\nu$.
        It gives a homomorphism $A \to \calo_Y(Y)$.
        We claim that it is injective.
        Otherwise, it factors through $A \to A/I$ and then $Y \to \Spec A$ factors through $\Spec A/I \to \Spec A$.
        It contradicts that $Y \to X$ is dominant.
        Since $Y$ is normal, $\calo_Y(Y)$ is integral closed in its total ring of fraction.
        Then $\calo_Y(Y)$ contains $A^\nu$.
        This shows that $X^\nu$ is the normalization of $X$.

        In general case, take an affine cover $\{U_i\}$ of $X$ and clue these $U_i^\nu$ by universal property.
    \end{proof}

    % \begin{proposition}
    %     Let $S = \Spec \kk$ or $\Spec \calo_K$ and $X$ an $S$-variety.
    %     Then the normalization $X^\nu \to X$ is birational.
    %     In particular, $\{ \xi \in X \colon \calo_{X,\xi} \text{ is normal} \}$ is open in $X$.
    % \end{proposition}

    \begin{lemma}\label{lem: normal rings verify R_1 and S_2}
        Let $A$ be a normal ring.
        Then $A$ verifies $(R_1)$ and $(S_2)$.
    \end{lemma}
    \begin{proof}
        Since all properties are local, we can assume $A$ is integral and local.

        For $(S_2)$, by Example \ref{eg: S_2 is equivalent to A/fA has no embedded point}, we only need to show that $\Ass_A A/f$ has no embedded point.
        Let $\frakp = (f:g) = \in \Ass_A A/fA$ and $t:= f/g \in \Frac A$.
        After Replacing $A$ by $A_\frakp$, we can assume that $\frakp$ is maximal.
        By definition, $t^{-1}\frakp \subset A$.
        If $t^{-1}\frakp \subset \frakp$, suppose $\frakp$ is generated by $(x_1,\cdots,x_n)$ and $t^{-1}(x_1,\cdots,x_n)^T = \Phi(x_1,\cdots,x_n)^T$ for $\Phi \in M_n(A)$.
        There is a monic polynomial $\chi(T) \in A[T]$ vanishing $\Phi$.
        Then $\chi(t^{-1}) = 0$ and $t^{-1} \in A$.
        This is impossible by definition of $t$.
        Then $t^{-1}\frakp = A$, and $\frakp = (t)$ is principal. 
        By Krull's Principal Ideal Theorem \ref{thm: Krull's principal ideal theorem}, $\idealht(\frakp) = 1$.

        Now we show that $A$ verifies $(R_1)$.
        Suppose $(A, \frakm)$ is local of dimension $1$.
        Choosing $a \in \frakm$, $A/a$ is of dimension $0$.
        Then by \ref{prop: characteristic of local artinian rings}, $\frakm^n \subset aA$ for some $n\geq 1$.
        Suppose $\frakm^{n-1} \not\subset aA$.
        Choose $b \in \frakm^{n-1} \setminus aA$ and let $t = a/b$.
        By construction, $t^{-1} \notin A$ and $t^{-1}\frakm \subset A$.
        After similar argument, we see that $\frakm = tA$, whence $A$ is regular.
    \end{proof}

    \begin{lemma}\label{lem: normal and regular are equivalent for noetherian rings of dimension 1}
        Let $(A,\frakm)$ be a noetherian local ring of dimension $1$.
        Then $A$ is normal iff $A$ is regular.
    \end{lemma}
    \begin{proof}
        By lemma \ref{lem: normal rings verify R_1 and S_2}, we just need to show that regularity implies normality.

        Let $t \in \frakm\setminus \frakm^2$.
        Since $A$ is regular, $\frakm = (t)$.
        Let $I \subset \frakm$ be an ideal.
        If $I \subset \bigcap_{n} \frakm^n$, then for every $a \in I$, there exists $a_n$ such that $a = a_n t^n$.
        Then we get an ascending chain of ideals $(a_1) \subset (a_2) \subset \cdots$.
        Hence $a=0$ by Nakayama's Lemma.
        Suppose $I$ is not zero.
        Then there is some $n$ such that $I \subset \frakm^n$ and $I \not\subset \frakm^{n+1}$.
        For every $at^n \in I \setminus \frakm^{n+1}$,  $a \notin \frakm$, whence $a$ is a unit in $A$.
        Then $I = (t^n)$.
        Hence $A$ is PID and hence normal.
    \end{proof}

    \begin{proposition}\label{prop: S_2 implies intersection of localization at height 1 prime is A}
        Let $A$ be a noetherian integral domain of dimension $\geq 1$ verifying $(S_2)$.
        Then 
        \[ A = \bigcap_{\frakp \in \Spec A, \idealht(\frakp) = 1} A_\frakp. \]
    \end{proposition}
    \begin{proof}
        Clearly $A \subset \bigcap A_\frakp$.
        Let $t = f/g \in \bigcap A_\frakp$.
        Since $f\in gA_\frakp$ and we have $g A = \bigcap (gA_\frakp \cap A)$, $f \in gA$.
        It follows that $t \in A$.
    \end{proof}

    \begin{theorem}[Serre's criterion for normality]
        Let $X$ be a locally noetherian scheme.
        Then $X$ is normal if and only if it verifies $(R_1)$ and $(S_2)$.
    \end{theorem}
    \begin{proof}
        One direction has been proved in Lemma \ref{lem: normal rings verify R_1 and S_2}.
        Suppose $X$ verifies $(R_1)$ and $(S_2)$.
        Again we can assume $X = \Spec A$ is affine and $A$ is local.
        By Remark \ref{rmk: total ring of fractions and normality of general reduced ring}, we just need to show that $A$ is integral closed in its total ring of fractions $Q$.
        Suppose we have 
        \[ \left(\frac{a}{b}\right)^n + c_1 \left(\frac{a}{b}\right)^{n-1} + \cdots + c_n = 0 \in Q. \]
        Since $A$ verifies $(S_2)$, $b A = \bigcap \nu_\frakp^{-1}(b_\frakp A_\frakp)$.
        So it is sufficient to show that $a_{\frakp} \in b_\frakp A_\frakp$ with $\idealht(\frakp) = 1$.
        Note that $A_\frakp$ is regular and hence normal by Lemma \ref{lem: normal and regular are equivalent for noetherian rings of dimension 1}.
        Then above equation gives us desired result.
    \end{proof}

    \begin{theorem}\label{thm: extension of function on normal scheme}
        Let $X$ be a normal and locally noetherian scheme.
        Let $F \subset X$ be a closed subset of codimension $\geq 2$.
        Then the restriction $H^0(X,\calo_X) \to H^0(X\setminus F, \calo_{X})$ is an isomorphism.
    \end{theorem}
    \begin{proof}
        By the exact sequences 
        \[ 0 \to \calf(X) \to \prod_i \calf(U_i) \to \prod_{i,j} \calf(U_i \cap U_j), \]
        where $\{U_i\}$ is an affine open cover of $X$, we can reduce to the case that $X$ is affine.
        Then $X = \Spec A$ for some normal noetherian ring $A$.
        For any prime ideal $\frakp \in \Spec A$ with $\idealht(\frakp) = 1$, we have $\frakp \in X \setminus F$.
        By Proposition \ref{prop: S_2 implies intersection of localization at height 1 prime is A}, the conclusion follows.
    \end{proof}

    \begin{theorem}[Valuation criterion for properness]\label{thm: valuation criterion for properness}
        Let $f:X \to Y$ be a morphism of finite type between noetherian schemes.
        Then $f$ is proper iff for any valuation ring $A$, $\KK = \Frac A$ and commutative diagram
        \[ \xymatrix{
            \Spec \KK \ar[r] \ar[d] & X \ar[d]^f \\
            \Spec A \ar[r] & Y
        }, \]
        the morphism $\Spec A \to Y$ factors through $f$ uniquely.
    \end{theorem}

    \begin{proposition}\label{prop: morphism defined on a generic point extends to a open subset}
        Let $X,Y$ be $S$-schemes with $S$ locally noetherian.
        Suppose $Y$ is of finite type over $S$.
        Let $\xi \in X$ and $f_x: \Spec \calo_{X,\xi} \to Y$ be a morphism.
        Then there exists an open subset $U \subset X$ containing $\xi$ such that the morphism extends to a morphism $U \to Y$.
    \end{proposition}
    \begin{proof}
        Replacing $S,X,Y$ by affine open neighborhoods of images of $\xi$, we can assume that $S = \Spec A$, $X=\Spec B$ and $Y = \Spec A[T_1,\cdots,T_n]/I$ are affine.
        Then we get a homomorphism $A[T_1,\cdots,T_n]/I \to B_{\xi}$ of $A$-algebra.
        Denote the image of $T_i$ by $f_i/g_i$ in $B_{\xi}$, where $f_i,g_i \in B$.
        Then above homomorphism factors through $B[1/g_1,\cdots,1/g_n] \to B_{\xi}$.
        Let $U$ be the open subset of $X$ defined by $g_1\cdots g_n \neq 0$.
        Then the morphism $f_x$ extends to a morphism $U \to Y$.
    \end{proof}

    \begin{theorem}\label{thm: extension of morphism form normal to proper}
        Let $X,Y$ be $S$-schemes of finite type with $S$ noetherian.
        Suppose $X$ is normal, and $Y$ is proper over $S$.
        Let $f:X \ratmap Y$ be a rational map.
        Then $f$ is well-defined on an open subset $U \subset X$ whose complement has codimension $\geq 2$.
    \end{theorem}
    \begin{proof}
        We can assume that $X$ is irreducible and hence integral.
        Suppose $f$ is defined on $U \subset X$.
        For every $\xi \in X$ with codimension $1$, we have following commutative diagram
        \[ \xymatrix{
            \Spec \fkk(X) \ar[r] \ar[d] & U \ar[r]^f & Y \ar[d] \\
            \Spec \calo_{X,\xi} \ar[rr] & & S
        }, \]
        By Theorem \ref{thm: valuation criterion for properness} and Proposition \ref{prop: morphism defined on a generic point extends to a open subset}, there exists an open subset $U_\xi \subset X$ containing $\xi$ such that the morphism extends to a morphism $U_\xi \to Y$.

        \Yang{To be completed.}
    \end{proof}

    \begin{remark}\label{rmk: extension of function on normal scheme and extension of morphism form normal to proper}
        Theorem \ref{thm: extension of function on normal scheme} and Theorem \ref{thm: extension of morphism form normal to proper} are very similar.
        However, they are base on different properties.
        Theorem \ref{thm: extension of function on normal scheme} is based on $(S_2)$, while Theorem \ref{thm: extension of morphism form normal to proper} is based on $(R_1)$.
        Philosophically, the $(S_k)$ conditions are used to control the ``bad part of codimension larger than $k$''.
        The $(R_k)$ conditions are used to control the ``bad part of codimension smaller than or equal to $k$''.
        We will see more examples in the next section.\Yang{To be completed.}
    \end{remark}




\subsection{Cohen-Macaulay schemes}

    \begin{definition}[Cohen-Macaulay]\label{def: Cohen-Macaulay}
        A noetherian local ring $(A,\frakm)$ is called \textit{Cohen-Macaulay} if $\dim A = \depth A$.
        A locally noetherian scheme $X$ is called \textit{Cohen-Macaulay} if $\calo_{X,\xi}$ is Cohen-Macaulay for any point $\xi \in X$.
    \end{definition}

    By definition, it is easy to see that $X$ is Cohen-Macaulay if and only if it verifies $(S_k)$ for all $k \geq 0$.

    \begin{example}[Non Cohen-Macaulay rings]

    \end{example}

    \begin{proposition}\label{prop: depth equals to grade}
        Let $(A, \frakm, \kk)$ be a noetherian local ring and $M$ a finite $A$-module.
        Then 
        \[ \depth M := \inf \{ i : \Ext^i_A(\kk,M) \neq 0 \}. \]
    \end{proposition}
    \begin{proof}
        Let $a \in \frakm$ be $M$-regular and $N = M/aM$.
        Then we claim that
        \[ \inf \{ i: \Ext^i_A(\kk,N) \neq 0 \} = \inf \{ i : \Ext^i_A(\kk,M) \neq 0 \} - 1. \]
        Indeed, we have an exact sequence
        \[ 0 \to M \xrightarrow{a} M \to N \to 0. \]
        It induces a long exact sequence
        \[ \cdots \to \Ext^{i-1}_A(\kk,M) \to \Ext^{i-1}_A(\kk,N) \to \Ext^i_A(\kk,M) \xrightarrow{\Ext^i_A(\kk,\Mult_a)} \Ext^i_A(\kk,M) \to \cdots. \]
        Note that $a \in \frakm$, then $\Ext^i_A(\kk,\Mult_a) = 0$.
        It follows that when $\Ext^{i-1}_A(\kk,M) = 0$, we have $\Ext^{i-1}_A(\kk,N) = 0$ iff $\Ext^i_A(\kk,M) = 0$, whence the claim.

        Let $n = \inf \{ i : \Ext^i_A(\kk,M) \neq 0 \}$.
        Induct on $n$.
        Suppose first $n = 0$.
        Since $\kk$ is a simple $A$-module, there is an injective homomorphism $\kk \to M$.
        Then $\frakm \in \Ass M$ and hence $\depth M = 0$.
        
        Suppose $n > 0$., let $a_1,\cdots,a_m \in \frakm$ be any $M$-regular sequence.
        Using the claim inductively on $M/(a_1,\cdots,a_m)M$, we have $n \geq \depth$.
        If $M$ has no regular element, then $\frakm \subset \bigcup_{\frakp \in \Ass M} \frakp$.
        Then $\frakm = \frakp$ for some $\frakp \in \Ass M$.
        This show that we can find $x \neq 0 \in M$ such that $\frakp = \Ann x$.
        It gives a homomorphism $\kk = A/\frakm \to M$.
        That is a contradiction and hence $M$ has a regular element.
        Let $a$ be $M$-regular and $N = M/aM$.
        Then $\depth N = n-1$ by the claim and induction hypothesis.
        Hence we have $\depth M \geq n$.
    \end{proof}

    \begin{corollary}\label{cor: induction on depth}
        Let $A$ be a noetherian ring, $M$ a finite $A$-module and $a \in A$ an $M$-regular element.
        Then $\depth M = \depth M/aM + 1$.
    \end{corollary}

    \begin{corollary}\label{cor: induction on S_d conditions}
        Let $A$ be a noetherian ring $a \in A$ a nonzero divisor.
        Then $A$ verifies $(S_d)$ iff $A/aA$ verifies $(S_{d-1})$.
    \end{corollary}

    \begin{definition}\label{def: unmixedness theorem}
        An ideal $I$ of a noetherian ring $A$ is called \textit{unmixed} if 
        \[ \idealht(I) = \idealht(\frakp), \quad \forall \frakp \in \Ass(A/I). \]
        Here $\idealht(I)$ is defined as 
        \[ \idealht(I) := \inf \{ \idealht(\frakp) : I \subset \frakp \}. \]
        We say that \textit{the unmixedness theorem holds for a noetherian ring $A$} if any ideal $I \subset A$ generated by $\idealht(I)$ elements is unmixed.
        We say that \textit{the unmixedness theorem holds for a locally noetherian scheme $X$} if $\calo_{X,\xi}$ is unmixed for any point $\xi \in X$.
    \end{definition}


    \begin{theorem}\label{thm: unmixedness theorem for Cohen-Macaulay schemes}
        Let $X$ be a locally noetherian scheme.
        Then the unmixedness theorem holds for $X$ if and only if $X$ is Cohen-Macaulay.
    \end{theorem}
    \begin{proof}
        We can assume that $X = \Spec A$ is affine.

        Suppose $X$ is Cohen-Macaulay.
        Let $I \subset A$ be an ideal generated by $a_1,\cdots,a_r$ with $r = \idealht(I)$.
        We claim that $a_1,\cdots,a_r$ is an $A$-regular sequence.
        If so, we get that the unmixedness theorem holds for $A$ by applying Example \ref{eg: S_1 is equivalent to A has no embedded point} on $A/I$. 
        Since $\idealht(a_1,\cdots,a_{r-1}) \leq r-1$ by Krull's Principal Ideal Theorem \ref{thm: Krull's principal ideal theorem} and $\idealht(a_1,\cdots,a_r) = r \leq \idealht(a_1,\cdots,a_{r-1}) + 1$, we have $\idealht(a_1,\cdots,a_{r-1}) = r-1$.
        By induction on $r$, we can assume that $a_1,\cdots,a_{r-1}$ is an $A$-regular sequence.
        Hence any prime ideal $\frakp \in \Ass A/(a_1,\cdots,a_{r-1})$ has height $r-1$.
        Now suppose $a_r$ is a zero divisor in $A/(a_1,\cdots,a_{r-1})$.
        Then there exists a prime ideal $\frakp \in \Ass A/(a_1,\cdots,a_{r-1})$ such that $a_r \in \frakp$.
        Then $I \subset \frakp$ and $\idealht(I) \leq r-1$.
        This contradicts that $\idealht(I) = r$.
        
        Suppose the unmixedness theorem holds for $A$.
        Let $\frakp \in \Spec A$ be a prime ideal with $\idealht(\frakp) = r$.
        Then $\frakp \in \Ass A$ if and only if $\idealht(\frakp) = 0$.
        If $r > 0$, there is a nonzero divisor $a \in \frakp$.
        By Krull's Principal Ideal Theorem \ref{thm: Krull's principal ideal theorem}, $\idealht(\frakp A/aA) = r-1$.
        Inductively, we can find a regular sequence $a_1,\cdots,a_r$ in $\frakp$.
        Then $\depth A_\frakp = r$.
        % \Yang{To be completed.}
    \end{proof}

    \begin{theorem}\label{thm: higher dimensional hartogs lemma for Cohen-Macaulay schemes}
        Let $X$ be a locally noetherian scheme.
        Suppose that $X$ is Cohen-Macaulay.
        Let $F \subset X$ be a closed subset of codimension $\geq k$.
        Then the restriction $H^i(X,\calo_X) \to H^i(X\setminus F, \calo_{X})$ induced by the 
         is an isomorphism.
    \end{theorem}
    \begin{proof}
        \Yang{To be completed.}
    \end{proof}


    
\subsection{Regular schemes}

    % \begin{proposition}
    %     Let $(A,\frakm)$ be a regular local ring.
    %     Then $A$ is integral.
    % \end{proposition}

    \begin{proposition}
        If $X$ verifies $(R_k)$, then $\codim_X X_{\text{sing}} \geq k+1$.
    \end{proposition}

    \begin{proposition}
        A regular scheme is Cohen-Macaulay.
    \end{proposition}

    \begin{corollary}
        A regular scheme is normal.
    \end{corollary}



