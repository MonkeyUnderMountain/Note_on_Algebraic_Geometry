\section{Elementary Results}

\Yang{To be completed}

\subsection{Notations}

    In the appendix and all the note, the ``ring'' is always commutative and with identity.
    We denote by \(\Spec A\) the set of prime ideals of a ring \(A\).
    We denote by \(\mSpec A\) the set of maximal ideals of \(A\).
    Let \(I \subset A\) be an ideal of \(A\).
    We define 
    \[ V(I) \coloneqq \{ \frakp \in \Spec A \colon I \subset \frakp\}. \]

    Let \(\fraka,\frakb\) be ideals of \(A\).
    We define 
    \[ (\fraka : \frakb) \coloneqq \{ a \in A \colon a\frakb \subset \fraka\}. \]
    This is an ideal of \(A\).
    
    Let \(\rad(A)\) be the Jacobian radical of \(A\), i.e., the intersection of all maximal ideals of \(A\).
    Let \(\nil(A)\) be the nilradical of \(A\), i.e., the ideal of \(A\) consisting of all nilpotent elements.
    
    \begin{proposition}
        Let \(A\) be a ring.
        Then we have 
        \[ \nil(A) = \bigcap_{\frakp \in \Spec A} \frakp. \]
    \end{proposition}
    \begin{proof}
        \Yang{To be completed.}
    \end{proof}

    \begin{proposition}\label{prop: prime avoidance lemma primity of prime ideals}
        Let $A$ be a ring, $\frakp,\frakp_i$ prime ideals of $A$ and \(\fraka,\fraka_i\) ideals of $A$. 
        \begin{enumerate}
            \item Suppose \(\fraka \subset \bigcup_{i=1}^n \frakp_i\). 
            Then there exists \(i\) such that \(\fraka \subset \frakp_i\).
            \item Suppose \(\bigcap_{i=1}^n \fraka_i \subset \frakp\). 
            Then there exists \(i\) such that \(\fraka_i \subset \frakp\).
        \end{enumerate}
    \end{proposition}
    \begin{proof}
        \Yang{To be completed.}
    \end{proof}

    Let \(M\) be an \(A\)-module.
    We say that \(M\) is \emph{finite} if there exists an exact sequence
    \[ A^n \to M \to 0. \]
    We say that \(M\) is \emph{coherent} if there exists an exact sequence
    \[ A^m \to A^n \to M \to 0. \]
    If \(A\) is a noetherian ring, then every finite \(A\)-module is coherent.

    \begin{definition}\label{def: support of a module}
        Let \(A\) be a ring and \(M\) an \(A\)-module.
        The \emph{support} of \(M\) is defined as
        \[
            \Supp M \coloneqq \{ \frakp \in \Spec A \colon M_\frakp \neq 0\}.
        \]
    \end{definition}

    The \emph{annihilator} of \(M\) is defined as
    \[ \Ann M \coloneqq \{ a \in A \colon aM = 0\}. \]
    This is an ideal of \(A\).

    \begin{proposition}\label{prop: support of a finite module}
        Let \(A\) be a ring and \(M\) a finite \(A\)-module.
        Then \(\Supp M = V(\Ann M)\).
        In particular, \(\Supp M\) is a closed subset of \(\Spec A\).
    \end{proposition}
    \begin{proof}
        \Yang{To be completed.}
    \end{proof}

    \begin{definition}\label{def: localization}
        Let \(A\) be a ring and \(S \subset A\) a multiplicative subset, i.e., \(1 \in S\) and \(s_1,s_2 \in S\) implies \(s_1 s_2 \in S\).
        The \emph{localization} of \(A\) at \(S\) is defined as
        \[ S^{-1}A \coloneqq A \times S / \sim, \]
        where \((a,s) \sim (b,t)\) if there exists \(u \in S\) such that \(u(at - bs) = 0\).
        \Yang{To be completed.}
    \end{definition}

    \begin{proposition}\label{prop: when is localization injective}
        
    \end{proposition}

\subsection{Nakayama's Lemma}

    \begin{theorem}[Nakayama's Lemma]\label{thm: Nakayama's lemma}
        Let $A$ be a ring and $\frakM$ be its Jacobi radical.
        Suppose $M$ is a finitely generated $A$-module.
        If $\fraka M = M$ for $\fraka \subset \frakM$, then $M = 0$.
    \end{theorem}
    \begin{proof}
        Suppose $M$ is generated by $x_1,\cdots,x_n$.
        Since $M = \fraka M$, formally we have $(x_1,\cdots,x_n)^T = \Phi (x_1,\cdots, x_n)^T$ for $\Phi \in M_n(\fraka)$.
        Then $(\Phi - \id) (x_1,\cdots, x_n)^T = 0$. 
        Note that $\det (\Phi - \id) = 1+a$ for $a \in \fraka \subset \frakM$.
        Then $\Phi-\id$ is invertible and then $M=0$.
    \end{proof}

    \begin{remark}\label{rem: counterexample of Nakayama's lemma when M is not finite}
        The finiteness of $M$ is crucial in Nakayama's Lemma.
        For example, let \(\overline{\bbz}\) be the ring of algebraic integers in \(\overline{\bbq}\).
        Choose a non-zero prime ideal \(\frakp\) of \(\overline{\bbz}\).
        Then we have that \(\frakp \overline{\bbz}_\frakp= \frakp^2 \overline{\bbz}_\frakp\).
        Indeed, if \(a \in \frakp \overline{\bbz}_\frakp\), let \(b = \sqrt{a} \in \overline{\bbz}_\frakp\).
        Then \(b^2 = a \in \frakp \overline{\bbz}_\frakp\) and whence \(b \in \frakp \overline{\bbz}_\frakp\) since \(\frakp\) is prime.
        It follows that \(a = b^2 \in \frakp^2 \overline{\bbz}_\frakp\).
    \end{remark}

    \begin{proposition}[Geometric form of Nakayama's Lemma]\label{prop: geometric form of Nakayama's lemma}
        Let $X = \Spec A$ be an affine scheme, $x\in X$ a closed point and $\calf$ a coherent sheaf on $X$.
        If $a_1,\cdots,a_k \in \calf(X)$ generate $\calf|_x = \calf \ten \rkk(x)$, then there is an open subset $U \subset X$ such that $a_i|_U$ generate $\calf(U)$. 
    \end{proposition}
    \begin{proof}
        \Yang{To be completed.}
    \end{proof}

    \begin{corollary}\label{cor: upper semicontinuity of dimension of restriction of coherent sheaf to fiber}
        Let \(X\) be a scheme and \(\calf\) a coherent sheaf on \(X\).
        Then the function \(x \mapsto \dim_{\rkk(x)} \calf|_x\) is upper semicontinuous.
    \end{corollary}
    \begin{proof}
        \Yang{To be completed.}
    \end{proof}

\subsection{Nullstellensatz}

    \begin{theorem}[Noether's Normalization Lemma]\label{thm: Noether's Normalization Lemma}
        Let $A$ be a $\kk$-algebra of finite type.
        Then there is an injection $\kk[T_1,\cdots,T_d] \injmap A$ such that $A$ is finite over $\kk[T_1,\cdots,T_d]$.
    \end{theorem}

    \begin{remark}
        Here $A$ does not need to be integral. 
        For example, 
    \end{remark}

    \begin{theorem}[Hilbert's Nullstellensatz]\label{thm: Nullstellensatz}
        Let $A$ be a 
    \end{theorem}
