\section{Regularity and Smoothness}

\subsection{Modules of differentials and derivations}

    In this subsection, let $R$ be a ring and $A$ an $R$-algebra.

    \begin{definition}[Derivation]\label{def: derivation}
        A \emph{derivation} of $A$ over $R$ is an $R$-linear map $\P: A \to M$ with an $A$-module such that for all $a, b \in A$, we have
        \[
            \P(ab) = a\P(b) + b\P(a).
        \]
        Given the module $M$, the set of all derivations of $A$ over $R$ into $M$ forms an \(A\)-module, denoted by $\Der_R(A, M)$.
    \end{definition}

    Given a module homomorphism \(f: M \to N\) of \(A\)-modules and a derivation \(\P \in \Der_R(A, M)\), the map $f \circ \P$ is a derivation of \(A\) over \(R\) into \(N\).
    % Hence we get a functor 
    % \[
    %     \Der_R(A, -): \Mod_A \to \Mod_A.
    % \]

    \begin{proposition}\label{prop: module of differentials}
        The functor \(\Der_R(A, -)\) is representable.
        The representing object is denoted by \(\Omega_{A/R}\), which is called the \emph{module of differentials} of \(A\) over \(R\).
    \end{proposition}
    \begin{proof}
        \Yang{To be completed.}
    \end{proof}

    \begin{proposition}\label{prop: module of differentials is stable under base change}
        Let \(A, R'\) be \(R\)-algebras and \(A':= A\ten_R R'\).
        Then the module of differentials \(\Omega_{A'/R'}\) is isomorphic to \(\Omega_{A/R} \ten_A A'\).
    \end{proposition}
    \begin{proof}
        \Yang{To be completed.}
    \end{proof}

    \begin{proposition}\label{prop: module of differentials is finite}
        Suppose \(A\) is of finite type over \(R\).
        Then the module of differentials \(\Omega_{A/R}\) is a finitely generated \(A\)-module.
    \end{proposition}
    \begin{proof}
        \Yang{To be completed.}
    \end{proof}

    \begin{theorem}\label{thm: the first exact sequence of differentials}
        Let \(A\) be an \(R\)-algebra and \(B\) an \(A\)-algebra. 
        Then there is a short exact sequence
        \[ \Omega_{A/R} \ten_A B \to \Omega_{B/R} \to \Omega_{B/A} \to 0. \]
    \end{theorem}
    \begin{proof}
        \Yang{To be completed.}
    \end{proof}

    \begin{theorem}\label{thm: the second exact sequence of differentials}
        Let \(A\) be an \(R\)-algebra and \(I\) an ideal of \(A\).
        Then there is a short exact sequence
        \[ I/I^2 \to \Omega_{A/R} \ten_A A/I \to \Omega_{(A/I)/R} \to 0. \]
    \end{theorem}
    \begin{proof}
        \Yang{To be completed.}
    \end{proof}

\subsection{Zariski's tangent space}

    \begin{definition}\label{def: Zariski's tangent space}
        Let \(A\) be a noetherian ring.
        For every $\frakp \in \Spec A$, \(\frakp/\frakp^2\) is a vector space over \(\rkk(\frakp)\).
        The \emph{Zariski's tangent space} $T_{A,\frakp}$ of \(A\) at \(\frakp\) is defined as the dual \(\rkk(\frakp)\)-vector space of \(\frakp/\frakp^2\).
    \end{definition}

    \begin{definition}\label{def: regular ring}
        A noetherian ring \(A\) is said to be \emph{regular} if for every prime ideal \(\frakp \in \Spec A\), we have 
        \[ \dim_{\rkk(\frakp)} T_{A,\frakp} = \dim A_\frakp, \]
        where \(\dim A_\frakp\) is the Krull dimension of the local ring \(A_\frakp\).
    \end{definition}

    \begin{proposition}\label{prop: regularity is a local property}
        Regularity is a local property, i.e., TFAE:
        \begin{enumerate}
            \item \(A\) is regular;
            \item for every prime ideal \(\frakp \in \Spec A\), the local ring \(A_\frakp\) is regular;
            \item for every maximal ideal \(\frakm \in \mSpec A\), the local ring \(A_\frakm\) is regular.
        \end{enumerate}
    \end{proposition}
    \begin{proof}
        \Yang{To be completed.}
    \end{proof}

    \begin{proposition}\label{prop: smoothness is geometric regularity}
        
    \end{proposition}

    \begin{example}\label{eg: regular but not smooth}
    \end{example}


\subsection{Jacobiian criterion}

