\section{Formal Completion}

\subsection{Formal completion of rings and modules}

    \begin{definition}\label{def: topological ring and topological modules}
        Let \(A\) be a ring and \(\mathcal{T}\) a topology on \(A\).
        We say that \((A, \mathcal{T})\) is a \emph{topological ring} if the operations of addition and multiplication are continuous with respect to the topology \(\mathcal{T}\).
        
        Given a topological ring \(A\).
        A \emph{topological \(A\)-module} is a pair \((M, \mathcal{T}_M)\) where \(M\) is an \(A\)-module and \(\mathcal{T}_M\) is a topology on \(M\) such that the addition and scalar multiplication is continuous.
        The morphisms of topological \(A\)-modules are the continuous \(A\)-linear maps.
        They form a category denoted by \(\TopMod_A\).
    \end{definition}

    \begin{definition}\label{def: I-adic topology}
        Let \(A\) be a ring and \(I\) an ideal of \(A\). 
        The \emph{\(I\)-adic topology} on \(A\) is the topology defined by the basis of open sets \(a + I^n\) for all \(n \geq 0\).

        A sequence \(\{a_n\}\) in \(A\) is said to \emph{converge to \(a \in A\)} if for every \(n\), there exists \(N\) such that for all \(m \geq N\), we have \(a_m - a \in I^n\).

        A sequence \(\{a_n\}\) in \(A\) is said to be \emph{Cauchy} if for every \(n\), there exists \(N\) such that for all \(m, k \geq N\), we have \(a_m - a_k \in I^n\).
    \end{definition}

    \begin{definition}[Formal Completion]\label{def: formal completion}
        Let \(A\) be a ring and \(I\) an ideal of \(A\). 
        The \emph{formal completion} of \(A\) with respect to \(I\), denoted by \(\widehat{A}\), is defined as 
        \[ \widehat{A} := \varprojlim ( \cdots \to A/I^n \to A/I^{n-1} \to \cdots \to A/I), \]
        where the maps are the natural projections \(A/I^n \to A/I^{n-1}\).

        Let \(M\) be a \(A\)-module.
        The \emph{formal completion} of \(M\) with respect to \(I\), denoted by \(\widehat{M}\), is defined as
        \[ \widehat{M} := \varprojlim ( \cdots \to M/I^n M \to M/I^{n-1} M \to \cdots \to M/I M), \]
        where the maps are the natural projections \(M/I^n M \to M/I^{n-1} M\).
    \end{definition}

    By the universal property of the inverse limit, 
    we get a covariant functor from the category of \(A\)-modules to the category of topological \(\widehat{A}\)-modules, 
    which sends an \(A\)-module \(M\) to \(\widehat{M}\) and a morphism \(f: M \to N\) to the induced morphism \(\widehat{f}: \widehat{M} \to \widehat{N}\).

    \begin{lemma}\label{lem: completion is exact}
        The functor of completion with respect to an ideal is exact.
        \Yang{finite}.
    \end{lemma}
    \begin{proof}
        \Yang{To be completed.}
    \end{proof}

    \begin{proposition}\label{prop: completion is complete}
        The formal completion \(\widehat{A}\) of a ring \(A\) with respect to an ideal \(I\) is a complete topological ring with respect to the \(I\)-adic topology.
        That is, every Cauchy sequence in \(\widehat{A}\) converges to an element in \(\widehat{A}\).
    \end{proposition}
    \Yang{To be completed.}

    \begin{lemma}\label{lem: completion is isomorphic to image of power series}
        Let \(\widehat{A}\) be the formal completion of a noetherian ring \(A\) with respect to an ideal \(I\). 
        Suppose that \(I\) is generated by \(a_1,...,a_n\). 
        Then we have an isomorphism of topological rings
        \[ \widehat{A} \cong A[[X_1, \ldots, X_n]]/(X_1-a_1, \cdots, X_n-a_n). \]
    \end{lemma}
    \begin{proof}
        \Yang{To be completed.}
    \end{proof}

    \begin{proposition}\label{prop: completion is noetherian}
        Let \(A\) be a noetherian ring and \(I\) an ideal of \(A\). 
        Then the formal completion \(\widehat{A}\) of \(A\) with respect to \(I\) is a noetherian ring.
    \end{proposition}
    \begin{proof}
        \Yang{To be completed.}
    \end{proof}

    \begin{proposition}\label{prop: completion is flat}
        Let \(A\) be a noetherian ring and \(I\) an ideal of \(A\). 
        Then the formal completion \(\widehat{A}\) of \(A\) with respect to \(I\) is a flat \(A\)-module.
    \end{proposition}
    \begin{proof}
        \Yang{To be completed.}
    \end{proof}

    \begin{proposition}\label{prop: completion and tensor product}
        Let \(\widehat{A}\) be completion of a noetherian ring \(A\) with respect to an ideal \(I\) and \(M\) a finite \(A\)-module. 
        Then the natural map \(M \ten_A \widehat{A} \to \widehat{M}\) is an isomorphism.
    \end{proposition}
    \begin{proof}
        \Yang{To be completed.}
    \end{proof}

    \begin{theorem}[Artin-Rees Lemma]\label{thm: Artin Rees Lemma}
        Let \(A\) be a noetherian ring, \(I\) an ideal of \(A\), \(M\) a finite \(A\)-module and \(N\) a submodule of \(M\). 
        Then there exists an integer \(N\) such that for all \(n \geq 0\), we have
        \[ (I^{N+n}M) \cap N = I^n (I^N M \cap N). \]
    \end{theorem}
    \begin{proof}
        \Yang{To be completed.}
    \end{proof}

    \begin{proposition}\label{prop: completion with respect to maximal ideal is local}
        Let \(A\) be a noetherian ring and \(\frakm\) a maximal ideal of \(A\). 
        Then the formal completion \(\widehat{A}\) of \(A\) with respect to \(\frakm\) is a local ring with maximal ideal \(\frakm \widehat{A}\).
    \end{proposition}
    \begin{proof}
        \Yang{To be completed.}
    \end{proof}

\subsection{Complete local rings}

    \begin{definition}[Coefficient rings]\label{def: coefficient rings}
        
    \end{definition}

    \begin{theorem}[Weierstrass Preparation Theorem]\label{thm: Weierstrass Preparation Theorem}
        Let \((A,\frakm)\) be a noetherian complete local ring, \(f = \sum_{n=0}^\infty a_n X^n \in A[[X]]\) a power series with \(f \not\equiv 0 \mod \frakm\).
        Then there exists a unique factorization of the form \(f = u g\), where \(u\) is a unit in \(A[[X]]\) and \(g\) is a polynomial of the form
        \[ g = X^d + b_{d-1} X^{d-1} + \cdots + b_0, \]
        where \(b_i \in \frakm\) for all \(i\).
    \end{theorem}

    \begin{theorem}[Hensel's Lemma]\label{thm: Hensel's Lemma}
        Let \((A,\frakm, \kk)\) be a noetherian complete local ring, 
        % and let \(f \in A[[X]]\) be a power series such that \(f(0) = 0\) and \(f'(0) \not\in \frakm\).
        % Then there exists a unique \(x \in A\) such that \(f(x) = 0\).
    \end{theorem}

    \begin{theorem}[Cohen Structure Theorem]\label{thm: Cohen Structure Theorem}
        Let \(A,\frakm,\kk\) be a noetherian complete local ring with coefficient field \(\kk\).
        Then 
        \begin{enumerate}
            \item If \(A\) is regular of dimension \(d\), then \(A \cong \kk[[X_1, \ldots, X_d]]\).
        \end{enumerate}
    \end{theorem}


\subsection{Unique factorization of regular local rings}




