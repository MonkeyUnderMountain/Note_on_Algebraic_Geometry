\section{Formal Completion}

\subsection{Formal completion of rings and modules}

    \begin{definition}\label{def: topological ring and topological modules}
        Let \(A\) be a ring and \(\mathcal{T}\) a topology on \(A\).
        We say that \((A, \mathcal{T})\) is a \emph{topological ring} if the operations of addition and multiplication are continuous with respect to the topology \(\mathcal{T}\).
        
        Given a topological ring \(A\).
        A \emph{topological \(A\)-module} is a pair \((M, \mathcal{T}_M)\) where \(M\) is an \(A\)-module and \(\mathcal{T}_M\) is a topology on \(M\) such that the addition and scalar multiplication is continuous.
        The morphisms of topological \(A\)-modules are the continuous \(A\)-linear maps.
        They form a category denoted by \(\TopMod_A\).
    \end{definition}

    \begin{definition}\label{def: I-adic topology}
        Let \(A\) be a ring, \(I\) an ideal of \(A\) and \(M\) an \(A\)-module. 
        The \emph{\(I\)-adic topology} on \(M\) is the topology defined by the basis of open sets \(x + I^kM\) for all \(x \in M, k \geq 0\).
        % A sequence \(\{a_n\}\) in \(M\) is said to \emph{converge to \(a \in M\)} if for every \(n\), there exists \(N\) such that for all \(m \geq N\), we have \(a_m - a \in I^nM\).
        % A sequence \(\{a_n\}\) in \(A\) is said to be \emph{Cauchy} if for every \(k\), there exists \(N\) such that for all \(n,m \geq N\), we have \(a_n - a_m \in I^k\).
    \end{definition}

    \begin{example}\label{eg: p-adic topology on Z}
        Let \(A = \bbz\) be the ring of integers and \(p\) a prime number.
        The \(p\)-adic topology on \(\bbz\) is defined by the metric
        \[ d(x, y) \coloneqq \|x-y\|_p \coloneqq p^{-v(x-y)}, \]
        where \(v\) is the valuation defined by the ideal \(p\bbz\).
    \end{example}

    Note that for \(I\)-adic topology, any homomorphism \(f: M \to N\) of \(A\)-modules is continuous since \( f(x + I^kN) \subset f(x) + I^kM \) for all \(x \in M\) and \(k \geq 0\).
    Hence the forgotten functor \(\TopMod_A \to \Mod_A\) gives an equivalence of categories.

    Let \(M\) be an \(A\)-module equipped with the \(I\)-adic topology.
    Note that \(M\) is Hausdorff as a topological space if and only if \(\bigcap_{n \geq 0} I^nM = \{0\}\).
    In this case, we say that \(M\) is \emph{\(I\)-adically separated}.
    
    When \(M\) is \(I\)-adically separated, we can see that \(M\) is indeed a metric space.
    Fix \(r \in (0,1)\). 
    For every \(x\neq y \in M\), there is a unique \(k \geq 0\) such that \(x - y \in I^kM\) but \(x - y \notin I^{k+1}M\).
    We can define a metric on \(M\) by
    \[ d(x, y) \coloneqq r^k. \]
    This metric induces the \(I\)-adic topology on \(M\).
    
    To analyze the \(I\)-adic separation property of \(M\), the following Artin-Rees Lemma is particularly useful.

    \begin{theorem}[Artin-Rees Lemma]\label{thm: Artin Rees Lemma}
        Let \(A\) be a noetherian ring, \(I\) an ideal of \(A\), \(M\) a finite \(A\)-module and \(N\) a submodule of \(M\). 
        Then there exists an integer \(r\) such that for all \(n \geq 0\), we have
        \[ (I^{r+n}M) \cap N = I^n (I^r M \cap N). \]
    \end{theorem}
    \begin{proof}
        Let
        \[ A' := A \oplus I X \oplus I^2 X^2 \oplus \cdots \subset A[X]\] 
        be a graded \(A\)-algebra.
        Note that if \(I = (a_1,\ldots,a_k)\), then \(A' = A[a_1X,\ldots,a_kX]\).
        Hence \(A'\) is a noetherian ring.
        Let 
        \[ M' := M \oplus I M X \oplus I^2 M X^2 \oplus \cdots\]
        be a graded \(A'\)-module.
        Then \(M'\) is a finite \(A'\)-module since it is generated by \(M\) and \(M\) is finite over \(A\).
        Let 
        \[ N' \coloneqq N \oplus (IM \cap N) X \oplus (I^2 M \cap N) X^2 \oplus \cdots\]
        be a graded submodule of \(M'\).
        Then \(N'\) is finite over \(A'\).
        Suppose \(N' = \sum A'x_i\) with \(x_i \in I^{d_i}M \cap N\).
        Choose \(r \geq d_i\) for all \(i\).
        Then the degree \(n+r\) part of \(N'\) is equal to degree \(n\) part of \(A'\) timing the degree \(r\) part of \(N'\).
        That is, for all \(n \geq 0\), \(I^{n+r}M \cap N = I^n (I^r M \cap N)\).
    \end{proof}

    \begin{corollary}\label{cor: subspace topology coincides with I-adic topology}
        Let \(A\) be a noetherian ring, \(I\) an ideal of \(A\), \(M\) a finite \(A\)-module and \(N\) a submodule of \(M\).
        Then the subspace topology on \(N\) induced by \(N \subset M\) coincides with the \(I\)-adic topology on \(N\).
    \end{corollary}
    \begin{proof}
        This is a direct consequence of the Artin-Rees Lemma.
    \end{proof}

    \begin{corollary}\label{cor: finite module over noetherian ring is I-adically separated}
        Let \(A\) be a noetherian ring, \(I\) an ideal of \(A\), and \(M\) a finite \(A\)-module. 
        Let \(N = \bigcap_{n \geq 0} I^n M\).
        Then \(I N = N\).
        In particular, if \(I \subset \rad(A)\), then \(M\) is \(I\)-adically separated.
    \end{corollary}
    \begin{proof}
        We have that 
        \[ N = I^{n+r}M \cap N = I^n (I^r M \cap N) = I^n N \subset IN \subset N.\]
        The latter conclusion follows from the Nakayama's Lemma.
    \end{proof}

    \begin{definition}\label{def: I-adically complete}
        Let \(A\) be a ring, \(I\) an ideal of \(A\) and \(M\) an \(A\)-module.
        We say that \(M\) is \emph{complete (with respect to \(I\)-adic topology)} 
        if \(M\) is \(I\)-adically separated and complete as a metric space with respect to the metric induced by the \(I\)-adic topology.
    \end{definition}

    \begin{lemma}\label{lem: formal completion exists}
        Let \(A\) be a ring, \(I\) an ideal of \(A\) and \(M\) an \(A\)-module. 
        Then the inverse limit
        \[ \widehat{M} \coloneqq \varprojlim ( \cdots \to M/I^nM \to M/I^{n-1}M \to \cdots \to M/IM) \]
        exists in the category of \(A\)-modules.
        Moreover, \(\widehat{A}\) is an \(A\)-algebra and \(\widehat{M}\) is an \(\widehat{A}\)-module.
    \end{lemma}
    \begin{proof}
        Let 
        \[ \widehat{M}:= \left\{ (x_n) \in \prod_{n \geq 0} M/I^n M \Big| x_{n+1} \mapsto x_n \right\}. \]
        We claim that \(\widehat{M}\) is that we desired.
        \Yang{To be completed.}
    \end{proof}

    \begin{definition}[Formal Completion]\label{def: formal completion}
        Let \(A\) be a ring, \(I\) an ideal of \(A\) and \(M\) an \(A\)-module. 
        The \emph{formal completion} of \(M\) with respect to \(I\), denoted by \(\widehat{M}\), is defined as
        \[ \widehat{M} := \varprojlim ( \cdots \to M/I^n M \to M/I^{n-1} M \to \cdots \to M/I M), \]
        where the maps are the natural projections \(M/I^n M \to M/I^{n-1} M\).
    \end{definition}

    \begin{example}\label{eg: p-adic integer ring}
        Let \(A = \bbz\) be the ring of integers and \(I = p\bbz\).
        The formal completion of \(\bbz\) with respect to \(p\bbz\) is the ring of \(p\)-adic integers, denoted by \(\bbz_p\).
        The elements of \(\bbz_p\) can be represented as infinite series of the form
        \[ a_0 + a_1 p + a_2 p^2 + \cdots, \]
        where \(a_i \in \{0, 1, \ldots, p-1\}\).
    \end{example}

    \begin{example}\label{eg: ring of formal power series}
        Let \(R\) be a ring, \(A=R[X_1, \ldots, X_n]\) and \(I = (X_1, \ldots, X_n)\).
        The formal completion of \(A\) with respect to \(I\) is the ring of formal power series \(R[[X_1, \ldots, X_n]]\).
        The elements of \(R[[X_1, \ldots, X_n]]\) can be represented as infinite series of the form
        \[ \sum_{i_1,\ldots,i_n} a_{i_1, \ldots, i_n} X_1^{i_1} \cdots X_n^{i_n}, \]
        where \(a_{i_1, \ldots, i_n} \in R\) and the multi-index \((i_1, \ldots, i_n)\) runs over all non-negative integers.
    \end{example}

    \begin{proposition}\label{prop: completion is complete}
        The formal completion \(\widehat{M}\) of a \(A\)-module \(M\) is complete, and image of \(M\) is dense in \(\widehat{M}\).
        Moreover, \(\widehat{M}\) is uniquely characterized by above properties.
    \end{proposition}
    \begin{proof}
        \Yang{To be completed.}
    \end{proof}

    By the universal property of the inverse limit, 
    we get a covariant functor from the category of \(A\)-modules to the category of topological \(\widehat{A}\)-modules, 
    which sends an \(A\)-module \(M\) to \(\widehat{M}\) and a morphism \(f: M \to N\) to the induced morphism \(\widehat{f}: \widehat{M} \to \widehat{N}\).

    \begin{lemma}\label{lem: completion is exact}
        Let 
        \[ 0 \to M_1 \to M_2 \to M_3 \to 0 \]
        be an exact sequence of finite \(A\)-modules.
        Then the sequence of \(\widehat{A}\)-modules
        \[ 0 \to \widehat{M_1} \to \widehat{M_2} \to \widehat{M_3} \to 0 \]
        is still exact.
    \end{lemma}
    \begin{proof}
        \Yang{To be completed.}
    \end{proof}

    \begin{proposition}\label{prop: completion and tensor product}
        Let \(\widehat{A}\) be completion of a noetherian ring \(A\) with respect to an ideal \(I\) and \(M\) a finite \(A\)-module. 
        Then the natural map \(M \ten_A \widehat{A} \to \widehat{M}\) is an isomorphism.
    \end{proposition}
    \begin{proof}
        Since \(A\) is noetherian and \(M\) is finite, we have an exact sequence
        \[ A^m \to A^n \to M \to 0. \]
        By Lemma \ref{lem: completion is exact}, we have an exact sequence
        \[ \widehat{A^m} \to \widehat{A^n} \to \widehat{M} \to 0. \]
        On the other hand, we have
        \[ A^m \ten_A \widehat{A} \to A^n \ten_A \widehat{A} \to M \ten_A \widehat{A} \to 0 \]
        by right exactness of the tensor product.
        Since the inverse limit commutes with finite direct sums, we complete the proof by the Five Lemma.
        % \Yang{To be completed.}
    \end{proof}

    \begin{proposition}\label{prop: completion is flat}
        Let \(A\) be a noetherian ring and \(I\) an ideal of \(A\). 
        Then the formal completion \(\widehat{A}\) of \(A\) with respect to \(I\) is a flat \(A\)-module.
    \end{proposition}
    \begin{proof}
        This is a direct consequence of Lemma \ref{lem: completion is exact} and Proposition \ref{prop: completion and tensor product}.
        % \Yang{To be completed.}
    \end{proof}

    \begin{lemma}\label{lem: completion is isomorphic to image of power series}
        Let \(\widehat{A}\) be the formal completion of a noetherian ring \(A\) with respect to an ideal \(I\). 
        Suppose that \(I\) is generated by \(a_1,...,a_n\). 
        Then we have an isomorphism of topological rings
        \[ \widehat{A} \cong A[[X_1, \ldots, X_n]]/(X_1-a_1, \cdots, X_n-a_n). \]
    \end{lemma}
    \begin{proof}
        \Yang{To be completed.}
    \end{proof}

    \begin{proposition}\label{prop: completion is noetherian}
        Let \(A\) be a noetherian ring and \(I\) an ideal of \(A\). 
        Then the formal completion \(\widehat{A}\) of \(A\) with respect to \(I\) is a noetherian ring.
    \end{proposition}
    \begin{proof}
        Note that \(A[[X_1, \ldots, X_n]]\) is noetherian by Hilbert's Basis Theorem.
        Then the conclusion follows from Lemma \ref{lem: completion is isomorphic to image of power series}.
    \end{proof}

    \begin{proposition}\label{prop: completion with respect to maximal ideal is local}
        Let \(A\) be a noetherian ring and \(\frakm\) a maximal ideal of \(A\). 
        Then the formal completion \(\widehat{A}\) of \(A\) with respect to \(\frakm\) is a local ring with maximal ideal \(\frakm \widehat{A}\).
    \end{proposition}
    \begin{proof}
        \Yang{To be completed.}
    \end{proof}

\subsection{Complete local rings}

    Let \((A, \frakm, \kk)\) be a noetherian complete local ring with respect to the \(\frakm\)-adic 
    topology.
    We say that \(A\) is \emph{of equal characteristic} if \(\characteristic A = \characteristic \kk\), and \emph{of mixed characteristic} if \(\characteristic A \neq \characteristic \kk\).
    In latter case, \(\characteristic \kk = p\) and \(\characteristic A = 0\) or \(\characteristic A = p^k\).

    The goal of this subsection is the following structure theorem for noetherian complete local rings due to Cohen. 

    \begin{theorem}[Cohen Structure Theorem]\label{thm: Cohen Structure Theorem}
        Let \((A,\frakm,\kk)\) be a noetherian complete local ring of dimension \(d\).
        Then 
        \begin{enumerate}
            \item \(A\) is a quotient of a noetherian regular complete local ring;
            \item if \(A\) is regular and of equal characteristic, then \(A \cong \kk[[X_1, \ldots, X_d]]\);
            \item if \(A\) is regular, of mixed characteristic \((0,p)\) and \(p \not\in \frakm^2\), then \(A \cong D[[X_1, \ldots, X_{d-1}]]\), where \((D,p,\kk)\) is a complete DVR;
            \item if \(A\) is regular, of mixed characteristic \((0,p)\) and \(p \in \frakm^2\), then \(A \cong D[[X_1, \ldots, X_{d}]]/(f)\), where \((D,p,\kk)\) is a complete DVR and \(f\) a regular parameter.
        \end{enumerate}
    \end{theorem}

    % \subsubsection{Some facts about complete local rings}

        To prove the Cohen Structure Theorem, we first list some preliminary results on complete local rings.
        They are independently important and can be used in other contexts.

        \begin{theorem}[Hensel's Lemma]\label{thm: Hensel Lemma}
            Let \((A,\frakm, \kk)\) be a complete local ring, \(f \in A[X]\) a monic polynomial and \(\overline{f} \in \kk[X]\) its reduction modulo \(\frakm\).
            Suppose that \(\overline{f} = \overline{g} \cdot \overline{h}\) for some monic polynomials \(\overline{g}, \overline{h} \in \kk[X]\) such that \(\gcd (\overline{g},\overline{h}) = 1\).
            Then the factorization lifts to a unique factorization \(f = g \cdot h\) in \(A[X]\) such that \(g\) and \(h\) are monic polynomials.
        \end{theorem}
        \begin{proof}
            Lift \(\overline{g}\) and \(\overline{h}\) to monic polynomials \(g_1, h_1 \in A[X]\).
            We inductively construct a sequence of monic polynomials \(g_n, h_n \in A[X]\) such that \(\Delta_n=f - g_nh_n \in \frakm^n[X]\) and \(g_n-g_{n+1}, h_n-h_{n+1} \in \frakm^{n}[X]\) for all \(n\geq 1\).
            Suppose that \(g_n\) and \(h_n\) are constructed.
            Let \(g_{n+1} = g_n + \varepsilon_n\) and \(h_{n+1} = h_n + \eta_n\) for \(\varepsilon_n, \eta_n \in \frakm^{n}[X]\).
            Then we have
            \[ f-g_{n+1}h_{n+1} = \Delta_n - (\varepsilon_n h_n + \eta_n g_n) + \varepsilon_n \eta_n. \]
            Hence we just need to choose \(\varepsilon_n\) and \(\eta_n\) such that
            \[ \varepsilon_n h_n + \eta_n g_n \equiv \Delta_n \mod \frakm^{n+1}, \quad \deg \varepsilon_n < \deg g_n, \quad \deg \eta_n < \deg h_n. \]
            Since \(\gcd (\overline{g},\overline{h}) = 1\), there exist \(\overline{u},\overline{v} \in \kk[X]\) such that \(\overline{u} \overline{g} + \overline{v} \overline{h} = 1\) and \(\deg \overline{u} < \deg \overline{g}\), \(\deg \overline{v} < \deg \overline{h}\).
            Lift \(\overline{u}\) and \(\overline{v}\) to \(u, v \in A[X]\) preserving the degrees.
            Then we have \(ug_n + vh_n \equiv 1 \mod \frakm\).
            Let \(\varepsilon_n = u \Delta_n\) and \(\eta_n = v \Delta_n\).
            Then we get the desired equation.
        \end{proof}

        % \begin{remark}\label{rmk: Hensel's Lemma does not require A to be noetherian}
        %     Note that the Hensel's Lemma does not require \(A\) to be noetherian.
        % \end{remark}

        \begin{proposition}\label{prop: Nakayama's Lemma for I-adically separated modules}
            Let \((A,\frakm,\kk)\) be a noetherian complete local ring and \(M\) an \(A\)-module that is \(\frakm\)-adically separated.
            Suppose \(\dim_\kk M/\frakm M < \infty\).
            Then the basis of \(M \ten_A \kk\) as \(\kk\)-vector space can be lifted to a generating set of \(M\) as an \(A\)-module.
        \end{proposition}
        \begin{proof}
            Let \(t_1,\ldots,t_n \in M\) such that their images in \(M/\frakm M\) form a basis of \(M/\frakm M\) as a \(\kk\)-vector space.
            Then \( M = t_1 A + \cdots + t_n A + \frakm M\).
            For every \(x \in M\), we can write 
            \[ x = a_{0,1}t_1 + \cdots + a_{0,n}t_n + m_1\] 
            for some \(a_{0,i} \in A\) and \(m_1 \in \frakm M\).
            Inductively, we have \(\frakm^k M = t_1 \frakm^k + \cdots + t_n \frakm^k + \frakm^{k+1} M\).
            Suppose that we have constructed \(m_k \in \frakm^k M\).
            Then we can write
            \[ m_k = a_{k,1}t_1 + \cdots + a_{k,n}t_n + m_{k+1}. \]
            Note that \(\sum_{k \geq 0} a_{k,i}\) converges in \(A\), denote its limit by \(a_i\).
            Then we have 
            \[x - a_1 t_1 + \cdots + a_n t_n = \sum_{i=1}^n\sum_{r \geq k} a_{r,i} t_i + m_{k} \in \frakm^kM\]
            for all \(k\).
            Since \(M\) is \(\frakm\)-adically separated, \(x = a_1 t_1 + \cdots + a_n t_n\).
            It follows that \(M = \sum A t_i\).
        \end{proof}

        The key to prove the Cohen Structure Theorem is the existence of coefficient rings.

        \begin{definition}[Coefficient rings]\label{def: coefficient rings}
            Let \((A,\frakm,\kk)\) be a noetherian complete local ring.
            
            When \(A\) is equal-characteristic, the coefficient ring (or coefficient field) is a homomorphism of rings \(\kk \to A\) such that \(\kk \to A \to A/\frakm\) is an isomorphism.
            
            When \(A\) is mixed-characteristic, the coefficient ring is a complete local ring \((R,pR,\kk)\) with a local homomorphism of rings \(R \injmap A\) such that the induced homomorphism \(R/pR \to A/\frakm\) is an isomorphism.
        \end{definition}
        \begin{remark}
            Recall that a homomorphism of local rings \(f:(A,\frakm_A) \to (B,\frakm_B)\) is said to be local if \(f^{-1}(\frakm_B) = \frakm_A\).
        \end{remark}

        \begin{theorem}\label{thm: existence of coefficient rings}
            Every noetherian complete local ring \((A,\frakm,\kk)\) has a coefficient ring.
        \end{theorem}

        Assume the existence of coefficient rings, we can prove the Cohen Structure Theorem.

        \begin{proof}[Proof of Cohen Structure Theorem]
            Let \(R\) be a coefficient ring of \(A\) and \(\frakm = (f_1, \ldots, f_d)\) a minimal generating set of \(\frakm\).
            Then we have a homomorphism of complete local rings 
            \[ \Phi: R[[X_1, \ldots, X_d]] \to A, \quad X_i \mapsto f_i. \] 
            Let \(\frakn\) be the maximal ideal of \(R[[X_1, \ldots, X_d]]\).
            Then \(\frakn A = \frakm\).
            By Proposition \ref{prop: Nakayama's Lemma for I-adically separated modules}, \(A\) is generated by \(1\) as an \(R[[X_1, \ldots, X_d]]\)-module.
            This implies that \(\Phi\) is surjective and (a) follows.

            If \(A\) is regular of equal characteristic, then \(\frakm\) is generated by a regular sequence.
            By consider the dimension of \(R[[X_1, \ldots, X_d]]\) and \(A\), we have that \(\Phi\) is an isomorphism.
            This proves (b).
            
            Note that if \(A\) is regular of mixed characteristic \((0,p)\) and \(p \not\in \frakm^2\), then \(\frakm\) is generated by \(p, f_1, \ldots, f_{d-1}\).
            Then consider the homomorphism of complete local rings
            \[  R[[X_1, \ldots, X_{d-1}]] \to A, \quad X_i \mapsto f_i.\]
            By the same argument as above, we have that it is an isomorphism.
            This proves (c).

            For (d), we have that \(\ker \Phi\) is of height \(1\) by the dimension argument.
            Since regular local rings are UFDs, we can write \(\ker \Phi = (f)\) for some \(f \in R[[X_1, \ldots, X_d]]\).
            Then we finish.
        \end{proof}

    \subsubsection{Existence of coefficient rings}

        % \begin{definition}[Coefficient rings]\label{def: coefficient rings}
        %     Let \((A,\frakm,\kk)\) be a noetherian complete local ring.
            
        %     When \(A\) is equal-characteristic, the coefficient ring (or coefficient field) is a homomorphism of rings \(\kk \to A\) such that \(\kk \to A \to A/\frakm\) is an isomorphism.
            
        %     When \(A\) is mixed-characteristic, the coefficient ring is a complete local ring \((R,pR,\kk)\) with a local homomorphism of rings \(R \injmap A\) such that the induced homomorphism \(R/pR \to A/\frakm\) is an isomorphism.
        % \end{definition}
        % \begin{remark}
        %     Recall that a homomorphism of local rings \(f:(A,\frakm_A) \to (B,\frakm_B)\) is said to be local if \(f^{-1}(\frakm_B) = \frakm_A\).
        % \end{remark}

        % This subsubsection we show the following existence result of coefficient rings due to Cohen.

        % \begin{theorem}\label{thm: existence of coefficient rings}
        %     Every noetherian complete local ring \((A,\frakm,\kk)\) has a coefficient ring.
        % \end{theorem}

        \begin{proof}[Proof of Theorem \ref{thm: existence of coefficient rings} in characteristic \(0\)]
            Note that for any \(n \in \bbz\), \(n \not\in \frakm\).
            Hence \(\bbq \subset A\).
            Let \( \Sigma := \{ \text{subfield in } A \} \) and \(K\) a maximal element in \(\Sigma\) with respect to the inclusion.
            The set \(\Sigma\) is non-empty since \(\bbq \in \Sigma\). 
            By Zorn's Lemma, \(K\) exists.
            Then \(K\) is a subfield of \(\kk\) by \(K \injmap A \surjmap A/\frakm \cong \kk\).
            We claim that \(K\) is a coefficient field of \(A\).

            Suppose there is \(\overline{t} \in \kk \setminus K\). 
            If \(\overline{t}\) is transcendent over \(K\), lift \(\overline{t}\) to an element \(t \in A\).
            Then for any polynomial \(f \neq 0\in K[T]\), we have \(f(\overline{t}) \neq 0 \in \kk\).
            Hence \(f(t) \notin \frakm\).
            This implies that \(1/f(t) \in A\), whence \(K(t) \subset A\).
            This contradicts the maximality of \(K\).
            If \(\overline{t}\) is algebraic over \(K\), let \(f \in K[T]\) be the minimal polynomial of \(\overline{t}\).
            Then \(f\) is irreducible in \(K[T]\) and \(f(\overline{t}) = 0\).
            Regard \(f\) as a polynomial in \(A[T]\) by \(K \injmap A\).
            Note that \(\characteristic A = 0\) implies that \(f\) is separable.
            By Hensel's Lemma (Theorem \ref{thm: Hensel Lemma}), we can lift the root \(\overline{t}\) to an element \(t \in A\) such that \(f(t) = 0\).
            Then \(K(t)\) is a field extension of \(K\) and \(K(t) \subset A\).
            This contradicts the maximality of \(K\) again.
        \end{proof}

        The same strategy does not work when \(\characteristic \kk = p > 0\) since there might be inseparable extensions.
        To fix this, we need to introduce the notion of \(p\)-basis.

        \begin{definition}\label{def: p-basis for field of characteristic p}
            Let \(\kk\) be a field of characteristic \(p\).
            A finite set \(\{t_1,\ldots,t_n\} \subset \kk \setminus \kk^p\) is called \emph{\(p\)-independent} if \([\kk(t_1,\ldots,t_n):\kk]=p^n\).
            A set \(\Theta \subset \kk \setminus \kk^p\) is called a \emph{\(p\)-independent} if its any finite subset is \(p\)-independent.
            A \emph{\(p\)-basis} for \(\kk\) is a maximal \(p\)-independent set \(\Theta \subset \kk \setminus \kk^p\).
        \end{definition}

        By definition, we have that \(\kk = \kk^p[\Theta]\) for any \(p\)-basis \(\Theta\) of \(\kk\).
        For any \(a\in \kk\) and \(\theta \in \Theta\), we can write \(a\) as a polynomial in \(\Theta\) with coefficients in \(\kk^p\).
        The degree of \(\theta\) in such polynomial representation is at most \(p-1\).
        Such polynomial representation is unique by definition of \(p\)-independence.

        Applying the Frobenius map \(n\) times, we have that \(\kk^{p^n} = \kk^{p^{n+1}}[\Theta^{p^n}]\).
        This follows that \(\kk = \kk^{p^n}[\Theta]\) for all \(n\).
        Moreover, for any \(a \in \kk\) and \(\theta \in \Theta\), we can write \(a\) as a polynomial in \(\Theta\) with coefficients in \(\kk^{p^n}\) and the degree of \(\theta\) is at most \(p^n-1\).
        Such polynomial representation is unique.

        Let \(\kk\) be a perfect field of characteristic \(p\).
        If there is \(a \in \kk \setminus \kk^p\), then \(\kk(a^{1/p})/\kk\) is an inseparable extension.
        This contradicts the perfectness of \(\kk\).
        Hence \(\kk = \kk^p\) and \(\kk\) has no nonempty \(p\)-basis.
            
        \begin{example}\label{eg: p-basis for imperfect fields}
            Let \(\kk = \bbf_p(t_1,\ldots,t_n)\).
            Then \(\kk^p = \bbf_p(t_1^p,\ldots,t_n^p)\).
            The set \(\{t_1,\ldots,t_n\}\) is a \(p\)-basis for \(\kk\).
        \end{example}

        \begin{proof}[Proof of Theorem \ref{thm: existence of coefficient rings} in characteristic \(p\)]
            Choose \(\Theta \subset A\) such that its image in \(A/\frakm\) is a \(p\)-basis for \(\kk\).
            Let \(A_n := A^{p^n} = \{a^{p^n} \colon a \in A\}\) and \(K \coloneqq \bigcap_{n \geq 0} (A_n[\Theta])\).
            Then we claim that \(K\) is a coefficient field of \(A\).

            First we show that \(A_n[\Theta] \cap \frakm \subset \frakm^{p^n}\).
            For every \(a \in A_n[\Theta]\), 
            if the degree of \(\theta\) in the polynomial representation of \(a\) is more than \(p^n-1\), 
            we can write \(\theta^k = \theta^{a p^n} \cdot \theta^{b}\) for some \(b < p^n\).
            Regard \(\theta^{a p^n} \in A^{p^n}\) as coefficients.
            Now assume that \(a \in A_n[\Theta] \cap \frakm\).
            Then consider the image of \(a\) in \(A/\frakm\).
            The image of \(a\) equals \(0\) implies every coefficient of \(a\) is in \(\frakm\).
            Such coefficients are of form \(b^{p^n}\) for some \(b \in A\), whence \(b \in \frakm\).
            Hence \(a \in \frakm^{p^n}\).
            This implies that \(K \cap \frakm = \bigcap_{n \geq 0} (A_n[\Theta] \cap \frakm) \subset \bigcap_{n \geq 0} \frakm^{p^n} = \{0\}\).
            Then \(K\) is a field and hence a subfield of \(\kk\).

            For any \(\overline{a} \in \kk\), note that 
            \( \kk = \kk^p[\overline{\Theta}] = \kk^{p^2}[\overline{\Theta}] = \cdots = \kk^{p^n}[\overline{\Theta}] = \cdots\).
            For every \(n\), write 
            \[ \overline{a} = \sum_{\mu_n} \overline{c}_{\mu_n}^{p^n} \mu_n =: P_{\overline{a},n}(\overline{c}_{\mu_n}), \]
            where \(\mu_n\) runs over all monomials in \(\overline{\Theta}\) with degree at most \(p^n-1\) and \(\overline{c}_{\mu_n} \in \kk\).
            We call this representation the \(p^n\)-development of \(\overline{a}\) with respect to \(\overline{\Theta}\).
            Plug the \(p^m\)-development of \(c_{\mu_n}\) into \(P_{\overline{a},n}\), we get the \(p^{n+m}\)-development of \(\overline{a}\).
            In formula, that is, 
            \[ P_{\overline{a},n}(P_{\overline{a},m}(\overline{c}_{\mu_{n+m}})) = P_{\overline{a},n+m}(\overline{c}_{\mu_{n+m}}). \]

            Lift \(\overline{c}_{\mu_n}\) to \(c_{\mu_n} \in A\) for all \(\mu_n\).
            Let \(a_n := P_{\overline{a},n}(c_{\mu_n}) = \sum_{\mu_n} c_{\mu_n}^{p^n} \mu_n \in A_n[\Theta]\).
            For \(m\geq n\), we have \(a_n - a_m \in A_n[\Theta] \cap \frakm \subset \frakm^{p^n}\).
            Hence \(a_n\) converges to an element \(a \in A\).
            Now we show that \(a \in K\).
            For every \(\mu_k\), let \(b_{\mu_k,n} \in A\) be the element getting by plugging \(c_{\mu_{n+k}}\) into the \(P_{\overline{c}_{\mu_k},n}\).
            Then \(b_{\mu_k,n}\) converges to an element \(b_{\mu_k} \in A\).
            By construction, we have
            \[ a = \lim_{n \to \infty} P_{\overline{a},n+k}(c_{\mu_{n+k}}) = \lim_{n \to \infty} P_{\overline{a},k}(b_{\mu_k,n}) = P_{\overline{a}}(b_{\mu_k}) = \sum_{\mu_k} b_{\mu_k}^{p^k} \mu_k \in A_k[\Theta],\quad \forall k. \] 
            It follows that \(a \in K\).
        \end{proof}

        \begin{lemma}\label{lem: coefficient ring in artin ring}
            Let \((A,\frakm,\kk)\) be a noetherian complete local ring of mixed characteristic.
            Suppose that \(\frakm^n = 0\) for some \(n \geq 1\).
            Then there exists a complete local ring \((R,pR,\kk)\) with \(R \subset A\).
        \end{lemma}
        \begin{proof}
            Fix a \(p\)-basis of \(\kk\) and lift it to \(\Theta \subset R\).
            Let \(q = p^{n-1}\) and 
            \[ \calm:= \left\{ \theta_1^{k_1} \cdots \theta_d^{k_d} | \ \theta_i \in \Theta, k_i \leq q-1 \right\},\quad S := \left\{ \sum_{\mu \in \calm,\text{ finite }} a_\mu \mu \bigg| a_\mu \in R^q \right\}. \]
            For any \(a,b \in A\), we claim that \(a \equiv b \mod \frakm\) if and only if \(a^q \equiv b^q \mod \frakm^n\).
            If \(a \equiv b \mod \frakm\), write \(a = b + m\) for some \(m \in \frakm\).
            Then \(a^p = b^p + p b^{q-1} m + \cdots + m^q\).
            Hence \(a^p \equiv b^p \mod \frakm^2\).
            Inductively, we have \(a^q \equiv b^q \mod \frakm^n\).
            Conversely, if \(a^q \equiv b^q \mod \frakm^n\), then \(a^q - b^q \in \frakm^n \subset \frakm\).
            Note that the Frobenius map \(x \mapsto x^q\) is injective on \(A/\frakm\).
            It follows that \(a \equiv b \mod \frakm\).
            By the claim, \(S\) maps to \(\kk^q[\Theta] = \kk\) bijectively.
            
            Let 
            \[ R \coloneqq S + pS + p^2 S + \cdots + p^{n-1} S. \]
            We claim that \(R\) is a subring of \(A\).
            If so, \(R/pR \cong \kk\) and we get a complete local ring \((R,pR,\kk)\).

            Take \(a,b \in A\).
            We have 
            \[ a^q + b^q = (a+b)^q + p c \in A^q + p A. \]
            Inductively, we have 
            \[ a^q + b^q \in A^q + p A^q + \cdots + p^{n-1} A^q. \]
            This implies that \(R\) is closed under addition.
            Note that \(\theta^a = \theta^{aq} \cdot \theta^b\) with \(b < q\).
            Then for any \(\mu,\nu \in \calm\), we have \(\mu \nu \in S\).
            Hence \(R\) is closed under multiplication.
            % \Yang{To be completed.}
        \end{proof}

        \begin{lemma}\label{lem: DVR with certain residue field}
            Let \(\kk\) be a field of characteristic \(p\).
            Then there exists a DVR \((D,p,\kk)\) of mixed characteristic \((0,p)\).
        \end{lemma}
        \begin{proof}
            Fix a well order \(\leq\) on \(\kk\) and for any \(a \in \kk\), set \(\kk_a\) be the subfield of \(\kk\) generated by all elements \(b \in \kk\) such that \(b \leq a\).
            Then \(\kk = \bigcup_{a \in \kk} \kk_a\).
            We construct DVRs \(D_a\) with residue field \(\kk_a\) such that \(D_a \subset D_b\) for \(a \leq b\).
            Begin from \(\kk_0 = \bbf_p\) and let \(D_0 = \bbz_{(p)}\).
            Suppose that \(D_a\) is constructed for all \(a < b\).
            If \(\kk_b/\kk_a\) is transcendental, then let \(D_b\) be the localization of \(D_a[b]\) at the prime ideal generated by \(p\).
            
            If \(\kk_b/\kk_a\) is algebraic, then let \(\overline{f} \in \kk_a[T]\) be the monic minimal polynomial of \(b\).
            Let \(\KK_a = \Frac(D_a)\) and \(K_b = \KK_a[T]/(f)\), where \(f\) is a monic lift of \(\overline{f}\) to \(D_a[T]\).
            Note that \(f\) is irreducible since \(\overline{f}\) is irreducible.
            Let \(D_b\) be the integral closure of \(D_a\) in \(K_b\).
            In general, \(D_b\) is a Dedekind domain.
            Consider the prime factorization \(pD_b = \frakp_1^{e_1} \cdots \frakp_k^{e_k}\) in \(D_b\).
            For every \(i\), \(D_b/\frakp_i\) is a field extension of \(\kk_a\) and \(\overline{f}\) has a root in \(D_b/\frakp_i\).
            Suppose \(\deg \overline{f} = \deg f= d\).
            It follows that \([(D_b/\frakp_i):\kk_a] = d\).
            Note that we have \(\sum_{i=1}^k e_if_i = [\KK_b:\KK_a] = d\).
            Hence \(k = 1\) and \(e_1 = 1\).
            It follows that \(pD_b\) is prime and \(D_b\) is a DVR with residue field \(\kk_b\).
            
            Let \(D = \bigcup_{a \in \kk} D_a\).
            Then \((D,pD,\kk)\) is the desired DVR.
        \end{proof}

        \begin{example}\label{eg: DVR with certain residue fields}
            Let \(\kk = \bbf_p(t)\).
            Then \(D = \bbz[t]_{(p)}\) is a DVR satisfying the condition in Lemma \ref{lem: DVR with certain residue field}.

            Let \(\kk = \overline{\bbf_p}\).
            For any \(n \geq 1\), let \(K_n = K_{n-1}(\zeta_{p^n-1})\) and \(K_0 = \bbq\).
            Let \(D_n := \calo_{K_n,\frakp_n}\) be the localization of the ring of integers of \(K_n\) at the prime \(\frakp_n\) lying above \(\frakp_{n-1}\).
            Then \(D\coloneqq \bigcup_n D_n\) is a DVR with residue field \(\kk\).
            % \Yang{To be completed.}
        \end{example}

        \begin{lemma}\label{lem: unique complete local ring of mixed characteristic (p^k,p)}
            Given \(\kk\) a field of characteristic \(p\), 
            there exists a unique complete local ring \((R,pR,\kk)\) of mixed characteristic \((p^n,p)\).
        \end{lemma}
        \begin{proof}
            The existence follows from Lemma \ref{lem: DVR with certain residue field}.
            To show the uniqueness, suppose that \((R',pR',\kk)\) is another complete local ring of mixed characteristic \((p^n,p)\).
            Fix a \(p\)-basis of \(\kk\) and lift it to \(\Theta \subset R\) and \(\Theta' \subset R'\) relatively.
            Let \(q = p^{n-1}\) and 
            \[ \calm:= \left\{ \theta_1^{k_1} \cdots \theta_d^{k_d} | \ \theta_i \in \Theta, k_i \leq q-1 \right\},\quad S := \left\{ \sum_{\mu \in \calm,\text{ finite }} a_\mu \mu \bigg| a_\mu \in R^q \right\}. \]
            Define \(\calm',S'\) similarly with \(\Theta'\) and \(R'\).
            Since \(S \to R \to \kk\) and \(S' \to R' \to \kk\) are bijections, we can define a bijective map \(\Phi: S \to S'\).

            Note that any element in \(S\) can be written as \(s + pr\) with \(s \in S\) and \(r \in R\) uniquely since \(S \to \kk\) is bijective.
            Inductively, we can write any element in \(R\) as
            \[ r = s + ps_1 + p^2 s_2 + \cdots + p^{n-1} s_{n-1}, \]
            where \(s_i \in S\).
            The similarly for \(R'\).
            Extend \(\Phi\) to \(R\) and we get a bijection between \(R\) and \(R'\).
            Note that by construction, \(\Phi\) preserves addition and multiplication.
            Hence we get a ring isomorphism \(\Phi: R \to R'\).
            % \Yang{To be completed.}
        \end{proof}

        \begin{proof}[Proof of Theorem \ref{thm: existence of coefficient rings} in mixed characteristic]
            Since \(A\) is complete, we have \(A = \varprojlim_n A/\frakm^n\).
            By Lemma \ref{lem: coefficient ring in artin ring}, there is a complete local ring \((R_n,pR_n,\kk)\) with \(R_n \subset A/\frakm^n\).
            By Lemma \ref{lem: unique complete local ring of mixed characteristic (p^k,p)}, such \(R_n\) is unique up to isomorphism.
            It follows that \(R_n \cong R_m/p^{k_n}\) for \(m \geq n\).
            We get an inverse system 
            \[ \cdots \to R_n \to R_{n-1} \to \cdots \to R_1 \cong \kk. \]
            Let \(R := \varprojlim_n R_n\).
            Then \((R,pR,\kk)\) is a complete local ring.
            The homomorphisms \(R_n \injmap A/\frakm^n\) induce a homomorphism of complete local rings \(R \injmap A\).
            This concludes the proof.
        \end{proof}


    % \subsubsection{Proof of Cohen Structure Theorem}

        

    
        


% \subsection{Unique factorization of regular local rings}
    
%     \begin{theorem}[Weierstrass Preparation Theorem]\label{thm: Weierstrass Preparation Theorem}
%         Let \((A,\frakm)\) be a noetherian complete local ring, \(f = \sum_{n=0}^\infty a_n X^n \in A[[X]]\) a power series with \(f \not\equiv 0 \mod \frakm\).
%         Then there exists a unique factorization of the form \(f = u g\), where \(u\) is a unit in \(A[[X]]\) and \(g\) is a polynomial of the form
%         \[ g = X^d + b_{d-1} X^{d-1} + \cdots + b_0, \]
%         where \(b_i \in \frakm\) for all \(i\).
%     \end{theorem}
%     \begin{proof}
%         \Yang{To be completed.}
%     \end{proof}




