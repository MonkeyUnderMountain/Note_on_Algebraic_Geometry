\section{Basepoint Free Theorem on Positive Characteristic}

    This section refers to \cite{Kee99}, \cite{Art70} and \cite{FGA05}.
    Throughout this section, all schemes are of finite type over a base scheme \(S\) with \(S\) noetherian.
    we assume that the base field \(\kkk\) is algebraically closed and of positive characteristic $p$.

\subsection{Preliminaries}

    \begin{theorem}[{Serre vanishing in relative setting, ref. \cite[Theorem 1.7.6]{Laz04a}}]\label{thm:Serre_vanishing_relative_setting}
        Let \(f:X \to S\) be a proper morphism of schemes, \(\call\) a line bundle and \(\calf\) a coherent sheaf on \(X\).
        Suppose that \(\call\) is relatively ample.
        Then there exists \(n_0 \in \bbn\) such that for all \(n \geq n_0\), the higher direct image sheaves \(R^if_* \calf \ten \call^{\otimes n}\) are zero for all \(i > 0\).
    \end{theorem}

    \begin{definition}\label{def:exceptional_locus_of_nef_line_bundle}
        Let \(X\) be a proper variety and \(\call\) a nef line bundle on \(X\).
        A closed subvariety \(Z \subseteq X\) is called the \emph{exceptional} for \(\call\) if \(\call^{\dim Z} \cdot Z = 0\).
        The \emph{exceptional locus} of \(\call\), denoted by \(\Exc \call\), is defined as the closure of the union of all exceptional subvarieties of \(\call\).
    \end{definition}

    If \(\call\) is semiample, then \(\Exc \call = \Exc \varphi\) for the fibration \(\varphi: X \to Y\) induced by \(\call\).

    \begin{definition}\label{def:EWM_bpf_semiample}
        Let \(X\) be a proper scheme and \(\call\) a nef line bundle on \(X\).
        We say that \(\call\) is \emph{endowed with a map (EWM)} if there is a proper morphism \(\varphi: X \to Y\) to a proper algebraic space such that \(\dim Z > \dim f(Z)\) if and only if \(Z\) is an exceptional subvariety of \(\call\).
        If such a morphism is a fibration, then it is unique, called the \emph{fibration associated to \(\call\)}.
    \end{definition}

    \begin{proposition}\label{lem:fundamental_properties_of_EWM}
        Let \(X\) be a proper variety and \(\call\) a nef line bundle on \(X\) endowed with a map.
        Let \(\varphi: X \to Y\) be the associated fibration.
        Then the \(\call\) is semiample iff there is line bundle \(\call_Y\) and \(m \in \bbz_{>0}\) such that \(\call^{\otimes m} = \varphi^*\call_Y\).
    \end{proposition}
    \begin{proof}
        \Yang{To be completed.}
    \end{proof}

    \begin{definition}\label{def:universally_homeomorphism}
        A morphism \(f: X \to Y\) of schemes is called a \emph{universal homeomorphism} 
        if for every \(Y\)-scheme \(Y'\), the base change \(X \times_Y Y' \to Y'\) is a homeomorphism between the underlying topological spaces.
    \end{definition}

    \begin{example}\label{eg:universal_homeomorphism}
        Let \(X\) be a scheme of finite type over \(\kkk\).
        Then the natural morphism \(X_{\text{red}} \to X\) is a universal homeomorphism.

        Let \(X\) be a scheme over \(S\) of characteristic \(p\).
        Then the absolute and relative Frobenius morphisms are universal homeomorphisms.
        \Yang{To be completed.}

        The morphism \(\Spec \bbc \to \Spec \bbr\) is not a universal homeomorphism.
    \end{example}

    \begin{lemma}\label{lem:pic0_over_overline_bbf_p_is_torsion}
        Let \(X\) be a projective scheme over \(\kkk = \overline{\bbf_p}\).
        Then \(\Pic^0(X)\) is a torsion group.
    \end{lemma}
    \begin{proof}
        \Yang{To be completed.}
    \end{proof}

\subsection{Algebraic space}

    \begin{definition}\label{def:Grothendieck_topology_and_site}
        Let \(\bfC\) be a category.
        A \emph{Grothendieck topology} on \(\bfC\) is a collection of sets of arrows \(\{U_i \to U\}_{i \in I}\), called \emph{covering}, for each object \(U\) in \(\bfC\) such that:
        \begin{enumerate}
            \item if \(V \to U\) is an isomorphism, then \(\{V \to U\}\) is a covering;
            \item if \(\{U_i \to U\}_{i \in I}\) is a covering and \(V \to U\) is a arrow, then the fiber product \(U_i \times_U V \to V\) exists and \(\{U_i \times_U V \to V\}\) is a covering of \(V\);
            \item if \(\{U_i \to U\}_{i \in I}\) and \(\{U_{ij} \to U_{i}\}_{j \in J_i}\) are coverings, then the collection of composition \(\{U_{ij} \to U_i \to U\}_{i\in I, j\in J_i}\) is a covering.
        \end{enumerate}
        A \emph{site} is a pair \((\bfC, \calj)\) where \(\bfC\) is a category and \(\calj\) is a Grothendieck topology on \(\bfC\).
    \end{definition}

    Note that sheaf is indeed defined on a site.

    \begin{definition}\label{def:sheaves_on_site}
        Let \((\bfC, \calj)\) be a site.
        A \emph{sheaf} on \((\bfC, \calj)\) is a functor \(\calf: \bfC^{op} \to \Set\) satisfying the following condition:
        for every object \(U\) in \(\bfC\) and every covering \(\{U_i \to U\}_{i \in I}\) of \(U\), 
        if we have a collection of elements \(s_i \in \calf(U_i)\) such that for every \(i,j\), 
        the pullback \(s_i|_{U_i \times_U U_j}\) and \(s_j|_{U_i \times_U U_j}\) are equal, 
        then there exists a unique element \(s \in \calf(U)\) such that for every \(i\), the pullback \(s|_{U_i} = s_i\).
    \end{definition}

    \begin{definition}\label{def:big_etale_site}
        Let \(X\) be a scheme.
        The \emph{big \'etale site} of \(X\), denoted by \((\Sch/X)_{\text{\'et}}\), is the category of schemes over \(X\) with the Grothendieck topology generated by \'etale morphisms,
        that is, a collection of morphisms \(\{U_i \to U\}_{i \in I}\) is a covering if and only if each \(U_i\) is \'etale over \(U\) and the union of their images is the whole \(U\).
    \end{definition}

    Let \(X\) be a scheme over \(S\).
    By Yoneda's Lemma, it is equivalent to give a functor \(h_X: \Sch_S^{op} \to \Set\) such that for any \(S\)-scheme \(T\), \(h_X(T) = \Hom_{\Sch_S}(T, X)\).
    \Yang{Easy to check that \(h_X\) is a sheaf on the big \'etale site \((\Sch/S)_{\text{\'et}}\).}

    \begin{definition}\label{def:etale_equivalent_relation}
        Let \(U\) be a scheme over a base scheme \(S\).
        An \emph{\'etale equivalence relation} on \(U\) is a morphism \(R \to U \times_S U\) between schemes over \(S\) such that:
        \begin{enumerate}
            \item the projections in two factors \(R \to U\) are \'etale and surjective;
            \item for every \(S\)-scheme \(T\), \(h_R(T) \to h_U(T) \times h_U(T)\) gives an equivalence relation on \(h_U(T)\) set-theoretically.
        \end{enumerate}
    \end{definition}

    \begin{definition}\label{def:algebraic_space_as_topological_space}
        An \emph{algebraic space} \(X\) over a base scheme \(S\) is an \(S\)-scheme \(U\) together with an \'etale equivalence relation \(R \to U \times_S U\).
    \end{definition}

    Let \(X = (U,R)\) be an algebraic space over \(S\).
    We explain \(X\) as a sheaf on the big \'etale site \((\Sch/S)_{\text{\'et}}\).
    For any scheme \(T\) over \(S\), \(h_R(T)\) is an equivalence relation on \(h_U(T)\).
    The rule sending \(T\) to the set of equivalence classes of \(h_R(T)\) gives a presheaf on the site \((\Sch/S)_{\text{\'et}}\).
    The sheafification of this presheaf is the sheaf associated to the algebraic space \(X\).
    Explicitly, we have
    \[ X(T) := \left\{ f = (f_i) \Bigg| \begin{array}{c}
        \{T_i\to T\} \text{ a covering, } f_i \in h_U(T_i) \text{ such}  \\
        \text{that } (f_i|_{T_i\times_T T_j},f_j|_{T_i\times_T T_j}) \in h_R(T_i \times_T T_j)
    \end{array}     \right\}\Bigg/ \sim, \]
    where 
    \[ \alpha \sim \beta \quad \text{ if } \exists \{S_i \to T\} \text{ such that } (\alpha|_{S_i},\beta|_{S_i}) \in h_R(S_i). \]
    
    % \begin{example}\label{eg:algebraic_space_as_quiotient}
    %     Let \(U = \bba_\bbc^1\) and \(R \subset U \times U\) given by \(y=x+n, n \in \bbz\).
    %     Then \(R\) is a disjoint union of lines in \(U \times U\).
    %     Easily see that the projection \(\pi_i: R \to U\) is \'etale and surjective for \(i=1,2\).
    %     Then \(X := (U,R)\) is an algebraic space.
    %     % \Yang{To be completed.}
    % \end{example}

    \begin{definition}\label{def:algebraic_space_as_sheaf}
        An \emph{algebraic space} over a base scheme \(S\) is a sheaf \(F\) on the big \'etale site \((\Sch/S)_{\text{\'et}}\) such that 
        \begin{enumerate}
            \item the diagonal morphism \(F \to F \times_S F\) is representable;
            \item there exists a scheme \(U\) over \(S\) and a map \(h_U \to F\) which is surjective and \'etale.
        \end{enumerate}
        The \emph{morphism between algebraic spaces} \(F_1,F_2\) is defined as a natural transformation of functors \(F_1, F_2\).
        % \Yang{to be completed.}
    \end{definition}

    \begin{remark}
        By Yoneda's Lemma, given a morphism \(h_U \to F\) between sheaves is the same as giving an element of \(F(U)\).
        We may abuse the notation.
    \end{remark}

    \begin{definition}\label{def:properties_of_sheaves_by_scheme}
        Let \(\calp\) be a property of morphisms of schemes satisfying the following conditions:
        \begin{enumerate}
            \item is preserved under any base change;
            \item is \'etale local on the base.\Yang{In \cite{Stacks}, this requires that ``fppf local''.}
        \end{enumerate}
        Let \(\alpha: F \to G\) be a representable morphism of sheaves on the big \'etale site \((\Sch/S)_{\text{\'et}}\).
        We say that \(\alpha\) has property \(\calp\) if for every \(h_T \to G\), the base change \(h_T\times_{G} F \to F\) has property \(\calp\).
    \end{definition}

    \begin{remark}
        The fiber product \(F_1 \times_F F_2\) is just defined as \(F_1 \times_F F_2(T) := F_1(T) \times_{F(T)} F_2(T)\) for any object \(T \in \Obj(\Sch_S)\).
        We say that a morphism \(f: F_1 \to F_2\) of sheaves is \emph{representable} if for every \(T \in \Obj(\Sch/S)\) and every \(\xi \in F_2(T)\), the sheaf \(F_1 \times_{F_2} h_T\) is representable as a functor.
        Here \(h_T \to F_2\) is given by 
        \[ h_T(U) \to F_2(U), \quad f \in \Hom(U,T) \mapsto F_2(f)(\xi) \in F_2(U). \]

        In our case, given an arbitrary \(h_U \to F \times F\) is equivalent to giving morphisms \(h_{U_i} \to F\) for \(i=1,2\).
        And the fiber product \(F \times_{F \times F} (h_{U_1} \times h_{U_2})\) is just the fiber product \(h_{U_1} \times_{F} h_{U_2}\).
        Hence the first condition in \cref{def:algebraic_space_as_sheaf} is equivalent to that \(h_{U_1} \times_F h_{U_2}\) is representable for any \(U_1,U_2\) over \(F\).
        This implies that \(h_U \to F\) is representable, whence the second condition in \cref{def:algebraic_space_as_sheaf} makes sense.
    \end{remark}

    % \begin{example}\label{eg:algebraic_space_as_sheaf}
    %     We explain \cref{eg:algebraic_space_as_quiotient} as a sheaf.
    %     For any scheme \(T\) over \(\bbc\), let 
    %     \[ h_X(T):= \Eq(\pi_1,\pi_2) = \{f\in h_R(T): \pi_1\circ f = \pi_2 \circ g\}. \]
    %     Easy to check that \(h_X\) is a sheaf on the big \'etale site \((\Sch/\bbc)_{\text{\'et}}\).

    %     First we show that \(h_U \to X\) is representable.
    %     \Yang{To be completed.}

    %     Second we show that \(h_U \to X\) is surjective and \'etale.

    %     Finally we show that the diagonal \(X \to X \times X\) is representable.

    %     \Yang{To be completed.}
    % \end{example}

    \begin{definition}\label{def:underlying_set_of_algebraic_space}
        Let \(X\) be an algebraic space over a base scheme \(S\).
        Two two morphisms form field \(\Spec k_i \to X\) is called equivalent if there is a common extension \(K \supset k_1,k_2\) such that we have \(\Spec K \to \Spec k_i \to X\) are the same for \(i=1,2\).
        The \emph{underlying point set} of \(X\), denote by \(|X|\), is defined as the set of equivalence classes of morphisms \(\Spec k \to X\) for all field \(k\) over the base field \(\kkk\).
    \end{definition}

    This definition coincides with the underlying set of a scheme.
    Let \(\alpha: X \to Y\) be a morphism of algebraic spaces.
    It induces a map \(|\alpha|: |X| \to |Y|\) by \(x \mapsto \alpha \circ x\) (vertical composition).

    \begin{proposition}[{ref. \cite[Lemma 66.4.6]{Stacks}}]\label{prop:underlying_topological_space_of_algebraic_space}
        There is a unique topology on \(|X|\) such that
        \begin{enumerate}
            \item if \(X\) is a scheme, then the topology coincides with the usual topology.
            \item every morphism of algebraic spaces \(f: X \to Y\) induces a continuous map \(|f|: |X| \to |Y|\).
            \item if \(U\) is a scheme and \(U \to X\) is \'etale, then the induced map \(|U| \to |X|\) is open.
        \end{enumerate}
    \end{proposition}

    This topology is called the \emph{Zariski topology} on \(|X|\).

    \begin{definition}\label{def:structure_sheaf_of_algebraic_space}
        Let \(X\) be an algebraic space over a base scheme \(S\).
        All \'etale morphisms \(U \to X\) with \(U\) scheme form a small site \(X_{\text{\'et}}\).
        All \'etale morphisms \(U \to X\) with \(U\) algebraic space form a small site \(X_{\text{sp, \'et}}\).
        The \emph{structure sheaf} \(\calo_X\) of \(X\) is given by \(U \mapsto \Gamma(U,\calo_U)\) for every \'etale morphism \(U \to X\) from a scheme.
        It extends to a sheaf on the site \(X_{\text{sp, \'et}}\) uniquely.
    \end{definition}

    \begin{example}\label{eg:C/Z_as_algebraic_space}
        Let \(U = \bba_\bbc^1\) and \(R \subset U \times U\) given by \(y=x+n, n \in \bbz\).
        Then \(R\) is a disjoint union of lines in \(U \times U\).
        Write \(R = \coprod_{n\in \bbz} R_n\) with \(R_n = \{(x,x+n): x \in \bbc\}\).
        Then the projection is given by 
        \begin{align*}
            &\pi_1|_{R_n}:R_n \to U, \quad (x,x+n) \mapsto x, \\
            &\pi_2|_{R_n}:R_n \to U, \quad (x,x+n) \mapsto x+n.
        \end{align*} 
        Easily see that the projection \(\pi_i: R \to U\) is \'etale and surjective for \(i=1,2\).
        Let \(r_{ij}:R \times U \to U \times U \times U\) be the morphism which maps \(((x,y),u)\) to \((a_1,a_2,a_3)\) where \(a_i = x\), \(a_j = y\) and \(a_k = u\) for \(k \neq i,j\).
        Since \(\Delta_U \to U\times U\) factors through \(R\), \((\pi_1,\pi_2) = (\pi_2,\pi_1)\) and 
        \(r_{12} \times_{(U\times U\times U)} r_{23}\) factors through \(r_{13}\), 
        we have that \(h_R(T)\) is an equivalence relation on \(h_U(T)\) for all \(T\) over \(S\).
        Then \(X := (U,R)\) is an algebraic space.
        
        % Easy to check that \(h_X\) is a sheaf on the big \'etale site \((\Sch/\bbc)_{\text{\'et}}\).
        We do not check the representability here but give an example.
        Let \(U \to X\) be the natural morphism given by \(\id_U \in X(U)\).
        % \[ 
        %     \begin{tikzcd}
        %         U \arrow[r, "f"] & R \arrow[r, shift left=0.5ex, "\pi_1"] 
        %                         \arrow[r, shift right=0.5ex, swap, "\pi_2"] & U
        %     \end{tikzcd} , 
        %     \quad x \mapsto (x,x+n) \text{ for } n \in \bbz, 
        % \]
        For any scheme \(T\) over \(\bbc\), we have
        \[ (U\times_X U)(T) = \{(f,g) \in h_{U\times U}(T): \exists \{T_i \to T\} \text{ s.t. } (f_i,g_i) \in h_R(T_i) \} = h_R(T). \]
        Hence the fiber product \(h_U \times_X h_U\) is represented by \(R\).
        
        We show that \(X \not\cong \bbc^{\times}\) by computing the the global sections.
        Consider the covering \(U \to X\), a section \(s \in \calo_X(X)\) is given by a section \(s \in \Gamma(U,\calo_U) = \bbc[t]\) such that \(\pi_1^*s = \pi_2^*s\) in \(\Gamma(R,\calo_R)\).
        This means that \(s(x+n) = s(x)\) for all \(n \in \bbz\).
        Hence \(s\) is a constant function.
        In particular, \(\calo_X(X) = \bbc \neq \bbc[t,t^{-1}]\).

        The underlying set \(|X|\) is union of the quotient set \(\bbc/\bbz\) and a generic point.
        \Yang{The Zariski topology on \(|X|\) is the quiotient topology induced by \(|U| \to |X|\).}
    \end{example}

    \begin{definition}\label{def:coherent_sheaf_on_algebraic_space}
        Let \(X\) be an algebraic space over a base scheme \(S\).
        A \emph{coherent sheaf} on \(X\) is a sheaf \(\calf\) on \(X_{\text{\'et}}\) such that for every covering \(\{U_i \to X\}\) with \(U_i\) schemes, the sheaf \(\calf|_{U_i}\) is coherent for every \(i\).
        It extends to a sheaf on the site \(X_{\text{sp, \'et}}\) uniquely.

        An \emph{ideal sheaf} on \(X\) is a coherent sheaf \(\cali \subset \calo_X\).
        It defines a closed subspace \(V(\cali) \subset X\) by \Yang{to be completed.}
        And every closed subspace \(Y \subset X\) is defined by an ideal sheaf \(\cali_Y\) such that \(V(\cali_Y) = Y\).
    \end{definition}

    \begin{definition}\label{def:formal_completion_of_algebraic_space}
        Let \(X\) be an algebraic space over a base scheme \(S\) and \(Y\) a closed subset of \(|X|\).
        The \emph{formal completion} of \(X\) along \(Y\), denoted by \(\frakX\), is the functor defined as 
        \[ (\Sch/S)_{\text{\'et}} \to \Set, \quad U \mapsto \{f: U \to X: f(|U|) \subset |Y|\}. \]
        \Yang{to be completed.}
    \end{definition}

    \begin{definition}\label{def:modification}
        Let \(X\) be an algebraic space and \(Y\) a closed subset of \(X\).
        A \emph{modification} of \(X\) along \(Y\) is a proper morphism \(f: X' \to X\) and a closed subset \(Y' \subset X'\) such that \(X'\setminus Y' \to X \setminus Y\) is an isomorphism and \(f^{-1}(Y) = Y'\).
    \end{definition}

    \begin{theorem}[{ref. \cite[Theorem 3.1]{Art70}}]\label{thm:Artin_existence_of_modification}
        Let \(Y'\) be a closed subset of an algebraic space \(X'\) of finite type over \(\kkk\).
        Let \(\frakX'\) be the formal completion of \(X'\) along \(Y'\).
        Suppose that there is a formal modification \(\frakf: \frakX' \to \frakX\).
        Then there is a unique modification 
        \[ f: X' \to X, \quad Y \subset X \]
        such that the formal completion of \(X\) along \(Y\) is isomorphic to \(\frakX\) and the induced morphism \(\frakX' \to \frakX\) is isomorphic to \(\frakf\).
    \end{theorem}

    \begin{theorem}[{ref. \cite[Theorem 6.2]{Art70}}]\label{thm:Artin_higher_direct_image_and_modification}
        Let \(\frakX'\) be a formal algebraic space and \(Y' = V(\cali')\) with \(\cali'\) the defining ideal sheaf of \(\frakX'\).
        Let \(f:Y' \to Y\) be a proper morphism.
        Suppose that 
        \begin{enumerate}
            \item for every coherent sheaf \(\calf\) on \(\frakX'\), we have 
                \[ R^1f_* \cali'^n\calf/\cali'^{n+1}\calf = 0, \quad \forall n \gg 0; \]
            \item for every \(n\), the homomorphism 
                \[ f_*(\calo_{\frakX'} / \cali'^{n})\ten_{f_*\calo_{Y'}} \calo_Y \to \calo_Y \]
                is surjective.
        \end{enumerate}
        Then there exists a modification \(\frakf: \frakX' \to \frakX\) and a defining ideal sheaf \(\cali\) of \(\frakX\) such that \(V(\cali) = Y\) and \(\frakf\) induces \(f\) on \(Y\).
    \end{theorem}

    \begin{theorem}[{ref. \cite[Theorem 6.1]{Art70}}]\label{thm:finite_modification}
        Let \(Y'\) be a closed algebraic subspace of an algebraic space \(X'\) and \(f_0:Y' \to Y\) a finite morphism.
        Then there exists a modification \(f:X' \to X\) whose restriction to \(Y'\) is \(f_0\).
        It is the amalgamated sum \(X = X' \amalg_{Y'} Y\) in the category of algebraic spaces \(\AlgSp\).
    \end{theorem}

    \begin{example}\label{eg:finite_modification_of_line_in_plane}
        Let \(X = \bba^2 = \Spec \kkk[x,y]\) and \(Y = V(y)\) be the \(x\)-axis.
        Let \(f_0:Y'=\bba^1 \to Y, x \mapsto x^2\).
        Then there exists a modification \(f:X' \to X\) such that the restriction \(f|_{Y'}: Y' \to Y\) is \(f_0\).
        \Yang{To be completed.}
    \end{example}

    \begin{lemma}\label{lem:relative_frob_factors_through_universal_homeomorphism}
        Let \(f: X \to Y\) be a finite morphism of algebraic space and is a universal homeomorphism.
        Then there exists \(q= p^n\) such that the relative Frobinius morphism \(\Frob_{X/\kkk}^n\) factors as 
        \[ \Frob_{X/\kkk}^n: X \xrightarrow{f} Y \to X^{(q)}. \]
    \end{lemma}
    \begin{proof}
        \Yang{To be completed.}
    \end{proof}

    \begin{corollary}\label{lem:amalgamated_sum_exist_for_finite_universal_homeomorphism_in_char_p}
        Let \(Z \to X\) be a finite universal homeomorphism of algebraic spaces and \(Z \to Y\) any morphism of algebraic spaces.
        Suppose that \(X,Y,Z\) are all of finite type over \(\kkk\).
        Then the amalgamated sum \(X \amalg_Z Y\) exists in the category of algebraic spaces.
        Moreover, \(Y \to X \amalg_Z Y\) is a finite universal homeomorphism.
    \end{corollary}
    \begin{proof}
        \Yang{To be completed.}
    \end{proof}

\subsection{A sufficient and necessary condition for basepoint free}

    \begin{proposition}\label{prop:bpf_for_non-reduced}
        Let \(g: X' \to X\) be a proper, finite universal homeomorphism between algebraic spaces.
        Then a line bundle \(\call\) on \(X\) is endowed with a map if and only if \(g^*\call\) is endowed with a map.
    \end{proposition}
    \begin{proof}
        \Yang{To be completed.}
    \end{proof}

    \begin{proposition}\label{prop:bpf_for_reducible}
        Let \(X\) be a projective scheme and \(\call\) a nef line bundle on \(X\).
        Assume that \(X = X_1 \cup X_2\) for closed subsets \(X_1\) and \(X_2\).
        Suppose that \(\call|_{X_i}\) is endowed with a map \(g_i:X_i \to Z_i\) for \(i = 1,2\).
        Assume that for all but finitely many points \(x \in X\), the geometric fiber of \(g_1|_{X_1\cap X_2}\) are connected.
        Then \(\call\) is endowed with a map \(g: X \to Z\).
    \end{proposition}
    \begin{proof}
        \Yang{To be completed.}
    \end{proof}

    \begin{proposition}\label{prop:basepoint_free_nef_big_iff_effective_locus_char_p}
        Let \(X\) be a proper variety and \(D\) a nef and big divisor on \(X\).
        Then we can write \(D = A + E\) where \(A\) is an ample divisor and \(E\) is an effective divisor.
        Then \(D\) is endowed with a map iff \(D|_{E_{red}}\) is endowed with a map.
    \end{proposition}
    \begin{proof}
        By \cref{prop:bpf_for_non-reduced}, we may assume that \(D|_E\) is endowed with a map \(f: E \to Z\).
        Let \(\call = \calo_X(-E)\) be the ideal sheaf of \(E\).
        note that \(-E = D - A\) and \(D\) is \(f\)-numerically trivial.
        Hence \(\call|_E\) is \(f\)-ample.
        By Serre's vanishing, for every coherent sheaf \(\calf\) on \(X\), 
        there exists \(n_0 \in \bbn\) such that for all \(n \geq n_0\), we have 
        \[ R^if_* \calf|_E \ten \call^{\otimes n} = R^if_* (\call^n\calf/\call^{n+1}\calf) = 0 \]
        for all \(i > 0\).
        

        \Yang{To be completed.}
    \end{proof}

    \begin{theorem}\label{thm:bpf_iff_bpf_on_exceptional_locus}
        Let \(X\) be a proper variety and \(\call\) a nef line bundle on \(X\).
        Then \(\call\) is basepoint free if and only if \(\call|_{\Exc \call}\) is basepoint free.
    \end{theorem}
    \begin{proof}
        \Yang{To be completed.}
    \end{proof}


\subsection{Basepoint free theorem on positive characteristic}

    \begin{theorem}\label{thm:basepoint_free_nef_big_char_p}
        Let $X$ be a normal projective \(\bbq\)-factorial threefold and \(B \in (0,1)\) a \(\bbq\)-divisor.
        Let \(\call\) be a nef and big line bundle on \(X\) such that \(\call - K_{(X,B)}\) is nef and big.
        Then \(\call\) is endowed with a map.
        Moreover, if \(\kkk = \overline{\bbf_p}\), \(\call\) is basepoint free.
    \end{theorem}
    \begin{proof}
        \Yang{To be completed.}
    \end{proof}


    