\section{Cone Theorem}


\subsection{Preliminary}

    \begin{theorem}[{Iitaka fibration, semiample case, ref. \cite[Theorem 2.1.27]{Laz04a}}]\label{thm: Iitaka fibration in semiample case}
        Let \(X\) be a projective variety and \(\call\) an semiample line bundle on \(X\).
        Then there exists a fibration \(\varphi: X \to Y\) of projective varieties 
        such that for any \(m\gg 0\) with \(\call^m\) base point free, we have that the morphism \(\varphi_{\call^m}\) induced by \(\call^m\) is isomorphic to \(\varphi\).
        Such a fibration is called the \emph{Iitaka fibration} associated to \(\call\).
    \end{theorem}

    \begin{theorem}[{Rigidity Lemma, ref. \cite[Lemma 1.15]{Deb01}}]\label{thm: Rigidity Lemma}
        Let \(\pi_i: X \to Y_i\) be proper morphisms of varieties over a field \(\kk\) for \(i=1,2\).
        Suppose that \(\pi_1\) is a fibration and \(\pi_2\) contracts \(\pi_1^{-1}(y_0)\).
        Then there exists a rational map \(\varphi: Y_1 \ratmap Y_2\) such that \(\pi_2 \circ \varphi = \pi_1\) and \(\varphi\) is well-defined near \(Y_1 \setminus \{y_0\}\). 
    \end{theorem}

    \begin{theorem}\label{thm:convex_separation_theorem}
        Let \(A,B \subset \bbr^n\) be disjoint convex sets.
        Then there exists a linear functional \(f: \bbr^n \to \bbr\) such that \(f|_A \leq c\) and \(f|_B \geq c\) for some \(c \in \bbr\).
    \end{theorem}

    % \begin{theorem}[{Upper semicontinuity of cohomology, ref. \cite[Chapter III, Theorem 12.8]{Har77}}]\label{thm: upper semicontinuity of cohomology}
    %     Let \(f:X \to Y\) be a projective morphism of noetherian schemes and \(\calf\) a coherent sheaf on \(X\) flat over \(Y\).
    %     Then the function \(y \mapsto h^i(y,\calf)\) is upper semicontinuous on \(Y\) for every \(i\).   
    % \end{theorem}

    % \begin{theorem}[{Grauert's Theorem, ref. \cite[Chapter III, Corollary 12.9]{Har77}}]\label{thm: Grauert's Theorem}
    %     Let \(f:X \to Y\) be a projective morphism of noetherian schemes and \(\calf\) a coherent sheaf on \(X\) flat over \(Y\).
    %     Suppose that \(Y\) is integral and \(h^i(y,\calf)\) is constant on \(Y\) for some \(i\).
    %     Then \(R^if_*\calf\) is locally free sheaf on \(Y\) and the natural map \(R^if_*\calf \ten \rkk(y) \to H^i(X_y,\calf_y)\) is an isomorphism for every \(y \in Y\).
    % \end{theorem}

    \begin{proposition}\label{prop:general_rational_curve_on_klt_variety_with_negative_intersection}
        Let \(X\) be a normal projective variety of dimension \(n\) and \(H\) an ample divisor on \(X\).
        Suppose that \(K_X\cdot H^{n-1} < 0\).
        Then for a general point \(x\in X\), there exists a rational curve \(\Gamma\) passing through \(x\) such that
        \[ 0 < H \cdot \Gamma \leq - 2n \cdot \frac{H^n}{K_X \cdot H^{n-1}}. \]
    \end{proposition}
    \begin{proof}[Schetch of proof]
        % Choose a curve \(C = H^{n-1}\) such that \(X\) is smooth near \(C\).
        Take a resolution \(f:Y \to X\), then \(f^*H\) is nef on \(Y\) and \(K_Y\cdot f^*H^{n-1} < 0\) since \(E \cdot f^*H^{n-1} = 0\).
        Choose an ample divisor \(H_Y\) on \(Y\) closed enough to \(f^*H\) such that \(K_Y \cdot H_Y^{n-1} < 0\).
        By \cite[Theorem 5]{MM86} and take limit for \(H_Y\).
    \end{proof}

    \begin{lemma}[{ref. \cite[Lemma]{Kaw91}}]\label{lem:the_pair_controls_the_exceptional_locus}
        Let \((X,B)\) be a projective klt pair and \(f:X \to Y\) a birational projective morphism.
        Let \(E\) be an irreducible component of dimension \(d\) of the exceptional locus of \(f\) and \(\nu:E^\nu \to X\) the normalization of \(E\).
        Suppose that \(f(E)\) is a point.
        Then for any ample divisor \(H\) on \(X\), we have
        \[ K_{E^\nu}\cdot \nu^*H^{d-1} \leq K_{(X,B)}|_{E^\nu}\cdot \nu^*H^{d-1}. \]
    \end{lemma}

    % \begin{lemma}\label{lem:higher_direct_image_of_exceptional_divisor}
    %     Let \(f:Y \to X\) be a birational morphism of projective varieties with \(Y\) smooth and \(X\) has only rational singularities.
    %     Let \(E\) be an effective exceptional divisor on \(Y\) and \(D\) a divisor on \(X\).
    %     Then we have
    %     \[ f_*(\calo_Y(f^*D + E)) \cong \calo_X(D), \quad R^if_*(\calo_Y(f^*D + E)) = 0,\quad \forall i > 0. \]
    % \end{lemma}
    % \begin{proof}
    %     \Yang{I am unable to proof this lemma.}
    % \end{proof}

\subsection{Non-vanishing Theorem}

    \begin{theorem}[Non-vanishing Theorem]\label{thm: non-vanishing theorem}
        Let \((X,B)\) be a projective klt pair and \(D\) a Cartier divisor on \(X\).
        Suppose that \(D\) is nef and \(aD-K_{(X,B)}\) is nef and big for some \(a > 0\).
        Then for \(m \gg 0\), we have 
        \[ H^0(X,mD) \neq 0. \]
    \end{theorem}

\subsection{Base Point Free Theorem}

    \begin{theorem}[Base Point Free Theorem]\label{thm: base point free theorem}
        Let \((X,B)\) be a projective klt pair and \(D\) a Cartier divisor on \(X\).
        Suppose that \(D\) is nef and \(aD-K_{(X,B)}\) is nef and big for some \(a > 0\).
        Then for \(m \gg 0\), \(mD\) is base point free.
    \end{theorem}

    \begin{remark}\label{rmk:statement_in_BPF_theorem_stronger_than_semiample}
        In general, we say that a Cartier divisor \(D\) is \emph{semiample} if there exists a positive integer \(m\) such that \(mD\) is base point free.
        The statement in Base Point Free Theorem (\cref{thm: base point free theorem}) is strictly stronger than the semiample condition.
        For example, let \(\call\) be a torsion line bundle, then \(\call\) is semiample but there exists no positive integer \(M\) such that \(m\call\) is base point free for all \(m>M\).
    \end{remark}

\subsection{Rationality Theorem}

    

    % \begin{lemma}\label{lem:polynomial_vanishes_along_strips}
    %     Let \(P(x,y) \neq 0\in \bbq[x,y]\) with \(\deg P \leq n\).
    %     Then for every \(r,\varepsilon > 0\) with \(n+1 \leq v(r)\varepsilon\) and for all \(M>0\), \(P\) is not identically zero on the set 
    %     \[ \Lambda_{r,\varepsilon} \coloneqq \{(p,q) \in \bbz^2 : p,q>M, 0\leq pr-q \leq \varepsilon\}.\]
    %     \Yang{Need to modify}
    % \end{lemma}
    % \begin{proof}
    %     If \(v(r)=v < \infty\), then for all \(m > 0\), there is a line \(L \)

    %     \Yang{To be completed.}
    % \end{proof}

    \begin{lemma}[{ref. \cite[Theorem 1.36]{KM98}}]\label{lem:Hilbert_polynomial}
        Let \(X\) be a proper variety of dimension \(n\) and \(D_1,\ldots,D_m\) Cartier divisors on \(X\).
        Then the Euler characteristic \(\chi(n_1D_1,\ldots,n_mD_m)\) is a polynomial in \((n_1,\cdots,n_m)\) of degree at most \(n\).
    \end{lemma}

    \begin{theorem}[Rationality Theorem]\label{thm: rationality theorem}
        Let \((X,B)\) be a projective klt pair, \(a = a(X) \in \bbz\) with \(aK_{(X,B)}\) Cartier and \(H\) an ample divisor on \(X\).
        Let 
        \[ t \coloneqq \inf \{s \geq 0: K_{(X,B)} + sH \text{ is nef}\} \]
        be the nef threshold of \((X,B)\) with respect to \(H\).
        Then \(t = v/u \in \bbq\) and 
        \[ 0 \leq v \leq a(X)\cdot (\dim X + 1). \]
    \end{theorem}
    \begin{proof}
        For every \(r \in \bbr_{> 0}\), let 
        \[ v(r) \coloneqq \begin{cases}
            v, & \text{if } r = \frac{v}{u} \in \bbq \text{ in lowest term; } \\
            \infty, & \text{if } r \in \bbr\setminus \bbq.
        \end{cases} \]
        We need to show that \(v(t) \leq a(\dim X + 1)\).
        For every \((p,q) \in \bbz_{>0}^2\), set \(D(p,q) \coloneqq paK_{(X,B)} + qH\).
        If \((p,q)\in \bbz^2_{>0}\) with \(0< atp-q < t\), then we have \(D(p,q)\) is not nef and \(D(p,q) - K_{(X,B)}\) is ample.

        \begin{step}\label{step_in_thm:rationality_theorem:polynomial_non-vanishing_on_strips}
            We show that a polynomial \(P(x,y) \neq 0 \in \bbq[x,y]\) of degree at most \(n\) is not identically zero on the set
            \[  \{(p,q) \in \bbz^2 : p,q>M, 0< atp-q < t\varepsilon\}, \quad \forall M > 0, \] 
            if \(v(t)\varepsilon > a(n+1)\). 
        \end{step}
        If \(v(t) = \infty\), for any \(n\), we show that we can find infinitely many lines \(L\) such that \(\#L\cap \Lambda \geq n+1\).
        If so, \(\Lambda\) is Zariski dense in \(\bbq^2\).
        Since \(1/at\in \bbr\setminus \bbq\), there exist \(p_0,q_0>M\) such that 
        \[ 0 < \frac{p_0}{q_0} - \frac{1}{at} < \frac{\varepsilon}{(n+1)a} \cdot \frac{1}{q_0}, \text{ i.e. } 0<atp_0 - q_0 < \frac{\varepsilon t}{n+1}. \]
        Then \((ip_0,iq_0) \in \Lambda \cap \{p_0y=q_0x\}\) for \(i = 1,\cdots,n+1\).
        Since \(M\) is arbitrary, there are infinitely many such lines \(L\).
        
        Suppose \(v(t) = v < \infty\) and \(t = v/u\).
        Then the inequality is equivalent to \(0 < aup - vq < \varepsilon v\).
        Note that \(\gcd(au,v) | a\), then \(aup - vq = a i\) has integer solutions for \(i = 1,\cdots,n+1\).
        Since \(v(t)\varepsilon > a(n+1)\), there are at least \(n+1\) lines which intersect \(\Lambda\) in infinitely many points.
        This enforce any polynomial which vanishes on \(\Lambda\) has degree at least \(n+1\).
        % \Yang{To be completed.}

        \begin{step}
            There exists an index set \(\Lambda \subset \bbz^2\) such that \(\Lambda\) contains all sufficiently large \((p,q)\) with \(0 \leq at p - q \leq t\) and 
            \[ Z \coloneqq \Bs |D(p,q)| = \Bs |D(p',q')| \neq \emptyset, \quad \forall (p,q), (p',q') \in \Lambda. \]
        \end{step}
        For every \((p,q) \in \bbz^2_{>0}\) with \(0 < atp-q < t\), choose \(k \in \bbz_{>0}\) such that \(k(atp - q) > t\).
        Then for all \(p', q' > kp\) with \(0 < atp'-q' < t\), we have 
        \[ p' - kp \geq 0, \quad q'-kp > t(p'-kp). \]
        It follows that 
        
        \Yang{To be completed.}

        \begin{step}\label{step_in_thm:rationality_theorem:non-vanishing_on_strips}
            Suppose the contradiction that \(v(t) > a(\dim X + 1)\).
            Then we show that \(H^0(X,D(p,q)) \neq 0\) for all \((p,q) \in \Lambda\).
            This is an analogue of Non-vanishing Theorem in the proof of Base Point Free Theorem (\cref{thm: base point free theorem}).
        \end{step}
        Let \(P(x,y)\coloneqq \chi(D(x,y))\) be the Hilbert polynomial of \(D(x,y)\).
        Note that \(P(0,n) = \chi(nH) \neq 0\) since \(H\) is ample.
        Then \(P(x,y) \neq 0\) and \(\deg P \leq \dim X\).
        By \cref{step_in_thm:rationality_theorem:polynomial_non-vanishing_on_strips}, \(P\) is not identically zero on \(\Lambda\).
        Note that \(D(p,q) - K_{(X,B)}\) is ample for all \((p,q) \in \Lambda\), then \(h^i(X,D(p,q)) = 0\) for all \(i > 0\) by Kawamata-Viehweg vanishing theorem (\cref{thm: Kawamata-Viehweg Vanishing Theorem for klt pair}).
        Then 
        \[ P(p,q) = \chi(D(p,q)) = h^0(X,D(p,q)) \neq 0 \]
        for some \((p,q) \in \Lambda\).
        This is equivalent to that \(Z \neq X\) and hence \(H^0(X,D(p,q)) \neq 0\) for all \((p,q) \in \Lambda\).

        % \Yang{To be completed.}

        \begin{step}
            We follow the same line of the proof of Base Point Free Theorem (\cref{thm: base point free theorem}) to show that there is a section which does not vanish on \(Z\).
        \end{step}
        Fix \((p,q) \in \Lambda\).
        If \(v(t) < \infty\), we assume that \(t=v/u\) and \(atp-q = a(n+1)/u\).
        Let \(f:Y \to X\) be a resolution such that 
        \begin{enumerate}
            \item \(K_{Y,B_Y} = f^*K_{(X,B)} + E_Y\) for some effective exceptional divisor \(E_Y\), and \(Y,B_Y\) is a klt pair;
            \item \(f^*|D(p,q)| = |L| + F\) for some effective divisor \(F\) and a base point free divisor \(L\), and \(f(\Supp F) = Z\);
            \item \(f^*D(p,q) - f^*K_{(X,B)} - E_0\) is ample for some effective \(\bbq\)-divisor \(E_0 \in (0,1)\), and coefficients of \(E_0\) are sufficiently small;
            \item \(B_Y + E_Y + F + E_0\) has snc support.
        \end{enumerate}
        \Yang{Such resolution exists by \cite{KM98}.}

        Let \(c := \inf\{ \lfloor B_Y + E_0 + t F \rfloor \neq 0\}\).
        Adjust the coefficients of \(E_0\) slightly such that \(\lfloor B_Y + E_0 + c F \rfloor = F_0\) for unique prime divisor \(F_0\) with \(F_0 \subset \Supp F\).
        Set \(\Delta_Y \coloneqq B_Y + c F + E_0 - F_0\).
        Then \((Y,\Delta_Y)\) is a klt pair.

        Let 
        \begin{align*}
            N(p',q') &\coloneqq f^*D(p',q') + E_Y - F_0 - K_{(Y,\Delta_Y)}  \\
                & = \Big(f^*D(p',q') - (1+c)f^*D(p,q)\Big) + \Big(f^*D(p,q) - f^*K_{(X,B)} - E_0\Big) + c \Big(f^*D(p,q) - F\Big).
        \end{align*} 
        Note that on 
        \[ \Lambda_0 := \{(p',q') \in \Lambda: 0< atp' - q' < atp - q,\  p',q'>(1+c)\max\{p,q\}\}, \]
        the divisor \(f^*D(p',q') - (1+c)f^*D(p,q) = f^*D(p'-(1+c)p,q'-(1+c)q)\) is ample, and hence \(N(p',q')\) is ample.

        By the exact sequence 
        \[ 0 \to \calo_Y(f^*D(p',q')+E_Y-F_0) \to \calo_Y(f^*D(p',q')+E_Y) \to \calo_{F_0}((f^*D(p',q')+E_Y)|_{F_0}) \to 0 \]
        and Kawamata-Viehweg Vanishing Theorem (\cref{thm: Kawamata-Viehweg Vanishing Theorem for klt pair}), we get a surjective map
        \[ H^0(Y,f^*D(p',q')+E_Y) \surjmap H^0(F_0,(f^*D(p',q')+E_Y)|_{F_0}). \]
        On \(F_0\), consider the polynomial \(\chi((f^*D(p',q')+E_Y)|_{F_0})\).
        Note that \(\dim F_0 = n-1\) and by the construction of \((p,q),\Lambda_0\), 
        similar to \cref{step_in_thm:rationality_theorem:non-vanishing_on_strips}, 
        we can show that \(\chi((f^*D(p',q')+E_Y)|_{F_0})\) is not identically zero on \(\Lambda_0\).
        By adjunction, we have \((f^*D(p',q')+E_Y)|_{F_0} = N(p',q')|_{F_0} + K_{(F_0,\Delta_Y|_{F_0})}\) with \(N(p',q')|_{F_0}\) ample and \((F_0,\Delta_Y|_{F_0})\) klt.
        Hence we can apply Kawamata-Viehweg Vanishing Theorem (\cref{thm: Kawamata-Viehweg Vanishing Theorem for klt pair}) to get
        \[ h^0(F_0,(f^*D(p',q')+E_Y)|_{F_0}) = \chi(F_0,(D(p',q')+E_Y)|_{F_0}) \neq 0.\]
        This combining with the surjective map contradict to the assumption that \(f(F_0) \subset Z = \Bs |D(p',q')|\).
    \end{proof}


\subsection{Cone Theorem and Contraction Theorem}

    \begin{theorem}[Cone Theorem]\label{thm: cone theorem}
        Let \((X,B)\) be a projective klt pair.
        Then there exist countably many rational curves \(C_i \subset X\) with 
        \[ 0 < -K_{(X,B)} \cdot C_i \leq 2 \dim X \]
        such that 
        \begin{enumerate}
            \item we have a decomposition of cones
            \[ \Psef_1(X) = \Psef_1(X)_{K_{(X,B)} \geq 0} + \sum \bbr_{\geq 0}[C_i]; \]
            \item and for any \(\varepsilon > 0\) and an ample divisor \(H\) on \(X\), we have 
            \[ \Psef_1(X) = \Psef_1(X)_{K_{(X,B)}+\varepsilon H \geq 0} + \sum_{\text{finite}} \bbr_{\geq 0}[C_i]. \]
        \end{enumerate}
    \end{theorem}
    \begin{proof}
        Let \(F_D \coloneqq \Psef_1(X) \cap D^\perp\) for a nef divisor \(D\) on \(X\).
        If \(\dim F_D = 1\), we also write \(R_D \coloneqq F_D\).
        Let \(H_1,\cdots,H_{\rho-1}\) be ample divisors on \(X\) such that they together with \(K_{(X,B)}\) form a basis of \(N^1(X)_\bbq\).
        Fix a norm \(\|\cdot\|\) on \(N_1(X)_\bbr\) and let \(S^{\rho-1} \coloneqq S(N_1(X)_\bbr)\) be the unit sphere in \(N_1(X)_\bbr\).
        
        \begin{step}\label{step_in_thm:cone_theorem:negative_extremal_faces_will_be_stable}
            There exists an integer \(N\) such that for every \(K_{(X,B)}\)-negative extremal face \(F_D\) and for every ample divisor \(H\), 
            there exists \(n_0, r \in \bbz_{>0}\) such that for all \(n>n_0\), \(\{0\} \neq F_{nD+rK_{(X,B)}+N H} \subset F_D\). 
        \end{step}
        Let \(N \coloneqq (a(X)(\dim X + 1))!\), where \(a(X)\) is the number in \cref{thm: rationality theorem}.
        For every \(n\), \(nD+H\) is an ample divisor and by \cref{thm: rationality theorem}, the nef threshold of \(K_{(X,B)}\) with respect to \(nD+H\) is of form
        \[ \inf \{s \geq 0: K_{(X,B)} + s(nD+H) \text{ is nef}\} = \frac{N}{r_n}, \quad r_n \in \bbz_{\geq 0}. \]
        Since \(K_{(X,B)} + (N/r_n)((n+1)D+H)\) is nef, we have \(r_{n} \leq r_{n+1}\).
        On the other hand, let \(\xi \in F_{D}\setminus \{0\}\). 
        Then \(\xi \cdot (K_{(X,B)} + (N/r_n)(nD+H)) \geq 0\) implies that
        \[ r_n \leq - N \cdot \frac{K_{(X,B)}\cdot \xi}{H \cdot \xi}. \]
        Hence \(r_n \to r \in \bbz_{\geq 0}\).
        It follows that \(rK_{(X,B)}+nND+NH\) is a nef but not ample divisor for all \(n \gg 0\).
        Note that for every nef divisors \(N_1,N_2\), we have \(F_{N_1+N_2} = F_{N_1} \cap F_{N_2}\).
        Then for all \(n \gg 0\), there exists \(m\) large enough such that
        \[ \{0\} \neq F_{rK_{(X,B)}+mND+N H} \subset F_{rK_{(X,B)}+nD+NH} \subset F_D. \]

        \begin{step}\label{step_in_thm:cone_theorem:nagetive_extremal_rays_form_a_lattice}
            Let \(\Phi: N_1(X)_{K_{(X,B)}<0} \to \bbr^{\rho-1}\) be the map defined by 
            \[ \alpha \mapsto \left( \frac{H_1 \cdot \alpha}{K_{(X,B)}\cdot \alpha},\ldots, \frac{H_{\rho-1} \cdot \alpha}{K_{(X,B)}\cdot \alpha}\right). \]
            We show that the image of \(R_D\) under \(\Phi\) lies in a \(\bbz\)-lattice in \(\bbr^{\rho-1}\).
        \end{step}
        Suppose \(R = \bbr_{\geq 0}\xi\) for a class \(\xi\).
        By \cref{step_in_thm:cone_theorem:negative_extremal_faces_will_be_stable}, we have \(R_{nD+rK_{(X,B)}+N H_i} = R_{D}\) for some integers \(n,r\).
        Then \( \xi \cdot (nD+rK_{(X,B)}+N H_i) = 0 \) implies that
        \[ \frac{H_i \cdot \xi}{K_{(X,B)}\cdot \xi} = \frac{-r}{N} \in \frac{1}{N}\bbz. \]
        It follows that the image of \(R_D\) under \(\Phi\) lies in \(\frac{1}{N} \bbz^{\rho-1}\).

        \begin{step}\label{step_in_thm:cone_theorem:negative_extremal_rays_are_rational}
            We show that every \(K_{(X,B)}\)-negative extremal ray of \(\Psef_1(X)\) is of the form \(R_D\) for some nef divisor \(D\) on \(X\).
        \end{step}
        Let \(R = \bbr_{\geq 0}\xi\) be a \(K_{(X,B)}\)-negative extremal ray.
        \Yang{Then \(R\) is of form \(D^\perp \cap \Psef_1(X)\) for some nef \(\bbr\)-divisor \(D\) on \(X\) by \cref{thm:convex_separation_theorem}.}
        We need to show that \(D\) can be choose as a nef \(\bbq\)-divisor.
        There is a sequence of nef but not ample \(\bbq\)-divisors \(D_m\) such that \(D_m \to D\) as \(m \to \infty\).
        We adjust \(D_m\) such that \(\dim F_{D_m} = 1\) for all \(n\).

        By re-choosing \(H_i\), we can assume that \(D = a_1H_1 + \cdots + a_{\rho-1}H_{\rho-1} + a_\rho K_{(X,B)}\) for \(a_i > 0\) since \(aD-K\) is ample for \(a \gg 0\).
        After truncation, we can assume that so is \(D_m\).
        Then \(F_{D_m}\) is \(K_{(X,B)}\)-negative.
        Note that \(F_{nD_m+r_iK_{(X,B)} + N H_i} \subset F_{D_m}\) for some \(r_i>0\) and \(n\gg 0\) by \cref{step_in_thm:cone_theorem:negative_extremal_faces_will_be_stable}.
        If \(\dim F_{D_m} > 1\), then not all \(H_i|_{F_{D_m}}\) are proportional to \(K_{(X,B)}|_{D_m}\).
        We can assume that\(r_1K_{(X,B)}+N H_1\) is not identically zero on \(F_{D_m}\).
        Then we can choose \(n\) large enough such that \(\|r_1K_{(X,B)}+N H_1\|/n < 1/m\).
        Replace \(D_m\) by \(D_m + (r_1K_{(X,B)}+N H_1)/n\).
        Inductively we construct \(D_m\) nef \(\bbq\)-divisor with \(D_m \to D\) and \(\dim F_{D_m} = 1\).
        
        Let \(R_{D_m} = \bbr_{\geq 0} \xi_m\).
        Suppose that \(\|\xi_m\|=\|\xi\| = 1\).
        By passing to a subsequence, we can assume that \(\xi_m\) converges.
        Then \(\xi_m \to \xi\) since \(\lim D_m \cdot \xi_m = D \cdot \lim \xi_m = 0\).
        However, \(\Phi\) is well-defined at \(\xi\) and the image of \(\xi_m\) under \(\Phi\) is discrete.
        Hence \(\xi=\xi_m\) for all \(m\) large enough.
        It follows that \(R = R_{D_m}\) for a nef \(\bbq\)-divisor \(D_m\).

        \begin{step}\label{step_in_thm:cone_theorem:nagetive_extremal_rays_have_class_of_rational_curves}
            We show that any \(K_{(X,B)}\)-negative extremal ray \(R_D\) contains the class of a rational curve \(C\) with \(0 < -K_{(X,B)} \cdot C \leq 2 \dim X\).
        \end{step}
        By \cref{thm: contraction theorem}, let \(\varphi_D: X \to Y\) be the contraction associated to \(R_D\) (note that we do not need the step to proof \cref{thm: contraction theorem}).
        If \(\dim Y < \dim X\), let \(F\) be a general fiber of \(\varphi_D\).
        \Yang{By adjunction, \((F,B|_F)\) is a klt pair and \(K_{(F,B|_F)} = K_{(X,B)}|_F\).
        Take \(H=aD-K_{(X,B)}\) for some \(a > 0\) such that \(H\) is ample on \(F\).
        By \cref{prop:general_rational_curve_on_klt_variety_with_negative_intersection}.}
        \Yang{In birational case, by adjunction, suppose \(\varphi_D(E)\) is a point. By \cref{lem:the_pair_controls_the_exceptional_locus}, 
        we can use \cref{prop:general_rational_curve_on_klt_variety_with_negative_intersection} to get the result.}

        \Yang{To be completed.}

        \begin{step}\label{step_in_thm:cone_theorem:finish_the_proof}
            Proof of the theorem.
        \end{step}
        Given an ample divisor \(H\) on \(X\), note that \(\varepsilon H\) has positive minimum \(\delta\) on \(\Psef_1(X) \cap S^{\rho-1}\).
        Note that the set 
        \[ \{\alpha \in \Psef_1(X)\cap S^{\rho-1} : K_{(X,B)}\cdot \alpha \leq -\varepsilon H\cdot \alpha\} \subset \{\alpha: K_{(X,B)}\cdot \alpha \leq -\delta\} \] 
        is compact, and \(\Phi\) is well-defined on it.
        By \cref{step_in_thm:cone_theorem:negative_extremal_rays_are_rational,step_in_thm:cone_theorem:nagetive_extremal_rays_form_a_lattice}, 
        there are only finitely many extremal rays on \(\Psef_1(X)_{K_{(X,B)}+\varepsilon H \leq 0}\).
        By \cref{step_in_thm:cone_theorem:nagetive_extremal_rays_have_class_of_rational_curves}, we get (b).

        For (a), note that any closed cone is equal to the closure of the cone generated by its extremal ray.
        We only need to show that the cone
        \[ \calc\coloneqq \Psef_1(X)_{K_{(X,B)} \geq 0} + \sum \bbr_{\geq 0}[C_i] \]
        is closed.
        Choose a Cauchy sequence \(\{\alpha_n\} \subset \calc\) such that \(\alpha_n \to \alpha \in N_1(X)_\bbr\).
        Note that \(\Psef_1(X)\) is closed, hence \(\alpha \in \Psef_1(X)\).
        We only need to consider the case \(\alpha \cdot K_{(X,B)} < 0\).
        We can choose an ample divisor and \(\varepsilon > 0\) such that \(\alpha \cdot (K_{(X,B)}+\varepsilon H) < 0\).
        Then \(\alpha_n \cdot (K_{(X,B)}+\varepsilon H) < 0\) for all \(n\) large enough.
        Note that \(\calc \cap \{K_{(X,B)}+\varepsilon H \leq 0\}\) is a polyhedral cone by \cref{step_in_thm:cone_theorem:nagetive_extremal_rays_form_a_lattice} and hence is closed.
        Then \(\alpha \in \calc\) and the conclusion follows.
    \end{proof}
    \begin{remark}\label{rmk:extremal_ray_may_not_be_exposed}
        \Yang{Thanks for my friend Qin for pointing out that the extremal ray in \cref{thm: cone theorem} may not be exposed.}
    \end{remark}


    \begin{theorem}[Contraction Theorem]\label{thm: contraction theorem}
        Let \((X,B)\) be a projective klt pair and \(F \subset \Psef_1(X)\) a \(K_{(X,B)}\)-negative extremal face of \(\Psef_1(X)\).
        Then there exists a fibration \(\varphi_F: X \to Y\) of projective varieties such that
        \begin{enumerate}
            \item an irreducible curve \(C \subset X\) is contracted by \(\varphi_F\) if and only if \([C] \in F\);
            \item up to linearly equivalence, any Cartier divisor \(G\) with \(F \subset G^{\perp} = \{\alpha \in N_1(X) : \alpha \cdot G= 0\}\) comes from a Cartier divisor on \(Y\), 
                i.e., there exists a Cartier divisor \(G_Y\) on \(Y\) such that \(G \sim \varphi_F^* G_Y\).
        \end{enumerate}
    \end{theorem}
    \begin{proof}
        We follow the following steps to prove the theorem.
        \begin{step}\label{step:K_negative_face_is_rational_in_thm:contraction_theorem}
            We show that there exists a nef divisor \(D\) on \(X\) such that \(F = D^\perp \cap \Psef_1(X)\).
            In other words, \(F\) is defined on \(N_1(X)_\bbq\).
        \end{step}
        We can choose an ample divisor \(H\) and \(n > 0\) such that \(K_{(X,B)}+(1/n)H\) is negative on \(F\) since \(F \cap S^{\rho-1}\) is compact and \(K_{(X,B)}\) is strictly negative on it,
        where \(S^{\rho-1}\) is the unit sphere in \(N_1(X)_\bbr\).
        Then by Cone Theorem (\cref{thm: cone theorem}), \(F\) is an extremal face of a rational polyhedral cone, namely \(\Psef_1(X)_{K_{(X,B)}+(1/n) H \leq 0}\).
        It follows that \(F^\perp \subset N^1(X)_\bbr\) is defined on \(\bbq\).
        Since \(F\) is extremal and \(K_{(X,B)}+(1/n)H\)-negative, the set \(\{L \in F^\perp: L|_{\Psef_1(X)\setminus F}>0\}\) has non-empty interior in \(F^\perp\) by \cref{thm: cone theorem,thm:convex_separation_theorem}.
        Then there exists a Cartier divisor \(D\) such that \(D \in F^\perp\) and \(D|_{\Psef_1(X)\setminus F} > 0\).
        It follows that \(D\) is nef and \(F = D^\perp \cap \Psef_1(X)\).
        
        \begin{step}
            Let \(\varphi: X \to Y\) be the Iitaka fibration associated to \(D\) by \cref{thm: Iitaka fibration in semiample case}.
            We show that \(\varphi\) is the desired fibration.
        \end{step}
        Note that \(\Psef_1(X)_{K_{(X,B)} \geq 0} \cap S^{\rho-1}\) is compact and \(D\) is strictly positive on it.
        Then there exist \(a \geq 0\) such that \(aD - K_{(X,B)}\) is strictly positive on \(\Psef_1(X)_{K_{(X,B)} \geq 0} \cap S^{\rho-1}\).
        And \(K_{(X,B)}\) is strictly negative on \(F\setminus \{0\}\) since \(F\) is \(K_{(X,B)}\)-negative.
        Then by Base Point Free Theorem (\cref{thm: base point free theorem}), we know that \(mD\) is base point free for all \(m \gg 0\).
        Hence we can apply \cref{thm: Iitaka fibration in semiample case} to get a fibration \(\varphi_D: X \to Y\).

        First we show that \(D\) comes from \(Y\).
        Note that \(mD\) and \((m+1)D\) induces the same fibration \(\varphi_D\) for \(m \gg 0\).
        Then there exists \(D_{Y,m}\) and \(D_{Y,m+1}\) such that \(\varphi_D^* D_{Y,m} \sim mD\) and \(\varphi_D^* D_{Y,m+1} \sim (m+1)D\).
        Then set \(D_Y = D_{Y,m+1} - D_{Y,m}\), we have \(\varphi_D^* D_Y \sim D\).

        Note that \(D_Y \equiv (1/m) D_{Y,m}\) and \(D_{Y,m}\) is ample.
        Hence \(D_Y\) is ample.
        Then for any curve \(C \subset X\), we have
        \[ D \cdot C = \varphi^* D_Y \cdot C = D_Y \cdot (\varphi_D)_* C. \]
        It follows that \(C\) is contracted by \(\varphi_D\) if and only if \(D \cdot C = 0\), which is equivalent to \([C] \in F\).
        
        Let \(G\) be arbitrary Cartier divisor on \(X\) such that \(F \subset G^\perp\).
        Since \(D\) is strictly positive on \(\Psef_1(X) \setminus F\), for \(m \gg 0\), let \(D'\coloneqq mD+G\), we have \(D'^\perp \cap \Psef_1(X) = F\).
        Then by the same argument as above, we get an other fibration \(\varphi_{D'}: X \to Y'\) such that a curve \(C\) is contracted by \(\varphi_{D'}\) if and only if \([C] \in F\).
        Then by Rigidity Lemma (\cref{thm: Rigidity Lemma}), we see that \(\varphi_D = \varphi_{D'}\) up to an isomorphism on \(Y\).
        In particular, \(D' \sim \varphi_D^* D'_Y\) for some Cartier divisor \(D'_Y\) on \(Y\).
        Then \(G = D' - mD\) also comes from \(Y\).
    \end{proof}
    \begin{remark}\label{rmk_K_negative_face_is_rational}
        The \cref{step:K_negative_face_is_rational_in_thm:contraction_theorem} is amazing.
        If \(F\) is not \(K_{(X,B)}\)-negative, then it may not be rational.
        For example, let \(X = E \times E\) for a general elliptic curve \(E\).
        By \cite[Lemma 1.5.4]{Laz04a}, we know that \(\Psef_1(X)\) is a circular cone.
        The we see there indeed exist some irrational extremal faces of \(\Psef_1(X)\).
    \end{remark}

    \begin{definition}\label{def:types_of_contractions_in_MMP}
        Let \((X,B)\) be a projective klt pair and \(R\) a \(K_{(X,B)}\)-negative extremal ray of \(\Psef_1(X)\) with contraction \(\varphi_R: X \to Y\).
        There are three types of contractions:
        \begin{enumerate}
            \item \emph{Divisorial contraction}: if \(\dim X = \dim Y\) and the exceptional locus of \(\varphi_R\) is of codimension one;
            \item \emph{Small contraction}: if \(\dim X = \dim Y\) and the exceptional locus of \(\varphi_R\) is of codimension at least two;
            \item \emph{Mori fiber space}: if \(\dim X > \dim Y\).
        \end{enumerate}
    \end{definition}

    \begin{proposition}\label{prop:divisorial_or_fibered_contraction_preverses_Q_factorial}
        Let \((X,B)\) be a \(\bbq\)-factorial projective klt pair and \(R\) a \(K_{(X,B)}\)-negative extremal ray of \(\Psef_1(X)\).
        Suppose that the contraction \(\varphi:X\to Y\) associated to \(R\) is either divisorial or a Mori fiber space. 
        Then \(Y\) is \(\bbq\)-factorial.
    \end{proposition}
    \begin{proof}
        Let \(D\) be a prime Weil divisor on \(Y\) and \(U \subset Y\) a big open smooth subset.
        Let \(R = \bbr_{\geq 0}[C]\) for an irreducible curve \(C\) contracted by \(\varphi\).
        Set \(D_X \coloneqq \overline{\varphi|_{\varphi^{-1}(U)}^{-1} D}\).
        Then \(D_X\) is a prime Weil divisor on \(X\) and hence is \(\bbq\)-Cartier.

        If \(\varphi\) is a Mori fiber space, then \(D_X|_F \equiv 0\) for general fiber \(F\) of \(\varphi\).
        Then by Contraction Theorem (\cref{thm: contraction theorem}), we see that \(mD_X \sim \varphi^* D'\) for some Cartier divisor \(D'\) on \(Y\).
        We have \(mD|_{U} \sim D'|_{U}\) since \(\varphi|_{\varphi^{-1}(U)}\) is a fibration.
        Then \(mD \sim D'\) and hence \(D\) is \(\bbq\)-Cartier.

        If \(\varphi\) is a divisorial contraction, let \(E\) be the exceptional divisor of \(\varphi\) and assume that \(\varphi^{-1}|_U\) is an isomorphism.
        Then \(E \cdot C \neq 0\) (otherwise \(E \sim_{\bbq} f^*E_Y\) for some Cartier \(\bbq\)-divisor \(E_Y\) on \(Y\)).
        Then we can choose \(a\in\bbq\) such that \((D_X + aE)\cdot C = 0\).
        By Contraction Theorem (\cref{thm: contraction theorem}), we have \(mD_X + maE \sim \varphi^* D'\) for some Cartier divisor \(D'\) on \(Y\).
        Then we also have \(D|_{U} \sim mD'|_{U}\) since \(\varphi|_{\varphi^{-1}(U)}\) is an isomorphism.
        Hence \(D\) is \(\bbq\)-Cartier.
    \end{proof}
    \begin{remark}\label{rmk:small_contraction_is_never_Q_factorial}
        If \(\varphi\) is a small contraction, then \(Y\) is never \(\bbq\)-factorial.
        Otherwise, let \(B_Y\) be the strict transform of \(B\) on \(Y\).
        Note that \(K_{(Y,B_Y)}|_U \sim K_{(X,B)}|_U\) on a big open subset \(U\).
        Suppose \(K_{(Y,B_Y)}\) is \(\bbq\)-Cartier.
        Then \(\varphi^*K_{(Y,B_Y)} \sim_\bbq K_{(X,B)}\).
        Then we have
        \[ \varphi^*K_{(Y,B_Y)}\cdot C = 0 = K_{(X,B)}\cdot C < 0. \]
        This is a contradiction.
    \end{remark}

    \begin{example}
        Let \(X = E\times E \times \bbP^1\).
    \end{example}

    