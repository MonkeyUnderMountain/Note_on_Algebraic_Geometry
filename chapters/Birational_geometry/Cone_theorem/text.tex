\section{Cone Theorem}


\subsection{Preliminary}

    \begin{theorem}[{Iitaka fibration, semiample case, ref. \cite[Theorem 2.1.27]{Laz04a}}]\label{thm: Iitaka fibration in semiample case}
        Let \(X\) be a projective variety and \(\call\) an semiample line bundle on \(X\).
        Then there exists a fibration \(\varphi: X \to Y\) of projective varieties 
        such that for any \(m\gg 0\) with \(\call^m\) base point free, we have that the morphism \(\varphi_{\call^m}\) induced by \(\call^m\) is isomorphic to \(\varphi\).
        Such a fibration is called the \emph{Iitaka fibration} associated to \(\call\).
    \end{theorem}

\subsection{Non-vanishing Theorem}

    \begin{theorem}[Non-vanishing Theorem]\label{thm: non-vanishing theorem}
        Let \((X,B)\) be a projective klt pair and \(D\) a Cartier divisor on \(X\).
        Suppose that \(D\) is nef and \(aD-K_{(X,B)}\) is nef and big for some \(a > 0\).
        Then for \(m \gg 0\), we have 
        \[ H^0(X,mD) \neq 0. \]
    \end{theorem}

\subsection{Base Point Free Theorem}

    \begin{theorem}[Base Point Free Theorem]\label{thm: base point free theorem}
        Let \((X,B)\) be a projective klt pair and \(D\) a Cartier divisor on \(X\).
        Suppose that \(D\) is nef and \(aD-K_{(X,B)}\) is nef and big for some \(a > 0\).
        Then \(D\) is semiample.
    \end{theorem}

\subsection{Rationality Theorem}

    \begin{theorem}[Rationality Theorem]\label{thm: rationality theorem}
        Let \((X,B)\) be a projective klt pair, \(a = a(X) \in \bbz\) with \(aK_{(X,B)}\) Cartier and \(H\) an ample divisor on \(X\).
        Let 
        \[ t \coloneqq \inf \{s \geq 0: K_{(X,B)} + sH \text{ is nef}\} \]
        be the nef threshold of \((X,B)\) with respect to \(H\).
        Then \(t = u/v \in \bbq\) and 
        \[ 0 \leq u \leq a(X)\cdot (\dim X + 1). \]
    \end{theorem}


\subsection{Cone Theorem and Contraction Theorem}

    \begin{theorem}[Cone Theorem]\label{thm: cone theorem}
        Let \((X,B)\) be a projective klt pair.
        Then there exist countably many rational curves \(C_i \subset X\) with 
        \[ 0 < -K_{(X,B)} \cdot C_i \leq 2 \dim X \]
        such that 
        \begin{enumerate}
            \item we have a decomposition of cones
            \[ \Psef_1(X) = \Psef_1(X)_{K_{(X,B)} \geq 0} + \sum \bbr_{\geq 0}[C_i]; \]
            \item and for any \(\varepsilon > 0\) and an ample divisor \(H\) on \(X\), we have 
            \[ \Psef_1(X) = \Psef_1(X)_{K_{(X,B)}+\varepsilon H \geq 0} + \sum_{\text{finite}} \bbr_{\geq 0}[C_i]. \]
        \end{enumerate}
    \end{theorem}
    \begin{proof}
        We follow the following steps to prove the theorem.
        \begin{step}\label{step1:thm:cone_theorem}
            Let \(F_D \coloneqq \Psef_1(X) \cap D^\perp\) for a nef divisor \(D\) on \(X\).
            We show that if \(\dim F_D > 1\) and \(F_D \not\subset \Psef_1(X)_{K_{(X,B)} \geq 0}\), then we can choose \(D'\) nef such that \(F_{D'} \subset F_{D}\) and \(\dim F_{D'} < \dim F_D\).
        \end{step}
        \Yang{To be completed.}

        \begin{step}
            If \(\dim F_D = 1\), we also write \(R_D \coloneqq F_D\).
            We show that 
            \[ \Psef_1(X) = \overline{ \Psef_1(X)_{K_{(X,B)} \geq 0} + \sum R_D }. \]
        \end{step}
        \Yang{To be completed.}

        \begin{step}
            For any \(\varepsilon > 0\) and an ample divisor \(H\) on \(X\), we show that
            \[ \Psef_1(X) = \Psef_1(X)_{K_{(X,B)}+\varepsilon H \geq 0} + \sum_{\text{finite}} R_D . \]
        \end{step}
        \Yang{To be completed.}

        \begin{step}
            We show that any \(K_{(X,B)}\)-negative extremal ray \(R_D\) contains the class of a rational curve \(C\) with \(0 < -K_{(X,B)} \cdot C \leq 2 \dim X\).
        \end{step}

        \Yang{To be completed.}
    \end{proof}


    \begin{theorem}[Contraction Theorem]\label{thm: contraction theorem}
        Let \((X,B)\) be a projective klt pair and \(F \subset \Psef_1(X)\) a \(K_{(X,B)}\)-negative extremal face of \(\Psef_1(X)\).
        Then there exists a fibration \(\varphi_F: X \to Y\) of projective varieties such that
        \begin{enumerate}
            \item an irreducible curve \(C \subset X\) is contracted by \(\varphi_F\) if and only if \([C] \in F\);
            \item any line bundle \(\call\) with \(F \subset \call^{\perp} = \{\alpha \in N_1(X) : \alpha \cdot \call = 0\}\) comes from a line bundle on \(Y\), 
                i.e., there exists a line bundle \(\call_Y\) on \(Y\) such that \(\call \cong \varphi_F^* \call_Y\).
        \end{enumerate}
    \end{theorem}
    \begin{proof}
        We follow the following steps to prove the theorem.
        \begin{step}\label{step:K_negative_face_is_rational_in_thm:contraction_theorem}
            We show that there exists a nef divisor \(D\) on \(X\) such that \(F = D^\perp \cap \Psef_1(X)\).
            In other words, \(F\) is defined on \(N_1(X)_\bbq\).
        \end{step}
        \Yang{To be completed.}

        \begin{step}
            We show that \(D\) is semiample.
        \end{step}
        \Yang{To be completed.}

        \begin{step}
            Let \(\varphi: X \to Y\) be the Iitaka fibration associated to \(D\) by \cref{thm: Iitaka fibration in semiample case}.
            We show that \(\varphi\) is the desired fibration.
        \end{step}
        \Yang{To be completed.}
    \end{proof}
    \begin{remark}\label{rmk_K_negative_face_is_rational}
        The \cref{step:K_negative_face_is_rational_in_thm:contraction_theorem} is amazing.
        If \(F\) is not \(K_{(X,B)}\)-negative, then it may not be rational.
        For example, let \(X = E \times E\) for a general elliptic curve \(E\).
        By \cite[Lemma 1.5.4]{Laz04a}, we know that \(\Psef_1(X)\) is a circular cone.
        The we see there indeed exist some irrational extremal faces of \(\Psef_1(X)\).
    \end{remark}

    