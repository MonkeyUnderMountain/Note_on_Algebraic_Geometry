\section{Cone Theorem}


\subsection{Preliminary}

    \begin{theorem}[Iitaka fibration]\label{thm: Iitaka fibration}
        Let \(X\) be a projective variety and \(\call\) a line bundle on \(X\).
        Let \(\varphi_n: X \ratmap Y_n\) be the dominant rational map associated to \(\call^n\).
        Then for \(n \gg 0\), the rational maps \(\varphi_n\) stable to a fibration \(\varphi_\infty: X \ratmap Y_\infty\) up to birational equivalence.
    \end{theorem}

    \begin{proof}
        Here we test cref for the step environment.
        Test \cref{step: test label} for a step label.
    \end{proof}


\subsection{Non-vanishing Theorem}

    \begin{theorem}[Non-vanishing Theorem]\label{thm: non-vanishing theorem}
        Let \((X,B)\) be a projective klt pair and \(D\) a Cartier divisor on \(X\).
        Suppose that \(D\) is nef and \(aD-K_{(X,B)}\) is nef and big for some \(a > 0\).
        Then for \(m \gg 0\), we have 
        \[ H^0(X,mD) \neq 0. \]
    \end{theorem}


\subsection{Base Point Free Theorem}

    \begin{theorem}[Base Point Free Theorem]\label{thm: base point free theorem}
        Let \((X,B)\) be a projective klt pair and \(D\) a Cartier divisor on \(X\).
        Suppose that \(D\) is nef and \(aD-K_{(X,B)}\) is nef and big for some \(a > 0\).
        Then \(D\) is semiample.
    \end{theorem}


\subsection{Rationality Theorem}

    \begin{theorem}[Rationality Theorem]\label{thm: rationality theorem}
        Let \((X,B)\) be a projective klt pair, \(a = a(X) \in \bbz\) with \(aK_{(X,B)}\) Cartier and \(H\) an ample divisor on \(X\).
        Let 
        \[ t \coloneqq \inf \{s \geq 0: K_{(X,B)} + sH \text{ is nef}\} \]
        be the nef threshold of \((X,B)\) with respect to \(H\).
        Then \(t = u/v \in \bbq\) and 
        \[ 0 \leq u \leq a(X)\cdot (\dim X + 1). \]
    \end{theorem}


\subsection{Cone Theorem and Contraction Theorem}

    \begin{theorem}[Cone Theorem]\label{thm: cone theorem}
        Let \((X,B)\) be a projective klt pair.
        Then there exist countably many rational curves \(C_i \subset X\) with 
        \[ 0 < -K_{(X,B)} \cdot C_i \leq 2 \dim X \]
        such that 
        \begin{enumerate}
            \item we have a decomposition of cones
            \[ \Psef_1(X) = \Psef_1(X)_{K_{(X,B)} \geq 0} + \sum \bbr_{\geq 0}[C_i]; \]
            \item and for any \(\varepsilon > 0\) and an ample divisor \(H\) on \(X\), we have 
            \[ \Psef_1(X) = \Psef_1(X)_{K_{(X,B)}+\varepsilon H \geq 0} + \sum_{\text{finite}} \bbr_{\geq 0}[C_i]. \]
        \end{enumerate}
    \end{theorem}
    \begin{proof}
        We only need to prove (b) and (a) follows from (b) by taking \(\varepsilon = 1/n\).

        \begin{step}
            We show that 
            \[ \Psef_1(X) = \Psef_1(X)_{K_{(X,B)} \geq 0} + \sum \bbr_{\geq 0}[C_i] \]
        \end{step}
        why it is so long?


        \begin{step}[Test Name]\label{step: test label}
            This is a test.
        \end{step}


        \Yang{To be completed.}
    \end{proof}

    \begin{proof}
        The follows are test steps for the step environment.
        \begin{step}
            test again.
            In this step, we refer to \ref{step: test label} for a test.
        \end{step}
        \begin{step}
            This is a test.
            Test cref \cref{thm: base point free theorem}.
        \end{step}
    \end{proof}

    \begin{theorem}[Contraction Theorem]\label{thm: contraction theorem}
        Let \((X,B)\) be a projective klt pair and \(F \subset \Psef_1(X)\) a \(K_{(X,B)}\)-negative extremal face of \(\Psef_1(X)\).
        Then there exists a fibration \(\varphi_F: X \to Y\) of projective varieties such that
        \begin{enumerate}
            \item an irreducible curve \(C \subset X\) is contracted by \(\varphi_F\) if and only if \([C] \in F\);
            \item any line bundle \(\call\) with \(F \subset \call^{\perp} = \{\alpha \in N_1(X) : \alpha \cdot \call = 0\}\) comes from a line bundle on \(Y\), 
                i.e., there exists a line bundle \(\call_Y\) on \(Y\) such that \(\call \cong \varphi_F^* \call_Y\).
        \end{enumerate}
    \end{theorem}

    