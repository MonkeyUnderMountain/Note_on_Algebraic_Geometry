\section{The First Properties of Curves}

Let \(\kkk\) be an algebraically closed field. 
Unless otherwise specified, everything is defined over \(\kkk\).
A \emph{curve} is a one-dimensional variety.

\subsection{Riemann-Roch Theorem for Curves}

    \begin{theorem}[Riemann-Roch Theorem for Curves]\label{thm:Riemann-Roch_Theorem_for_Curves}
        Let \(C\) be a smooth proper curve of genus \(g\) over \(\kkk\).
        Then for any divisor \(D\) on \(C\), we have
        \[ h^0(D) - h^1(D) = \deg D + 1 - g. \]
        That is, the number \(\deg D + \chi(\calO_C(D))\) is independent of \(D\).
    \end{theorem}
    \begin{proof}
        \Yang{To be filled.}
    \end{proof}

\subsection{Classification of Curves}


\subsection{Hurwitz's Formula}

    \begin{theorem}[Hurwitz's Formula]\label{thm:Hurwitz_Formula}
        \Yang{To be filled.}
    \end{theorem}


\subsection{Positivity on Curves}

    \begin{theorem}\label{thm:bpf_and_very_ample_divisor_on_curve}
        Let \(C\) be a smooth proper curve of genus \(g\) over \(\kkk\) and \(D\) a divisor on \(C\).
        \begin{enumerate}
            \item If \(\deg D \geq 2g\), then \(D\) is base point free.
            \item If \(\deg D \geq 2g + 1\), then \(D\) is very ample.
        \end{enumerate}
    \end{theorem}
    \begin{proof}
        \Yang{To be filled.}
    \end{proof}
