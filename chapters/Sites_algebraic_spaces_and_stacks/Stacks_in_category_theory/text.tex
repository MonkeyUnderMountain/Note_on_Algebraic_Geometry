\section{Stacks in category theory}

\subsection{Fibered categories and descent conditions}

    \begin{definition}\label{def:strongly_Cartesian_morphism}
        Let \(\bfS\) be a category and \(\bfp: \bfX \to \bfS\) a functor.
        A morphism \(f: b \to a\) in \(\bfX\) is called \emph{strongly Cartesian} if for every object \(c \in \Obj(\bfX)\), the diagram
        \[ \begin{tikzcd}
            \Hom_{\bfX}(c, b) \arrow[r, "f \circ -"] \arrow[d, "\bfp"'] & \Hom_{\bfX}(c, a) \arrow[d, "\bfp"] \\
            \Hom_{\bfS}(w, v) \arrow[r, "\bfp(f) \circ -"] & \Hom_{\bfS}(w, u)
        \end{tikzcd} \]
        is a pullback of sets, where \(u=\bfp(a), v=\bfp(b),w=\bfp(c)\).
    \end{definition}

    The condition in \cref{def:strongly_Cartesian_morphism} can be interpreted as follows: for any diagram as below black part with \(\bfp(g) = \bfp(f) \circ \alpha\), 
    \[ \begin{tikzcd}
        c \arrow[rd,gray,"h"] \arrow[rrd,bend left,"g"] \arrow[dd, mapsto] & & \\
         & b \arrow[d,mapsto] \arrow[r, "f"] & a \arrow[d,mapsto] \\
        w \arrow[r, "\alpha"] & v \arrow[r, "\bfp(f)"] & u
    \end{tikzcd} \]
    there exists a unique gray morphism \(h: c \to a\) such that \(\bfp(h) = \alpha\) and \(f \circ h = g\).

    \begin{notation}\label{notation:object_and_morphisms_over_base}
        Let \(\bfS\) be a category and \(\bfp: \bfX \to \bfS\) a functor.
        For \(a,b \in \Obj(\bfX)\) and \(f\in \Hom_{\bfX}(a, b)\), we say that \(a\) is \emph{over} \(\bfp(a)\) and \(f\) is \emph{over} \(\bfp(f)\).
        In a diagram, we have 
        \[ \begin{tikzcd}
            \bfX \arrow[d, "\bfp"'] & a \arrow[d, mapsto] \arrow[r, "f"] & b \arrow[d, mapsto] \\
            \bfS & \bfp(a) \arrow[r, "\bfp(f)"] & \bfp(b)
        \end{tikzcd} \]
    \end{notation}

    \begin{definition}\label{def:fibered_categories}
        Let \(\bfS\) be a category.
        A category \(\bfX\) over \(\bfS\) via \(\bfp\) is called a \emph{category fibred} over the site \(\bfS\) if
        for every morphism \(\iota: v\to u\) in \(\bfS\) and every object \(a \in \Obj(\bfX)\) over \(u\), there exists an object \(b \in \Obj(\bfX)\) over \(v\) and a strongly Cartesian morphism \(f: b \to a\) over \(\iota\).
        Such an object \(b\) is called a \emph{pullback} of \(a\) along \(\iota\), and is often denoted by \(\iota^*a\).
    \end{definition}

    \begin{definition}\label{def:fiber_of_fiber_category}
        Let \(\bfS\) be a site and \(\bfp: \bfX \to \bfS\) a category fibred over \(\bfS\).
        For every object \(u \in \Obj(\bfS)\), the \emph{fiber} of \(\bfX\) over \(u\) is the category \(\bfX_u\) given by
        \[ \Obj(\bfX_u) = \{ a \in \Obj(\bfX) \mid \bfp(a) = u \}, \quad \Hom_{\bfX_u}(a,b) = \{ f \in \Hom_{\bfX}(a,b) \mid \bfp(f) = \id_u \}. \]
    \end{definition}

    \begin{remark}\label{rmk:choice_of_pullback_in_fibered_category}
        Note that in \cref{def:fibered_categories}, the pullback \(r^*b\) of an object \(b\) along a morphism \(r\) is not necessarily unique.
        \Yang{To be continued.}
    \end{remark}

    % \Yang{Why do we need the Cartesian morphisms exists?}

    \begin{example}\label{eg:presheaves_as_category_fibered_in_set}
        Let \(\bfS\) be a category and \(\calF: \bfS^{op} \to \Set\) be a presheaf on \(\bfS\) taking values in \(\Set\).
        We can construct a category \(\bfF\) fibred over \(\bfS\) as follows:
        \begin{itemize}
            \item The objects of \(\bfF\) are pairs \((U, x)\) where \(U \in \Obj(\bfS)\) and \(x \in \calF(U)\);
            \item morphisms from \((V, y)\) to \((U, x)\) in \(\bfF\) are morphisms \(\iota: V \to U\) in \(\bfS\) such that \(\calF(\iota)(x) = y\),
                denoted by \(\res_{\iota}\).
        \end{itemize}
        The functor \(\bfp: \bfF \to \bfS\) is defined by \(\bfp(U, x) = U\) on objects and \(\bfp(\iota) = \iota\) on morphisms.
        If one has the diagram
        \[ \begin{tikzcd}
            (W, z)  \arrow[rrd,bend left,"\res_{\tau}"] \arrow[dd, mapsto] & & \\
                    & (V, y) \arrow[d,mapsto] \arrow[r, "\res_{\iota}"] & (U, x) \arrow[d,mapsto] \\
            W \arrow[r, "\sigma"] & V \arrow[r, "\iota"] & U
        \end{tikzcd} \]
        with \(\bfp(\res_{\tau}) = \iota \circ \sigma\).
        By definition, we have \(\tau = \iota \circ \sigma\) and \(\calF(\tau)(x) = z\),\(\calF(\iota)(x) = y\).
        Thus, we have \(\calF(\sigma)(y) = z\).
        This verifies that \(\res_{\sigma}\) is a strongly Cartesian morphism.
        Note that the fiber of \(\bfF\) over an \(U \in \Obj(\bfS)\) is the discrete category associated to the set \(\calF(U)\).
        Therefore, presheaves of sets can be viewed as categories fibred in sets.

        Conversely, given a category \(\bfF\) fibred in sets over \(\bfS\) via \(\bfp: \bfF \to \bfS\),
        one can construct a presheaf of sets \(\calF: \bfS^{op} \to \Set\) by defining \(\calF(U) = \Obj(\bfF_U)\) for each \(U \in \Obj(\bfS)\),
        and for each morphism \(\iota: V \to U\) in \(\bfS\), defining \(\calF(\iota): \calF(U) \to \calF(V)\) by sending an object \(x \in \calF(U)\) to its pullback \(\iota^*x \in \calF(V)\) along \(\iota\).
        This establishes an equivalence between presheaves of sets on \(\bfS\) and categories fibred in sets over \(\bfS\).
    \end{example}

    \begin{example}\label{eg:toy_example_of_fibered_categories}
        \Yang{case \(\bfS = set, group\). To be added.}
    \end{example}

    \begin{slogan}\label{slogan:presheaf_is_set-fibered_category}
        Presheaves of sets are categories fibered in sets.
    \end{slogan}

    In following, we describe categories fibered in groupoids.

    \begin{definition}\label{def:presheaves_of_morphisms_associated_to_fibred_categories}
        Let \(\bfX\) be a category fibred over a category \(\bfS\) via \(\bfp: \bfX \to \bfS\).
        For every \(u \in \Obj(\bfS)\) and every pair of objects \(a,b\) over \(u\), we define the \emph{presheaf of morphisms} \(\Hom_{\bfX}(a,b): (\bfS/u)^{\op} \to \Set\) by
        \[ \Hom_{\bfX}(a,b)(\iota: v \to u) = \Hom_{\bfX_v}(\iota^*a, \iota^*b) \]
        for every morphism \(\iota: v \to u\) in \(\bfS/u\).
        For a morphism \(\alpha: w \to v\) in \(\bfS/u\), the restriction map
        \[ \Hom_{\bfX}(a,b)(\iota) \to \Hom_{\bfX}(a,b)(\iota \circ \alpha) \]
        is given by sending a morphism \(f: \iota^*a \to \iota^*b\) in \(\bfX_v\) to the pullback morphism 
        \Yang{\(\alpha^*f: (\iota \circ \alpha)^*a \to (\iota \circ \alpha)^*b\) need to conjugate with a natural transformation.}
        in \(\bfX_w\).
        \Yang{To be checked.}
    \end{definition}

    In a diagram, the presheaf of morphisms can be visualized as follows:
    \[\begin{tikzcd}
        && {\iota^*a} &&&& a \\
        {\tau^*a} &&& {\iota^*b} &&&& b \\
        & {\tau^*b} \\
        && v &&&& u \\
        w
        \arrow[from=1-3, to=1-7]
        \arrow[bend right,red!80!black, from=1-3, to=2-4]
        \arrow["f"{description}, bend left,red!80!black, from=1-3, to=2-4]
        \arrow[from=1-3,red!80!black, to=2-4]
        \arrow[maps to, from=1-3, to=4-3]
        \arrow[maps to, from=1-7, to=4-7]
        \arrow["{f|_w}"{description},blue!80!black, bend left, from=2-1, to=3-2]
        \arrow[bend right,blue!80!black, from=2-1, to=3-2]
        \arrow[from=2-1,blue!80!black, to=3-2]
        \arrow[maps to, from=2-1, to=5-1]
        \arrow[from=2-4, to=2-8]
        \arrow[maps to, from=2-4, to=4-3]
        \arrow[maps to, from=2-8, to=4-7]
        \arrow[maps to, from=3-2, to=5-1]
        \arrow["\iota",red!80!black,  from=4-3, to=4-7]
        \arrow["\alpha",from=5-1, to=4-3]
        \arrow["\tau", from=5-1, blue!80!black, to=4-7]
    \end{tikzcd}\]

    \begin{proposition}\label{prop:presheaves_of_morphisms_is_sheaf_iff_fibered_category_in_groupoids}
        Let \(\bfS\) be a category and \(\bfp: \bfX \to \bfS\) a category fibred over \(\bfS\).
        Then \(\bfX\) is a category fibred in groupoids if and only if for every object \(u \in \Obj(\bfS)\) and every pair of objects \(a,b\) over \(u\), the presheaf of morphisms \(\Hom_{\bfX}(a,b): (\bfS/u)^{\op} \to \Set\) is a sheaf.
        \Yang{To be checked.}
    \end{proposition}

    \begin{definition}\label{def:category_fibered_in_groupoids}
        Let \(\bfS\) be a category.
        A category \(\bfX\) fibred over \(\bfS\) via \(\bfp: \bfX \to \bfS\) is called a \emph{category fibred in groupoids} over \(\bfS\) if for every object \(u \in \Obj(\bfS)\) and every pair of objects \(a,b\) over \(u\), the presheaf of morphisms \(\Hom_{\bfX}(a,b): (\bfS/u)^{\op} \to \Set\) is a sheaf.
        \Yang{To be checked.}
    \end{definition}


    Now let us discuss how sheaves fit into the framework of fibered categories.
    Of course, we need assume the base category \(\bfS\) is a site.
    The glued condition for sheaves can be interpreted in terms of descent data in fibered categories.

    \begin{definition}\label{def:descent_data_in_categories}
        Let \(\bfS\) be a site and \(\bfp: \bfX \to \bfS\) a fibered category over \(\bfS\).
        Let \(U \in \Obj(\bfS)\) and \(\{ U_i \to U \}\) be a covering in \(\bfS\).
        A \emph{descent datum} for objects of \(\bfX\) relative to the covering \(\{ U_i \to U \}\) consists of
        \begin{itemize}
            \item a collection of objects \(a_i \in \Obj(\bfX_{U_i})\) for each \(i\),
            \item a collection of isomorphisms \(\varphi_{ij}: a_j|_{U_{ij}} \to a_i|_{U_{ij}}\) in \(\bfX_{U_{ij}}\) for each pair \((i,j)\), where \(U_{ij} = U_i \times_U U_j\),
        \end{itemize}
        such that the cocycle condition
        \[ \varphi_{ik}|_{U_{ijk}} = \varphi_{ij}|_{U_{ijk}} \circ \varphi_{jk}|_{U_{ijk}} \]
        holds for all triples \((i,j,k)\), where \(U_{ijk} = U_i \times_U U_j \times_U U_k\).
        \Yang{To be checked.}
    \end{definition}

    \begin{example}\label{eg:descent_data_for_presheaves}
        \Yang{To be added.}
    \end{example}

    \begin{definition}\label{def:effective_descent_data_in_categories}
        Let \(\bfS\) be a site and \(\bfp: \bfX \to \bfS\) a fibered category over \(\bfS\).
        A descent datum \((\{ a_i \}, \{ \varphi_{ij} \})\) for objects of \(\bfX\) relative to a covering \(\{ U_i \to U \}\) in \(\bfS\) is called \emph{effective} if there exists an object \(a \in \Obj(\bfX_U)\) and isomorphisms \(\psi_i: a|_{U_i} \to a_i\) in \(\bfX_{U_i}\) such that for all pairs \((i,j)\), the diagram
        \[ \begin{tikzcd}
            a|_{U_{ij}} \arrow[r, "\psi_j|_{U_{ij}}"] \arrow[d, "\psi_i|_{U_{ij}}"'] & a_j|_{U_{ij}} \arrow[d, "\varphi_{ij}"] \\
            a_i|_{U_{ij}} \arrow[r, "\varphi_{ij}"'] & a_j|_{U_{ij}}
        \end{tikzcd} \]
        commutes.
        \Yang{To be checked.}
    \end{definition}

    \begin{slogan}\label{slogan:descent_data_and_effectiveness}
        Descent data are like gluing data for objects, and effectiveness means that the glued object exists.
    \end{slogan}


\subsection{Prestacks and stacks}



    \begin{definition}\label{def:prestacks_as_functor}
        A \emph{prestack} over the site \(\bfS\) is a category \(\bfX\) fibered in groupoids over \(\bfS\).
    \end{definition}

    % \begin{remark}\label{rmk:prestack_compares_with_presheaf_as_functor}
    %     Let \(\bfS\) be a site.
    %     A presheaf of sets on \(\bfS\) can be viewed as a functor \(\bfS^{op} \to \Set\).
    %     A prestack over \(\bfS\) can be viewed as a functor \(\bfS^{op} \to \Grpd\) by associating to each object \(u \in \Obj(\bfS)\) the fiber category \(\bfX_u\), which is a groupoid, and to each morphism \(u \to v\) in \(\bfS\) the pullback functor \(\bfX_v \to \bfX_u\).
    %     Thus, prestacks can be seen as a generalization of presheaves of sets, where the values are groupoids instead of sets.
    %     \Yang{To be checked.}
    % \end{remark}

    \begin{slogan}\label{slogan:prestack_and_presheaf}
        Prestacks are ``presheaf remembering automorphisms''.
    \end{slogan}


    \begin{example}\label{eg:presheaf_is_prestack}
        presheaf is a prestack.
        \Yang{To be added.}
    \end{example}

    \begin{example}\label{eg:prestack_of_moduli_of_curves}
        The moduli problem of classifying algebraic curves of a fixed genus \(g\) can be formulated as a prestack over the site of schemes.
        Consider the category \(\bfM_g\) whose objects are families of smooth projective curves of genus \(g\) over schemes, and whose morphisms are isomorphisms of such families.
        The functor \(\bfp: \bfM_g \to \Sch\) sending a family of curves to its base scheme makes \(\bfM_g\) a category fibred in groupoids over \(\Sch\).
        For each scheme \(S\), the fiber category \(\bfM_{g,S}\) consists of families of smooth projective curves of genus \(g\) over \(S\) and their isomorphisms.
        The descent data for objects in \(\bfM_g\) relative to a covering of schemes correspond to gluing families of curves along isomorphisms on overlaps, which is effective due to the nature of algebraic curves.
        Thus, \(\bfM_g\) is a prestack over the site of schemes.
        \Yang{To be revised.}
    \end{example}

    % \Yang{Where is the 2-category?}

    % \begin{theorem}[Yoneda 2-Lemma]\label{thm:Yoneda_2_lemma}
    %     Let \(\bfS\) be a site, and let \(\bfp: \bfX \to \bfS\) and \(\bfq: \bfY \to \bfS\) be prestacks over \(\bfS\).
    %     Then the functor
    %     \[
    %         \Fun_{\bfS}(\bfX, \bfY) \to (\bfp_*, \bfq_*)
    %     \]
    %     given by \(\Phi \mapsto \Phi_*\) is an equivalence of categories.
    %     \Yang{To be revised.}
    % \end{theorem}


    \begin{proposition}\label{prop:existence_of_fiber_product_of_prestacks}
        Let \(\bfS\) be a site, and let \(\bfp: \bfX \to \bfS\), \(\bfq: \bfY \to \bfS\), and \(\bfr: \bfZ \to \bfS\) be prestacks over \(\bfS\).
        Let \(\Phi: \bfX \to \bfZ\) and \(\Psi: \bfY \to \bfZ\) be morphisms of prestacks over \(\bfS\).
        Then the fiber product \(\bfX \times_{\bfZ} \bfY\) exists in the category of prestacks over \(\bfS\).
        \Yang{To be checked.}
    \end{proposition}

    \begin{definition}\label{def:stacks_in_categories}
        Let \(\bfS\) be a site.
        A prestack \(\bfp: \bfX \to \bfS\) is called a \emph{stack} over the site \(\bfS\) if for every object \(U \in \Obj(\bfS)\) and every covering \(\{ U_i \to U \}\) in \(\bfS\), the descent data for objects of \(\bfX\) relative to the covering \(\{ U_i \to U \}\) are effective.
        \Yang{To be revised.}
    \end{definition}

    \begin{definition}\label{def:morphism_of_stacks}
        Let \(\bfS\) be a site, and let \(\bfp: \bfX \to \bfS\) and \(\bfq: \bfY \to \bfS\) be stacks over \(\bfS\).
        A \emph{morphism of stacks} \(F: \bfX \to \bfY\) over \(\bfS\) is a functor \(F: \bfX \to \bfY\) such that \(\bfq \circ F = \bfp\).
        \Yang{To be checked.}
    \end{definition}

    \begin{slogan}\label{slogan:stacks_and_sheaves}
        Stacks are to prestacks as sheaves are to presheaves.
    \end{slogan}

    \begin{example}\label{eg:functor_of_points_as_stacks}
        Let \(X\) be a scheme over a base noetherian scheme \(S\).
        The functor of points \(h_X: (\Sch/S)_{\et}^{\op} \to \Set\) is a sheaf, and thus a stack.
    \end{example}

    \begin{construction}\label{const:stackification_of_prestacks}
        Let \(\bfS\) be a site, and let \(\bfp: \bfX \to \bfS\) be a prestack over \(\bfS\).
        There exists a stack \(\bfp^{+}: \bfX^{+} \to \bfS\) over \(\bfS\) together with a morphism of prestacks \(F: \bfX \to \bfX^{+}\) over \(\bfS\) satisfying the following universal property:
        for every stack \(\bfp': \bfY \to \bfS\) over \(\bfS\) and every morphism of prestacks \(G: \bfX \to \bfY\) over \(\bfS\),
        there exists a unique morphism of stacks \(G^{+}: \bfX^{+} \to \bfY\) over \(\bfS\) such that \(G = G^{+} \circ F\).
        The stack \(\bfX^{+}\) is called the \emph{stackification} of the prestack \(\bfX\).
        \Yang{To be checked.}
    \end{construction}

    \begin{example}\label{const:quotient_stack_of_schemes}
        Let \(S\) be a noetherian scheme, and let \(G\) be a group scheme over \(S\) acting on a scheme \(X\) over \(S\) via a morphism \(\sigma: G \times_S X \to X\).
        The \emph{quotient stack} \([X/G]\) is defined as following:
        \begin{itemize}
            \item For each scheme \(U\) over \(S\), the objects of \([X/G](U)\) are pairs \((P, f)\) where \(P\) is a \(G\)-torsor over \(U\) and \(f: P \to X\) is a \(G\)-equivariant morphism over \(S\).
            \item Morphisms between two objects \((P, f)\) and \((P', f')\) in \([X/G](U)\) are given by \(G\)-equivariant morphisms \(\varphi: P \to P'\) over \(U\) such that \(f' \circ \varphi = f\).
        \end{itemize}
        The assignment \(U \mapsto [X/G](U)\) defines a stack over the site \((\Sch/S)_{\et}\).
        This stack captures the quotient of \(X\) by the action of \(G\) in a way that respects the group action and the torsor structure.
        \Yang{To be added.}
    \end{example}


    \begin{notation}
        As \cref{eg:presheaves_as_category_fibered_in_set}, we can associate a prestack \(\bfX\) over a \(\bfS\) to a functor \(\calX: \bfS^{\op} \to \Grpd\) by setting \(\bfX_u = \calX(u)\) for each \(u \in \Obj(\bfS)\) and defining the pullback functors accordingly.
        In particular, we can talk about representability of such prestacks.
        \Yang{To be revised.}
        \Yang{Why do not we just talk about sheaves of groupoid?}
    \end{notation}

    \begin{definition}\label{def:representable_of_morphisms_between_stacks}
        Let \(\bfS\) be a site, and let \(\bfX, \bfY\) be prestacks over \(\bfS\).
        A morphism of prestacks \(F: \bfX \to \bfY\) over \(\bfS\) is called \emph{representable} if for every \(\bfZ \to \bfY\) over \(\bfS\) with \(\bfZ\) representable in \(\bfS\),
        the fiber product \(\bfX \times_{\bfY} \bfZ\) is representable in \(\bfS\).
        % \Yang{To be checked.}
    \end{definition}


    % \begin{definition}\label{def:quotient_stack}
    %     Let \(\bfS\) be a site, and let \(G\) be a group object in \(\bfS\) acting on an object \(X \in \Obj(\bfS)\).
    %     The \emph{quotient stack} \([X/G]\) is the stack over \(\bfS\) defined as follows:
    %     \begin{itemize}
    %         \item For each object \(U \in \Obj(\bfS)\), the groupoid \([X/G](U)\) has as objects the pairs \((P, f)\), where \(P\) is a \(G\)-torsor over \(U\) and \(f: P \to X\) is a \(G\)-equivariant morphism.
    %         \item Morphisms between two objects \((P, f)\) and \((P', f')\) in \([X/G](U)\) are given by \(G\)-equivariant morphisms \(\varphi: P \to P'\) such that \(f' \circ \varphi = f\).
    %     \end{itemize}
    %     The assignment \(U \mapsto [X/G](U)\) defines a stack over \(\bfS\).
    %     \Yang{To be checked.}
    % \end{definition}