\section{Picard Groups of Abelian Varieties}

\subsection{Pullback along group operations}

    \begin{theorem}[Seesaw Theorem]\label{thm: seesaw theorem}
        Let \(A\) be an abelian variety over \(\kkk\). 
        
    \end{theorem}

    \begin{theorem}[Theorem of the cube]\label{thm: theorem of the cube}
        Let \(X,Y,Z\) be completed varieties over \(\kkk\) and \(\call\) a line bundle on \(X \times Y \times Z\). 
        Suppose that there exist \(x \in X(\kkk), y \in Y(\kkk), z \in Z(\kkk)\) such that the restriction \(\call|_{\{x\} \times Y \times Z}\), \(\call|_{X \times \{y\} \times Z}\) and \(\call|_{X \times Y \times \{z\}}\) are trivial. 
        Then \(\call\) is trivial.
    \end{theorem}
    \begin{proof}
        \Yang{To be completed.}
    \end{proof}

    \begin{remark}\label{rmk: theorem of the cube by rigidity lemma and picard scheme}
        If we assume the existence of the Picard scheme, then the theorem of the cube can be deduced from the Rigidity Lemma.
        \Yang{To be completed.}
    \end{remark}

    \begin{proposition}
        Let \(A\) be an abelian variety over \(\kkk\), \(f,g,h: X \to A\) morphisms from a variety \(X\) to \(A\) and \(\call\) a line bundle on \(A\).
        Then 
        \[ (f+g+h)^*\call \cong (f+g)^*\call \ten (f+h)^*\call \ten (g+h)^*\call \ten f^*\call^{-1} \ten g^*\call^{-1} \ten h^*\call^{-1}. \]
    \end{proposition}
    \begin{proof}
        \Yang{To be completed.}
    \end{proof}

    \begin{proposition}\label{prop: pull back of line bundles along the multiplication by n}
        Let \(A\) be an abelian variety over \(\kkk\), \(n\in \bbz\) and \(\call\) a line bundle on \(A\). 
        Then we have 
        \[ [n]_A^*\call \cong \call^{\ten \frac{1}{2}(n^2+n)} \ten [-1]_A^*\call^{\ten \frac{1}{2}(n^2-n)}. \]
    \end{proposition}
    \begin{proof}
        \Yang{To be completed.}
    \end{proof}

    \begin{theorem}[Theorem of the square]\label{thm: theorem of the square}
        Let \(A\) be an abelian variety over \(\kkk\), \(x,y \in A(\kkk)\) two points and \(\call\) a line bundle on \(A\).
        Then 
        \[ t_{x+y}^*\call \ten \call \cong t_x^*\call \ten t_y^*\call. \]
    \end{theorem}

    \begin{remark}\label{rmk: theorem of the square and homomorphism to Picard group}
        We can define a map
        \[ \Phi_\call: A(\kkk) \to \Pic(A), \quad x \mapsto t_x^*\call \ten \call^{-1}. \]
        Then theorem of the square implies that \(\Phi_\call\) is a homomorphism of groups.
        When we vary \(\call\), this gives an action of \(A(\kkk)\) on \(\Pic(A)\).
        \Yang{To be completed.}
    \end{remark}


\subsection{Positivity}

    \begin{theorem}\label{thm: abelian varieties are projective}
        Let $A$ be an abelian variety over $\kk$. 
        Then $A$ is projective.
    \end{theorem}
    \begin{proof}
        \Yang{To be completed.}
    \end{proof}


\subsection{Isogenies and finite subgroups}

    \begin{theorem}\label{thm: torsion subgroups of abelian varieties}
        Let \(A\) be an abelian variety of dimension \(d\) over \(\kkk\). 
        Then the subgroup \(A[n]\) of \(n\) torsion points is finite and we have 
        \begin{enumerate}
            \item if \(n\) is coprime to \(\characteristic(\kk)\), then \(A[n] \cong (\bbz/n\bbz)^{2d}\);
            \item if \(n = p^k\) for \(p = \characteristic(\kk) > 0\)
        \end{enumerate}
    \end{theorem}
    \begin{proof}
        \Yang{To be completed.}
    \end{proof}


\subsection{Dual abelian varieties}

    \begin{theorem}\label{thm: dual abelian varieties}
        Let $A$ be an abelian variety over $\kk$. 
        Then \(\Pic^0(A) \) has a natural structure of an abelian variety, called the \emph{dual abelian variety} of $A$, denoted by $A^\vee$.
    \end{theorem}

    \begin{proposition}\label{prop: the Poincare line bundle}
        
    \end{proposition}