\section{The First Properties of Abelian Varieties}

\subsection{Definition and examples of Abelian Varieties}
    \begin{theorem}[Rigidity Lemma]\label{thm: Rigidity Lemma}
        Let \(\pi_i: X \to Y_i\) be proper morphisms of varieties over a field \(\kk\) for \(i=1,2\).
        Suppose that \(\pi_1\) is a fibration and \(\pi_2\) contracts \(\pi_1^{-1}(y_0)\).
        Then there exists a rational map \(\varphi: Y_1 \ratmap Y_2\) such that \(\pi_2 \circ \varphi = \pi_1\) and \(\varphi\) is well-defined near \(Y_1 \setminus \{y_0\}\). 
    \end{theorem}

    \begin{definition}\label{def: abelian varieties}
        Let \(S\) be a scheme.
        An \emph{abelian scheme over \(S\)} is a group object in the category \(\Sch_S\) such that the structure morphism is proper, smooth and a fibration.
        If \(S = \Spec \kk\) for some field \(\kk\), then it is called an \emph{abelian variety over \(\kk\)}.
    \end{definition}
    
    \begin{example}\label{eg: elliptic curves as abelian varieties}
        
    \end{example}

    \begin{example}\label{eg: product of abelian varieties as abelian varieties}
        
    \end{example}

    \begin{example}\label{eg: base change of abelian varieties as abelian varieties}
        
    \end{example}

    In the following, we will always assume that \(A\) is an abelian variety over a field \(\kk\) of dimension \(d\).

    Temporarily, we will use the notation \(e_A,m_A,i_A\) to denote the identity section, multiplication morphism and inversion morphism of an abelian variety \(A\).

    \begin{proposition}\label{prop: abelian varieties are smooth}
        Let \( A \) be an abelian variety. 
        Then \( A \) is smooth.
    \end{proposition}
    \begin{proof}
        Note that there is an open subset \(U\subset A\) which is smooth.
        Then apply the left translation morphism \(l_a\).
    \end{proof}

    \begin{proposition}\label{prop: abelian varieties have trivial cotangent bundle}
        Let \( A \) be an abelian variety.
        Then the cotangent bundle \( \Omega_A \) is trivial, i.e., \(\Omega_A \cong \calo_A^{\oplus d}\) where \(d = \dim A\).
    \end{proposition}
    \begin{proof}
        \Yang{To be completed.}
    \end{proof}

    \begin{lemma}\label{lem: rigidity lemma for X product type}
        Let \(p: X \times Y \to Z\) be a proper morphism of varieties over \(\kk\) such that \(p\) contracts \(\{x_0\} \times Y\) for some point \(x_0 \in X\).
        Then there exists a unique morphism \(f: Y \to Z\) such that \(p = f \circ p_Y\).
    \end{lemma}
    \begin{proof}
        \Yang{To be completed.}
    \end{proof}

    \begin{theorem}\label{thm: structure of morphism between abelian varieties}
        Let \(A\) and \(B\) be abelian varieties. 
        Then any morphism \(f: A \to B\) with \(f(e_A) = e_B\) is a group homomorphism.
    \end{theorem}
    \begin{proof}
        \Yang{To be completed.}
    \end{proof}

    \begin{proposition}\label{prop: abelian varieties are abelian groups}
        Let \( A \) be an abelian variety. 
        Then \( A \) is an abelian group.
    \end{proposition}
    \begin{proof}
        Note that a group is abelian if and only if the inversion map is a homomorphism of groups.
        Then the conclusion follows from Theorem \ref{thm: structure of morphism between abelian varieties}.
    \end{proof}

    From now on, we will use the notation \(0, +, [-1]_A, t_a\) to denote the identity section, addition morphism, inversion morphism and translation by \(a\) of an abelian variety \(A\).
    For every \(n \in \bbn^*\), the homomorphism of multiplication by \(n\) is defined as 
    \[ [n]_A: A \xrightarrow{\Delta} A \times A \xrightarrow{[n-1]_A\times \id_A} A \times A \xrightarrow{+} A, \]
    where \(\Delta\) is the diagonal morphism.

    \begin{proposition}\label{prop: multiplication by n is etale}
        Let \(A\) be an abelian variety over \(\kkk\) and \(n\) a positive integer.
        Then the multiplication by \(n\) morphism \([n]_A: A \to A\) is finite surjective and \'etale.
    \end{proposition}
    \begin{proof}
        \Yang{To be completed.}
    \end{proof}

\subsection{Complex abelian varieties}

    \begin{theorem}\label{thm: complex abelian varieties are complex tori}
        Let \(A\) be a complex abelian variety.
        Then \(A\) is a complex torus, i.e., there exists a lattice \(\Lambda \subset \bbc^d\) such that \(A \cong \bbc^d / \Lambda\).
        Conversely, let \(A = \bbc^n/\Lambda\) be a complex torus for some lattice \(\Lambda\).
        Then \(A\) is a complex abelian variety if and only if \(\Lambda\) \Yang{To be completed.} 
    \end{theorem}


