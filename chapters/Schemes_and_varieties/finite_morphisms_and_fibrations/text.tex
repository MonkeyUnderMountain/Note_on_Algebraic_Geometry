\section{Finite morphisms and fibrations}

    The main references for this section are \cite{Har77}, \cite{Laz04a} and \cite{FOV99:joinsAndIntersections}.



\subsection{Finite morphisms}

    \begin{theorem}\label{thm:finite_preserve_ampleness}
        Let \(f: Y \to X\) be a finite morphism of schemes. 
        If \(\calL\) is an ample line bundle on \(X\), then \(f^*\calL\) is an ample line bundle on \(Y\).
        If and only if.
    \end{theorem}

\subsection{Fibrations}

    \begin{definition}\label{def:fibration}
        Let \(S\) be an integral excellent scheme and \(X\) be an integral scheme over \(S\) of finite type via the structure morphism \(f: X \to S\).
        Suppose that \(f\) is dominant.
        We say that \(f\) is a \emph{fibration} if \(\fkk(S)\) is algebraically closed in \(\fkk(X)\).
    \end{definition}

    \begin{theorem}\label{thm:general_irreducible_of_fibration}
        Let \(f: X \to S\) be a fibration.
        Then for a general point \(\sigma \in S\), the fiber \(X_\sigma\) is integral.
    \end{theorem}
    \begin{proof}

        \Yang{}
    \end{proof}

    \begin{slogan}
        Fibration is the relatively version of ``integral''.
    \end{slogan}

    \begin{proposition}\label{prop:fibration_for_proper_morphism}
        Let \(S\) be an integral excellent scheme and \(X\) be an integral scheme over \(S\) of finite type via the structure morphism \(f: X \to S\).
        Suppose that \(f\) is proper surjective and that \(S\) is normal.
        Then \(f\) is a fibration if and only if \(f_*\calO_X = \calO_S\).
    \end{proposition}
    \begin{proof}
        \Yang{}
    \end{proof}

    \begin{proposition}\label{prop:fibration_and_connected_fibers}
        Let \(f: X \to Y\) be a fibration with \(Y\) normal.
        Then the fibers of \(f\) are connected.
    \end{proposition}

    \begin{remark}\label{rmk:normality_in_definition_of_fibration}
        If we drop the normality assumption in Proposition \ref{prop:fibration_for_proper_morphism}, then the condition \(f_*\calO_X = \calO_Y\) is still sufficient to guarantee that \(f\) is a fibration.
        However, the converse may fail.

        \Yang{To be added.}
    \end{remark}


    \begin{theorem}[Zariski's Main Theorem]\label{thm:Zariski_main_theorem}
        Let \(f: Y \to X\) be a birational finite type morphism of excellent integral schemes.
        Suppose that \(X\) is normal.
        Then the fiber of \(f\) are connected.
    \end{theorem}

    \begin{theorem}[Stein factorization]\label{thm:Stein_factorization}
        Let \(f: Y \to X\) be a proper morphism of noetherian schemes.
        Then there exists a factorization
        \[
            Y \xrightarrow{g} Z \xrightarrow{h} X,
        \]
        where \(g\) is a proper morphism with connected fibers and \(h\) is a finite morphism.
        Moreover, this factorization is unique up to isomorphism.
        \Yang{To be checked.}
    \end{theorem}

    \begin{theorem}[Rigidity Lemma]\label{thm:rigidity_lemma}
        Let \(f: Y \to X\) be a fibration of noetherian schemes.
        Let \(g: Y \to Z\) be a morphism such that the restriction \(g|_{f^{-1}(x)} : f^{-1}(x) \to Z\) is constant for every point \(x \in X\).
        Then there exists a unique morphism \(h: X \to Z\) such that \(g = h \circ f\).
        \Yang{}
    \end{theorem}



% \subsection{Bertini}


    % \begin{theorem}\label{thm:general_fiber_of_fibration_is_integral}
    %     Let \(X \to Y\) be a fibration of varieties over a field \(k\).
    %     Then for a general point \(y \in Y\), the fiber \(X_y\) is integral.
    % \end{theorem}

    % \begin{theorem}\label{thm:general_fiber_of_fibration_is_R_n_S_n}
    %     Let \(X \to Y\) be a fibration of varieties over a field \(k\).
    %     Suppose that the total space \(X\) satisfies \(R_n\) (resp. \(S_n\)) for some integer \(n \geq 1\).
    %     Then for a general point \(y \in Y\), the fiber \(X_y\) also satisfies \(R_n\) (resp. \(S_n\)).
    % \end{theorem}

    % \begin{corollary}\label{cor:general_fiber_of_fibration_is_normal}
    %     Let \(X \to Y\) be a fibration of varieties over a field \(k\).
    %     Suppose that the total space \(X\) is normal.
    %     Then for a general point \(y \in Y\), the fiber \(X_y\) is also normal.
    % \end{corollary}


    % \begin{corollary}\label{thm:bertini_of_irreducibility}
    %     Let \(X \subseteq \bbP^n_k\) be an irreducible closed subvariety of dimension \(\geq 2\).
    %     Then for a general hyperplane \(H \subseteq \bbP^n_k\), the (scheme-theoretic) intersection \(X \cap H\) is integral.
    % \end{corollary}
    % \begin{proof}
    %     \Yang{}
    % \end{proof}

    % \begin{corollary}\label{thm:curve_through_finite_given_point}
    %     Let \(X\) be a quasi-projective variety over a field \(\kk\) of dimension \(\geq 2\).
    %     Given any finitely many closed points \(x_1, x_2, \ldots, x_r\in X(\kkk)\), there exists an irreducible curve \(C \subseteq X\) passing through all the given points.
    % \end{corollary}
    % \begin{proof}
    %     \Yang{}
    % \end{proof}

    % \begin{theorem}\label{thm:bertini_of_R_n_S_n}
    %     Let \(X \subseteq \bbP^n_k\) be a closed subvariety.
    %     Suppose that \(X\) satisfies \(R_n\) (resp. \(S_n\)) for some integer \(n \geq 1\).
    %     Then for a general hyperplane \(H \subseteq \bbP^n_k\), then the (scheme-theoretic) intersection \(X \cap H\) also satisfies \(R_n\) (resp. \(S_n\)).
    % \end{theorem}

    % \begin{corollary}\label{cor:bertini_of_normality}
    %     Let \(X \subseteq \bbP^n_k\) be a normal closed subvariety.
    %     Then for a general hyperplane \(H \subseteq \bbP^n_k\), the (scheme-theoretic) intersection \(X \cap H\) is also normal.
    % \end{corollary}