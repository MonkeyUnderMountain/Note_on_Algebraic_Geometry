\section{Normal, Cohen-Macaulay, and regular schemes}


\subsection{Normal schemes}

    \begin{definition}\label{def:normal_scheme}
        A scheme \(X\) is called \emph{normal} if for every open affine subset \(U = \Spec A\) of \(X\), the ring \(A\) is an integrally closed domain.
        \Yang{To be checked.}
    \end{definition}

    \begin{definition}\label{def:normalization}
        The \emph{normalization} of a scheme \(X\) is a normal scheme \(\widetilde{X}\) together with a finite birational morphism \(\pi:\widetilde{X}\to X\) such that for every normal scheme \(Y\) and every birational morphism \(f:Y\to X\), there exists a unique morphism \(g:Y\to \widetilde{X}\) such that \(f = \pi \circ g\).
        \Yang{To be checked.}
    \end{definition}

    \begin{theorem}\label{thm:existence_of_normalization}
        Let \(X\) be a scheme. Then there exists a normalization \(\widetilde{X}\) of \(X\).
    \end{theorem}

    \begin{theorem}[Hartog's phenomenon]\label{thm:Hartogs_phenomenon}
        Let \(X\) be a normal integral scheme, and let \(U\subseteq X\) be an open subset whose complement has codimension at least \(2\).
        Then every regular function on \(U\) extends uniquely to a regular function on \(X\), i.e., the restriction map \(\Gamma(X,\calO_X) \to \Gamma(U,\calO_X)\) is an isomorphism.
        \Yang{To be checked.}
    \end{theorem}

    \begin{proposition}\label{prop:normal_scheme_is_regular_in_codimension_1}
        Let \(X\) be a normal scheme, and let \(x \in X\) be a point with codimension \(1\). Then \(X\) is regular at \(x\).
    \end{proposition}

\subsection{Cohen-Macaulay schemes}

    \begin{definition}\label{def:local_cohomology}
        Let \(X\) be a scheme, and let \(Z\subseteq X\) be a closed subset.
        For a sheaf of \(\calO_X\)-modules \(\calF\), the \emph{local cohomology} \(H_Z^i(X,\calF)\) is defined as the \(i\)-th right derived functor of the functor \(\Gamma_Z(X,-):\Mod(\calO_X)\to \Ab\) that sends a sheaf of \(\calO_X\)-modules \(\calG\) to the abelian group of sections of \(\calG\) with support in \(Z\), i.e.,
        \[
            H_Z^i(X,\calF) := \sfR^i \Gamma_Z(X,\calF).
        \]
        \Yang{To be checked.}
    \end{definition}

    \begin{definition}\label{def:Cohen-Macaulay_scheme}
        A scheme \(X\) is called \emph{Cohen-Macaulay} if for every point \(x \in X\), the local ring \(\calO_{X,x}\) is a Cohen-Macaulay ring.
        \Yang{To be checked.}
    \end{definition}

    \begin{theorem}\label{thm:Cohen-Macaulay_scheme_and_vanishing_of_local_cohomology}
        Let \(X\) be a Cohen-Macaulay scheme, and let \(Z\subseteq X\) be a closed subset of codimension at least \(2\).
        Then for every sheaf of \(\calO_X\)-modules \(\calF\), the local cohomology \(H_Z^i(X,\calF) = 0\) for every \(i < 2\).
        \Yang{To be checked.}
    \end{theorem}

    \begin{definition}\label{def:regular_immersion}
        Let \(X\) be a scheme, and let \(Y \subseteq X\) be a closed subscheme defined by a sheaf of ideals \(\calI\).
        The closed immersion \(i:Y \to X\) is called a \emph{regular immersion} if for every point \(y \in Y\), the ideal \(\calI_y \subseteq \calO_{X,y}\) can be generated by a regular sequence.
        \Yang{To be checked.}
    \end{definition}

    \begin{definition}\label{def:locally_complete_intersection}
        A scheme \(X \to S\) is called a \emph{locally complete intersection} over \(S\) if the morphism \(X \to S\) can be factored as a regular immersion \(X \to Y\) followed by a smooth morphism \(Y \to S\).
        \Yang{To be completed.}
    \end{definition}


% \subsection{Locally complete intersection}


\subsection{Regular schemes}

    We first define the tangent space of a scheme at a point.

    There are many descriptions of the tangent space of a scheme at a point. 
    Here we give one of them.

    Let $X$ be a scheme over a field $\kk$, and let $x \in X(\kk)$.

    \begin{proposition}\label{prop:tangent_space_as_image_of_ring_of_dual_numbers}
        Let \(\Spec \kk[\epsilon]/(\epsilon^2)\) be the spectrum of the ring of dual numbers over \(\kk\) with point \(*:\Spec \kk \to \Spec \kk[\epsilon]/(\epsilon^2)\).
        The tangent space $T_x X$ is naturally isomorphic to the set of morphisms $\Spec \kk[\epsilon]/(\epsilon^2) \to X$ that send \(*\) to $x$, i.e.
        \[ T_x X \cong \{ f: \Spec \kk[\epsilon]/(\epsilon^2) \to X \mid f(*) = x \}. \]
    \end{proposition}
    \begin{proof}
        \Yang{To be filled.}
    \end{proof}