\section{Relative objects}

\subsection{Relative schemes}

    \begin{definition}\label{def:O_X_algebras_and_graded_O_X_algebras}
        Let \(X\) be a scheme.
        An \(\calO_X\)-algebra is a sheaf .
        \Yang{To be continued...}
    \end{definition}

    \begin{definition}\label{def:relative_Spec}
        Let \(X\) be a scheme and \(\calA\) be a quasi-coherent \(\calO_X\)-algebra.
        The relative Spec of \(\calA\), denoted by \(\calSpec_X \calA\), is the scheme obtained by gluing the affine schemes \(\Spec \calA(U) \to U\) for all affine open subsets \(U \subset X\).
        \Yang{To be continued...}
    \end{definition}

    \begin{proposition}\label{prop:geometric_vector_bundles}
        Let \(X\) be a scheme and \(\calE\) be a locally free sheaf of finite rank on \(X\).
        Then the relative Spec of the symmetric algebra of \(\calE\), denoted by \(\bbV(\calE) = \Spec_X \Sym_{\calO_X} \calE\), is called the geometric vector bundle associated to \(\calE\).
        The projection morphism \(\pi : \bbV(\calE) \to X\) is affine and for any open subset \(U \subset X\), we have \(\pi^{-1}(U) \cong \Spec \Sym_{\calO_X(U)} \calE(U)\).
        \Yang{To be continued...}
        \Yang{To be revised, need to take dual.}
    \end{proposition}

    \begin{definition}\label{def:relative_Proj}
        Let \(X\) be a scheme and \(\calA\) be a quasi-coherent graded \(\calO_X\)-algebra such that \(\calA_0 = \calO_X\) and \(\calA\) is generated by \(\calA_1\) as an \(\calO_X\)-algebra.
        The relative Proj of \(\calA\), denoted by \(\calProj_X \calA\), is the scheme obtained by gluing the affine schemes \(\Proj \calA(U)\) for all affine open subsets \(U \subset X\).
        The projection morphism \(\pi : \calProj_X \calA \to X\) is projective and for any open subset \(U \subset X\), we have \(\pi^{-1}(U) \cong \Proj \calA(U)\).
        \Yang{To be continued...}
    \end{definition}


\subsection{Relative ampleness and projective morphisms}

    \begin{definition}\label{def:relative_ample_line_bundle}
        Let \(X\) be an \(S\)-scheme via a morphism \(f : X \to S\).
        A line bundle \(\calL\) on \(X\) is called \emph{relatively ample} (or \(f\)-ample) if for every affine open subset \(U \subset S\), the restriction \(\calL|_{f^{-1}(U)}\) is an ample line bundle on the scheme \(f^{-1}(U)\).
        \Yang{To be revised.}
    \end{definition}

    \begin{definition}\label{def:projective_morphism}
        Let \(f : X \to S\) be a morphism of schemes.
        The morphism \(f\) is called \emph{projective} if there exists a quasi-coherent
    \end{definition}

    \begin{remark}\label{rmk:definitions_of_projective_morphism}
        There are subtle differences between various definitions of projective morphisms in the literature. (see \cite[abaaba]{FGA05})
        
    \end{remark}

\subsection{Blowing up}

    \begin{definition}\label{def:blow_up}
        Let \(X\) be a scheme and \(\calI \subset \calO_X\) be a quasi-coherent sheaf of ideals.
        The blow up of \(X\) along \(\calI\), denoted by \(\Bl_\calI X\), is defined to be the relative Proj of the Rees algebra of \(\calI\):
        \[
            \Bl_\calI X = \calProj_X \bigoplus_{n=0}^\infty \calI^n.
        \]
        % The projection morphism \(\pi : \Bl_\calI X \to X\) is projective and for any open subset \(U \subset X\), we have \(\pi^{-1}(U) \cong \Bl_{\calI(U)} U\).
        % The exceptional divisor of the blowing up is defined to be the closed subscheme \(E = \pi^{-1}(V(\calI))\) of \(\Bl_\calI X\).
        \Yang{To be continued...}
    \end{definition}

    \begin{proposition}\label{prop:blow_up_is_birational_and_projective}
        Let \(X\) be a scheme and \(\calI \subset \calO_X\) be a quasi-coherent sheaf of ideals.
        The blow up morphism \(\pi : \Bl_\calI X \to X\) is projective.
        Moreover, if the support of \(\calO_X / \calI\) is nowhere dense in \(X\), then \(\pi\) is birational.
        \Yang{To be continued...}
    \end{proposition}

    \begin{proposition}\label{prop:blow_up_regular_is_regular}
        Let \(X\) be a regular scheme and \(\calI \subset \calO_X\) be a quasi-coherent sheaf of ideals.
        Then the blow up \(\Bl_\calI X\) is also a regular scheme.
        \Yang{To be continued...}
    \end{proposition}

    \begin{proposition}\label{prop:exceptional_locus_of_blow_up}
        Let \(X\) be a scheme and \(\calI \subset \calO_X\) be a quasi-coherent sheaf of ideals.
        The exceptional locus of the blow up morphism \(\pi : \Bl_\calI X \to X\) is equal to the inverse image of the support of \(\calO_X / \calI\):
        \[
            \Exc(\pi) = \pi^{-1}(\Supp(\calO_X / \calI)).
        \]
        \Yang{To be continued...}
    \end{proposition}

% \subsection{Relative ampleness and relative morphisms}


