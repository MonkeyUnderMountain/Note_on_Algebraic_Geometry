\section{Differentials and duality}

Let \(\kkk\) be an algebraically closed field.
Unless otherwise specified, all schemes and varieties are assumed to be defined over \(\kkk\).

\subsection{The sheaves of differentials}

    \begin{definition}\label{def:canonical_divisor_on_normal_varieties}
        Let \(X\) be a normal variety over \(\kkk\) of dimension \(n\).
        If \(X\) is smooth, then the \emph{canonical divisor} \(K_X\) is defined to be \(c_1(\omega_X)\).
        In general, let \(U\subseteq X\) be the smooth locus of \(X\) and \(i:U\hookrightarrow X\) be the inclusion map.
        Then the \emph{canonical divisor} \(K_X\) is defined to be any Weil divisor on \(X\) such that \(\calO_X(K_X)\cong i_*\omega_U\).
        Note that \(U\) is big in \(X\) since \(X\) is normal, so such a Weil divisor always exists and is unique up to linear equivalence.
    \end{definition}

    
