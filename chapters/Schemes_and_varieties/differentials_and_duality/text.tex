\section{Differentials and duality}

Let \(S\) be a base noetherian scheme, \(\kkk\) be an algebraically closed field.
Unless otherwise specified, all schemes are assumed to be defined and of finite type over \(S\) and all varieties are assumed to be defined over \(\kkk\).

\subsection{The sheaves of differentials}

    \begin{definition}\label{def:sheaves_of_differentials}
        Let \(f:X \to S\) be an \(S\)-scheme.
        The \emph{sheaf of differentials} of \(X\) over \(S\), denoted by \(\Omega_{X/S}\), is the \(\calO_X\)-module locally given by
        \[ \Omega_{X/S}(U) = \Omega_{\calO_X(U)/\calO_S(V)} \]
        for any affine open subsets \(U \subseteq X\) and \(V \subseteq S\) with \(f(U) \subseteq V\).
    \end{definition}

    \begin{proposition}\label{prop:differential_sheaves_under_base_change}
        Let \(X\) and \(T\) be \(S\)-schemes and \(X_T := X \times_S T\) be the base change of \(X\) along \(T \to S\).
        Let \(p: X_T \to X\) be the projection morphism.
        Then there is a natural isomorphism of \(\calO_{X_T}\)-modules
        \[ \Omega_{X_T/T} \cong p^* \Omega_{X/S}. \]
    \end{proposition}
    \begin{proof}
        Given by algebras, see \Yang{ref.}
        \Yang{To be continued.}
    \end{proof}

    \begin{proposition}\label{prop:differentials_sheaves_under_localization}
        Let \(X\) be an \(S\)-scheme and \(U \subseteq X\) be an open subscheme.
        Then there is a natural isomorphism of \(\calO_U\)-modules
        \[ \Omega_{U/S} \cong \Omega_{X/S}|_U. \]
        Furthermore, let \(\xi \in X\), then there is a natural isomorphism of \(\calO_{X,\xi}\)-modules
        \[ \Omega_{X/S,\xi} \cong \Omega_{\calO_{X,\xi}/\calO_{S,f(\xi)}}. \]
        \Yang{To be checked.}
    \end{proposition}
    \begin{proof}
        \Yang{To be continued.}
    \end{proof}

    \begin{proposition}\label{prop:Omega_X_is_locally_free_when_X_is_regular_variety}
        Let \(X\) be a regular variety over \(\kkk\) of dimension \(n\).
        Then \(\Omega_{X/\kkk}\) is a locally free sheaf of rank \(n\).
    \end{proposition}
    \begin{proof}
        \Yang{To be continued.}
    \end{proof}

    \begin{proposition}\label{prop:Omega_X_is_reflexive_when_X_is_normal}
        Let \(X\) be a normal variety over \(\kkk\) of dimension \(n\).
        Then \(\Omega_{X/\kkk}\) is a reflexive sheaf of rank \(n\).
    \end{proposition}
    \begin{proof}
        \Yang{To be continued.}
    \end{proof}

    \begin{theorem}[Euler sequence for projective bundle]\label{thm:Euler_sequence_for_projective_bundle}
        Let \(X\) be a normal variety over \(\kkk\) and \(\calE\) be a locally free sheaf of rank \(r+1\) on \(X\).
        Let \(\pi: \bbP_X(\calE) \to X\) be the projective bundle associated to \(\calE\).
        Then there is an exact sequence of \(\calO_{\bbP_X(\calE)}\)-modules
        \[
            0 \to \Omega_{\bbP_X(\calE)/X} \to \pi^*\calE(-1) \to \calO_{\bbP_X(\calE)} \to 0.
        \]
        Here \(\pi^*\calE(-1)\) is twisted by the tautological line bundle \(\calO_{\bbP_X(\calE)}(-1)\).
    \end{theorem}
    \begin{proof}
        \begin{step}\label{step_in_Euler_sequence:affine_base}
            First assume that \(X = \Spec A\) is affine and \(\calE\) is free.
        \end{step}
        
        Fix a basis \(e_0, \ldots, e_r\) of the free \(A\)-module \(\calE(X)\).
        
        

        \Yang{To be continued.}
    \end{proof}

    \begin{definition}\label{def:canonical_divisor_on_normal_varieties}
        Let \(X\) be a normal variety over \(\kkk\).
        The \emph{canonical divisor} \(K_X\) of \(X\) is defined to be the Weil divisor class \(c_1(\Omega_{X/\kkk})\).
    \end{definition}

    \begin{example}\label{eg:canonical_divisor_of_projective_space}
        Let \(\bbP^n_\kkk\) be the projective space of dimension \(n\) over \(\kkk\).
        Then the canonical divisor \(K_{\bbP^n_\kkk} \sim -(n+1)H\), where \(H\) is a hyperplane in \(\bbP^n_\kkk\).
        \Yang{To be checked.}
    \end{example}

\subsection{Fundamental sequences}

    \begin{theorem}[The first fundamental sequence of differentials]\label{thm:the_first_fundamental_sequence_of_differentials}
        Let \(f: X \to Y\) be a morphism of schemes.
        Then there is a natural exact sequence of \(\calO_X\)-modules
        \[
            f^* \Omega_{Y/S} \to \Omega_{X/S} \to \Omega_{X/Y} \to 0.
        \]
    \end{theorem}
    \begin{proof}
        \Yang{To be completed.}
    \end{proof}

    \begin{proposition}\label{prop:the_first_fundamental_sequence_exact_when}
        Let \(f: X \to Y\) be a surjective and generically finite morphism of normal varieties over \(\kkk\).
        Then the first fundamental sequence of differentials is exact on the left.
    \end{proposition}
    \begin{proof}
        \Yang{To be completed.}
    \end{proof}
    
    \begin{corollary}[Ramification formula]\label{cor:ramification_formula}
        Let \(f: X \to Y\) be a finite morphism of normal varieties. 
        Then 
        \[
            K_X = f^*K_Y + R_f,
        \]
        where 
        \[ R_f := \sum_{D \subseteq X \text{ prime divisor}} \left( \Mult_D f^*\big(f(D)\big)-1 \right) D \]
        is the ramification divisor of \(f\).
        \Yang{To be checked. definition of ramification divisor needs to be checked.}
    \end{corollary}
    \begin{proof}
        \Yang{To be completed.}
    \end{proof}


    \begin{theorem}[The second fundamental sequence of differentials]\label{thm:the_second_fundamental_sequence_of_differentials}
        Let \(Z \subseteq X\) be a closed subscheme defined by the sheaf of ideals \(\calI \subseteq \calO_X\).
        Then there is a natural exact sequence of \(\calO_X\)-modules
        \[
            \calI/\calI^2 \to \Omega_{X/S}|_Z \to \Omega_{Z/S} \to 0.
        \]
        Suppose further that \(Z \to X\) is a regular immersion.
        Then the above sequence is also exact on the left.
    \end{theorem}
    \begin{proof}
        \Yang{To be completed.}
    \end{proof}
    
    \begin{corollary}[Adjunction formula]\label{cor:adjunction_formula_smooth_case}
        Let \(X\) be a normal variety and \(Z \subseteq X\) be a prime Cartier divisor which is normal as variety.
        Then 
        \[ K_Z = (K_X + Z)|_Z. \]
    \end{corollary}
    \begin{proof}
        Since both \(X\) and \(Z\) are normal, they are smooth in codimension \(1\).
        Removing the singular locus of \(X\) and \(Z\), we may assume that both \(X\) and \(Z\) are smooth varieties.
        This is valid since the canonical divisor is determined by the smooth locus.

        Since \(Z\) is Cartier, it is a local complete intersection in \(X\).
        By \cref{thm:the_second_fundamental_sequence_of_differentials}, we have the exact sequence
        \[ 0 \to \calI_Z/\calI_Z^2 \to \Omega_{X/\kkk}|_Z \to \Omega_{Z/\kkk} \to 0. \]
        Note that \(Z\) is of codimension \(1\) in \(X\), so \(\calI_Z \cong \calO_X(-Z)\) and thus \(\calI_Z/\calI_Z^2 \cong \calO_X(-Z)|_Z\).
        Taking \(c_1\), we obtain
        \[ c_1(\Omega_{X})|_Z = c_1(\Omega_{Z}) + c_1(\calO_X(-Z))|_Z. \]
        That is,
        \[ K_X|_Z = K_Z - Z|_Z. \]
        Rearranging gives the desired result.
        \Yang{To be revised. restriction of Weil divisors needs to be clarified.}
    \end{proof}

\subsection{Serre duality}

    \begin{definition}[Dualizing sheaf]\label{def:dualizing_sheaf}
        Let \(X\) be a proper scheme of dimension \(n\) over \(\kkk\).
        A \emph{dualizing sheaf} on \(X\) is a coherent sheaf \(\omega_X^\circ\) together with a trace map \(\tr_X: H^n(X, \omega_X^\circ) \to \kkk\) such that for every coherent sheaf \(\calF\) on \(X\), the natural pairing
        \[
            \Hom(\calF, \omega_X^\circ) \times H^n(X, \calF) \to H^n(X, \omega_X^\circ) \xrightarrow{\tr_X} \kkk
        \]
        induces an isomorphism
        \[
            \Hom(\calF, \omega_X^\circ) \cong H^n(X, \calF)^\vee.
        \]
    %    \Yang{To be revised.} 
    \end{definition}


    \begin{theorem}\label{thm:existence_of_dualizing_sheaf}
        Let \(X\) be a projective scheme of dimension \(n\) over \(\kkk\).
        Then there exists a dualizing sheaf \(\omega_X^\circ\) on \(X\) up to isomorphism.
        Moreover, if \(X\) is smooth, \(\omega_X^\circ \cong \omega_X = \bigwedge^n \Omega_{X/\kkk}\).
        % Then the dualizing sheaf \(\omega_X\) is a coherent sheaf on \(X\) and is in fact a dualizing complex concentrated in degree \(-n\).
        % \Yang{To be revised.}
    \end{theorem}
    \begin{proof}
        \Yang{To be completed.}
    \end{proof}


    \begin{theorem}[Serre duality]\label{thm:Serre_duality}
        Let \(X\) be a projective, Cohen-Macaulay variety of dimension \(n\) over \(\kkk\) with dualizing sheaf \(\omega_X^\circ\).
        Then for every coherent sheaf \(\calF\) on \(X\), there is a natural isomorphism
        \[ \Ext^i(\calF, \omega_X^\circ) \cong H^{n-i}(X, \calF)^\vee. \]
        % \Yang{there are some errors. Need to be revised}
    \end{theorem}
    \begin{proof}
        \Yang{To be completed.}
    \end{proof}

    \Yang{When \(\calF\) is locally free, we have \(\Ext^i(\calF,\omega_X^\circ) \cong H^i(X,\omega_X^\circ \ten \calF^\vee)\).}

    \begin{corollary}\label{cor:Serre_duality_for_the_canonical_divisor}
        Let \(X\) be a projective, normal variety of dimension \(n\) over \(\kkk\).
        Then for every integer \(m\) and \(0 \leq i \leq n\), there is a natural isomorphism
        \Yang{To be completed.}
    \end{corollary}

\subsection{Logarithm version}