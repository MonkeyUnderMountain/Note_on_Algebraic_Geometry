\section{Schemes as functors}


\subsection{The functor of points}

    Let \(X\) be a scheme over a base scheme \(S\).
    The \emph{functor of points} of \(X\) is the functor \(h_X(-)\colon (\Sch/S)^{\op} \to \Set\) defined by \(T \mapsto h_X(T) = \Hom_S(T, X)\).


\subsection{What is a scheme?}

    For a scheme \(X\) over \(S\), we will often identify \(X\) with its functor of points \(h_X\).
    In this way, we can think of a scheme as a functor from \((\Sch/S)^{\op}\) to \(\Set\).

    The underlying topological space of \(X\) can be recovered from the functor of points \(h_X\) as follows:
    The points of \(X\) correspond to the morphisms from the spectrum of a field to \(X\).

    The structure sheaf of \(X\) can also be recovered from the functor of points \(h_X\).