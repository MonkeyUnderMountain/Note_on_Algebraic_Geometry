\section{Line Bundles and Divisors}

\subsection{Cartier Divisors}

    \begin{definition}\label{def:Cartier_divisor}
        Let \(X\) be a scheme. 
        A \emph{Cartier divisor} on \(X\) is a global section of the sheaf of groups \(\calK_X^*/\calO_X^*\), where \(\calK_X\) is the sheaf of total quotient rings of \(X\).
        Equivalently, a Cartier divisor \(D\) can be represented by an open covering \(\{U_i\}\) of \(X\) and a collection of rational functions \(f_i \in \calK_X^*(U_i)\) such that for any \(i, j\), the function \(f_i/f_j \in \calO_X^*(U_i \cap U_j)\).
        We denote a Cartier divisor by \(D = \{(U_i, f_i)\}\).
    \end{definition}

\subsection{Line Bundles and Picard Group}

    \begin{definition}\label{def:picard_group}
        Let \(X\) be a scheme. 
        The \emph{Picard group} of \(X\) is defined to be \(\Pic(X) = H^1(X, \calO_X^*)\).
        The group operation is given by the tensor product of line bundles.
    \end{definition}

    \begin{definition}\label{def:algebraically_equivalent_line_bundles}
        Let \(X\) be a scheme over a field \(\kk\) and \(\calL, \calL'\) two line bundles on \(X\).
        We say that \(\calL\) and \(\calL'\) are \emph{algebraically equivalent} if there exists a \Yang{non-singular} variety \(T\) over \(\kk\), two points \(t_0, t_1 \in T(\kkk)\) and a line bundle \(\calM\) on \(X \times T\) such that 
        \[ \calM|_{X \times \{t_0\}} \cong \calL, \quad \calM|_{X \times \{t_1\}} \cong \calL'. \]
        We denote it by \(\calL \sim_{\text{alg}} \calL'\).
        \Yang{To be checked.}
    \end{definition}

\subsection{Weil Divisors and Reflexive Sheaves}

    To talk about Weil divisors, we need to work with normal schemes.

    \begin{definition}\label{def:Weil_divisor}
        Let \(X\) be a normal integral scheme.
        A \emph{Weil divisor} on \(X\) is a formal sum 
        \[ D = \sum_{Z} n_Z Z, \]
        where the sum runs over all prime divisors \(Z\) of \(X\) (i.e., integral closed subschemes of codimension 1) and \(n_Z \in \bbZ\), such that for any affine open subset \(U = \Spec A \subseteq X\), only finitely many \(Z\) intersecting \(U\) have nonzero coefficients \(n_Z\).
        The group of Weil divisors on \(X\) is denoted by \(\WDiv(X)\).
    \end{definition}

    \begin{definition}\label{def:reflexive_sheaves}
        Let \(X\) be a scheme and \(\calF\) a coherent sheaf on \(X\).
        The \emph{dual sheaf} of \(\calF\) is defined as \(\calF^\vee = \calHom_{\calO_X}(\calF, \calO_X)\).
        The sheaf \(\calF\) is called \emph{reflexive} if the natural map \(\calF \to (\calF^\vee)^\vee\) is an isomorphism.
    \end{definition}
