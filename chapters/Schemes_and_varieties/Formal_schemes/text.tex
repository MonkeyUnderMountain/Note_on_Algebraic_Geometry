\section{Formal schemes}


\subsection{Definitions and examples}

    \begin{definition}\label{def:formal_completion}
        Let \(X\) be a noetherian scheme, and let \(Z\subseteq X\) be a closed subset defined by a sheaf of ideals \(\calI\).
        The \emph{formal completion} of \(X\) along \(Z\) is the ringed space \((Z, \calO_X^{\wedge})\), where \(\calO_X^{\wedge} = \varprojlim \calO_X/\calI^n\).
        \Yang{To be added.}
    \end{definition}

    \begin{definition}\label{def:formal_schemes}
        Let \(X\) be a noetherian scheme, and let \(Z\subseteq X\) be a closed subset. 
        The \emph{formal completion} of \(X\) along \(Z\) is the ringed space \((Z, \calO_X^{\wedge})\), where \(\calO_X^{\wedge} = \varprojlim \calO_X/\calI^n\) with \(\calI\) being the sheaf of ideals defining \(Z\).
        A \emph{formal scheme} is a ringed space that is locally isomorphic to such a formal completion.
    \end{definition}


\subsection{Theorem on formal functions}

    \begin{theorem}\label{thm:theorem_on_formal_functions}
        
    \end{theorem}