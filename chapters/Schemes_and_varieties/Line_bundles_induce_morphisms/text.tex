\section{Line bundles induce morphisms}


\subsection{Ample and basepoint free line bundles}

    The story begins with the following theorem, which uses global sections of a line bundle to construct a morphism to projective space.

    \begin{theorem}\label{thm:morphism_to_projective_space}
        Let \(A\) be a ring and \(X\) an \(A\)-scheme.
        Let \(\calL\) be a line bundle on \(X\) and \(s_0,\ldots,s_n\in\Gamma(X,\calL)\).
        Suppose that \(\{s_i\}\) generate \(\calL\), i.e., \(\bigoplus_i \calO_X s_i \to \calL\) is surjective.
        Then there is a unique morphism \(f:X\to \bbP^n_A\) such that \(\calL\cong f^*\calO(1)\) and \(s_i=f^*x_i\), where \(x_i\) are the standard coordinates on \(\bbP^n_A\).   
    \end{theorem}
    \begin{proof}
        \Yang{To be continued.}
    \end{proof}


    \begin{definition}\label{def:ample_line_bundle}
        A line bundle \(\calL\) on a scheme \(X\) is \emph{ample} if for every coherent sheaf \(\calF\) on \(X\), there exists \(n_0>0\) such that for all \(n\ge n_0\), \(\calF\otimes \calL^{\otimes n}\) is globally generated.
        \Yang{To be continued.}
    \end{definition}

    \begin{definition}\label{def:very_ample_line_bundle}
        A line bundle \(\calL\) on a scheme \(X\) is \emph{very ample} if there exists a closed embedding \(i:X\to \bbP^n_A\) such that \(\calL\cong i^*\calO(1)\).
        \Yang{To be continued.}
    \end{definition}

    \begin{definition}\label{def:base_locus}
        Let \(\calL\) be a line bundle on a scheme \(X\) and \(V\subseteq \Gamma(X,\calL)\) a subspace.
        The \emph{base locus} of \(V\) is the closed subset
        \[
            \Bs(V) = \{x\in X : s(x)=0 \text{ for all } s\in V\}.
        \]
        If \(\Bs(V)=\emptyset\), we say that \(V\) is \emph{base-point free}.
        \Yang{To be continued.}
    \end{definition}

    \begin{definition}\label{def:linear_system}
        A \emph{linear system} on a scheme \(X\) is a pair \((\calL,V)\) where \(\calL\) is a line bundle on \(X\) and \(V\subseteq \Gamma(X,\calL)\) is a subspace.
        The dimension of the linear system is \(\dim V - 1\).
        A linear system is \emph{base-point free} if \(V\) is base-point free.
        A linear system is \emph{complete} if \(V=\Gamma(X,\calL)\).
        \Yang{To be continued.}
    \end{definition}

    \begin{theorem}\label{thm:ample_very_ample_and_bpf}
        Let \(X\) be a scheme over a ring \(A\) and \(\calL\) a line bundle on \(X\).
        Then the following are equivalent:
        \begin{enumerate}
            \item \(\calL\) is ample.
            \item For some \(n>0\), \(\calL^{\otimes n}\) is very ample.
            \item For some \(n>0\), \(\calL^{\otimes n}\) is base-point free.
            \item For every coherent sheaf \(\calF\) on \(X\), there exists \(n_0>0\) such that for all \(n\ge n_0\), \(\calF\otimes \calL^{\otimes n}\) is generated by global sections.
        \end{enumerate}
        \Yang{To be continued.}
    \end{theorem}

\subsection{Asymptotic behavior}

    \begin{definition}\label{def:section_ring}
        Let \(X\) be a scheme and \(\calL\) a line bundle on \(X\).
        The \emph{section ring} of \(\calL\) is the graded ring
        \[
            R(X,\calL) = \bigoplus_{n\ge 0} \Gamma(X,\calL^{\otimes n}),
        \]
        with multiplication induced by the tensor product of sections.
        \Yang{To be continued.}
        
    \end{definition}

    \begin{definition}\label{def:semiample_line_bundle}
        A line bundle \(\calL\) on a scheme \(X\) is \emph{semiample} if for some \(n>0\), \(\calL^{\otimes n}\) is base-point free.
        \Yang{To be continued.}
        
    \end{definition}

    \begin{theorem}\label{thm:fibration_associated_to_semiample_line_bundle}
        Let \(X\) be a scheme over a ring \(A\) and \(\calL\) a semiample line bundle on \(X\).
        Then there exists a morphism \(f:X\to Y\) over \(A\) such that \(\calL\cong f^*\calO_Y(1)\) for some very ample line bundle \(\calO_Y(1)\) on \(Y\).
        Moreover, \(Y=\Proj R(X,\calL)\) and \(f\) is induced by the natural map \(R(X,\calL)\to \Gamma(X,\calL^{\otimes n})\).
        \Yang{To be continued.  }
        
    \end{theorem}

    \begin{definition}\label{def:big_line_bundle}
        A line bundle \(\calL\) on a scheme \(X\) is \emph{big} if the section ring \(R(X,\calL)\) has maximal growth, i.e., there exists \(C>0\) such that
        \[
            \dim \Gamma(X,\calL^{\otimes n}) \ge C n^{\dim X}
        \]
        for all sufficiently large \(n\).
        \Yang{To be continued.}
        
    \end{definition}


\subsection{Iitaka fibration}

    \begin{theorem}\label{thm:iitaka_fibration}
        Let \(X\) be a projective variety over a field \(k\) and \(\calL\) a line bundle on \(X\).
        Then there exists a unique rational map \(f:X\dashrightarrow Y\) to a projective variety \(Y\) such that:
        \begin{enumerate}
            \item The general fiber of \(f\) is connected.
            \item The dimension of \(Y\) is equal to the Iitaka dimension of \(\calL\), i.e., the transcendence degree of the section ring \(R(X,\calL)\) minus one.
            \item For some \(n>0\), the linear system associated to \(\calL^{\otimes n}\) defines the map \(f\).
        \end{enumerate}
        The map \(f\) is called the \emph{Iitaka fibration} associated to \(\calL\).
        \Yang{To be continued.}
        
    \end{theorem}