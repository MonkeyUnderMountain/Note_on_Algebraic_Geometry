\section{Projective morphisms and ``positive'' line bundles}


\subsection{Ample line bundles}

    \begin{definition}\label{def:very_ample_line_bundle}
        A line bundle \(\calL\) on a scheme \(X\) is \emph{very ample} if there exists a closed embedding \(i:X\to \bbP^n_A\) such that \(\calL\cong i^*\calO(1)\).
        \Yang{To be continued.}
    \end{definition}

    \begin{theorem}[Serre Vanishing]\label{thm:Serre_vanishing}
        Let \(X\) be a projective scheme over a field \(k\) and \(\calL\) an ample line bundle on \(X\). 
        Then for any coherent sheaf \(\calF\) on \(X\), there exists an integer \(N\) such that for all \(n \geq N\), we have
        \[ H^i(X, \calF \otimes \calL^{\otimes n}) = 0 \]
    \end{theorem}

    \begin{definition}\label{def:Hilbert_polynomial_w.r.t_ample_line_bundles}
        Let \((X,\calO_X(1))\) be a projective scheme over a field \(k\) and \(\calF\) a coherent sheaf on \(X\).
        The \emph{Hilbert polynomial} of \(\calF\) with respect to \(\calO_X(1)\) is the polynomial 
        \[
            P_{\calF}(n) = \chi(X, \calF \otimes \calO_X(n)) = \sum_{i=0}^{\infty} (-1)^i \dim_k H^i(X, \calF \otimes \calO_X(n)).
        \]
        \Yang{To be continued.}

        Let \(Z \subseteq X\) be a closed subscheme with ideal sheaf \(\calI_Z\).
        The \emph{Hilbert polynomial} of \(Z\) with respect to \(\calO_X(1)\) is defined as \(P_Z(n) = P_{\calO_X/\calI_Z}(n)\).
        \Yang{To be revised.}
    \end{definition}

\subsection{Ample and basepoint free line bundles}

    The story begins with the following theorem, which uses global sections of a line bundle to construct a morphism to projective space.

    \begin{theorem}\label{thm:morphism_to_projective_space}
        Let \(A\) be a ring and \(X\) an \(A\)-scheme.
        Let \(\calL\) be a line bundle on \(X\) and \(s_0,\ldots,s_n\in\Gamma(X,\calL)\).
        Suppose that \(\{s_i\}\) generate \(\calL\), i.e., \(\bigoplus_i \calO_X\cdot s_i \to \calL\) is surjective.
        Then there is a unique morphism \(f:X\to \bbP^n_A\) such that \(\calL\cong f^*\calO(1)\) and \(s_i=f^*x_i\), where \(x_i\) are the standard coordinates on \(\bbP^n_A\). 
        \Yang{We need a more ``functorial'' expression.}  
    \end{theorem}
    \begin{proof}
        Let \(U_i\coloneqq \{\xi \in X: s_i(\xi) \not\in \frakm_\xi \calL_\xi\}\) be the open subset where \(s_i\) does not vanish.
        Since \(\{s_i\}\) generate \(\calL\), we have \(X=\bigcup_i U_i\).
        Let \(V_i\) be given by \(x_i \neq 0\) in \(\bbP^n_A\).
        On \(U_i\), let \(f_i:U_i\to V_i \subseteq \bbP^n_A\) be the morphism induced by the ring homomorphism
        \[ A\left[\frac{x_0}{x_i},\ldots,\frac{x_n}{x_i}\right] \to \Gamma(U_i,\calO_X), \quad \frac{x_j}{x_i} \mapsto \frac{s_j}{s_i}. \]
        Easy to check that on \(U_i\cap U_j\), \(f_i\) and \(f_j\) agree.
        Thus we can glue them to get a morphism \(f:X\to \bbP^n_A\).
        By construction, we have \(s_i=f^*x_i\) and \(\calL\cong f^*\calO(1)\).
        If there is another morphism \(g:X\to \bbP^n_A\) satisfying the same properties, then on each \(U_i\), \(g\) must agree with \(f_i\) by the same construction.
        Thus \(g=f\).
    \end{proof}

    \begin{proposition}\label{prop:different_choices_of_sections_give_different_morphisms_which_differ_by_a_linear_transformation}
        Let \(X\) be a \(\kk\)-scheme for some field \(\kk\) and \(\calL\) is a line bundle on \(X\).
        Suppose that \(\{s_0,\ldots,s_n\}\) and \(\{t_0,\ldots,t_m\}\) span the same subspace \(V\subseteq \Gamma(X,\calL)\) and both generate \(\calL\).
        Let \(f:X\to \bbP^n_\kk\) and \(g:X\to \bbP^m_\kk\) be the morphisms induced by \(\{s_i\}\) and \(\{t_j\}\) respectively.
        Then there exists a linear transformation \(\phi:\bbP^n_\kk \ratmap \bbP^m_\kk\) which is well defined near image of \(f\) and satisfies \(g=\phi \circ f\).
    \end{proposition}
    \begin{proof}
        \Yang{To be continued.}
    \end{proof}

    \begin{example}\label{eg:d-uple_embedding_or_Veronese_embedding}
        Let \(X=\bbP^n_A\) with \(A\) a ring and \(\calL=\calO_{\bbP^n}(d)\) for some \(d>0\).
        Then \(\Gamma(X,\calL)\) is generated by the global sections \(S_{i_0,\ldots,i_n}=T_0^{i_0}T_1^{i_1}\cdots T_n^{i_n}\) for all \((i_0,\ldots,i_n)\) with \(i_0+\cdots+i_n=d\), where \(T_i\) are the standard coordinates on \(\bbP^n\).
        The they induce a morphism \(f:X\to \bbP^N_A\) where \(N=\binom{n+d}{d}-1\).
        If \(A=\kk\) is a field, on \(\kk\)-point level, it is given by
        \[
            [x_0:\cdots:x_n] \mapsto [\ldots:x_0^{i_0}x_1^{i_1}\cdots x_n^{i_n}:\ldots],
        \]
        where the coordinates on the right-hand side are indexed by all \((i_0,\ldots,i_n)\) with \(i_0+\cdots+i_n=d\).
        This is called the \emph{\(d\)-uple embedding} or \emph{Veronese embedding} of \(\bbP^n\) into \(\bbP^N\).
    \end{example}

    \begin{example}\label{eg:Segre_embedding}
        Let \(X=\bbP^m_A \times_A \bbP^n_A\) with \(A\) a ring and \(\calL=\pi_1^*\calO_{\bbP^m}(1) \otimes \pi_2^*\calO_{\bbP^n}(1)\), where \(\pi_1\) and \(\pi_2\) are the projections.
        Let \(T_0,\ldots,T_m\) and \(S_0,\ldots,S_n\) be the standard coordinates on \(\bbP^m\) and \(\bbP^n\) respectively.
        Then \(\Gamma(X,\calL)\) is generated by the global sections \(T_i S_j = \pi_1^*T_i \otimes \pi_2^*S_j\) for \(0\leq i \leq m\) and \(0\leq j \leq n\).
        They induce a morphism \(f:X\to \bbP^{(m+1)(n+1)-1}_A\).
        If \(A=\kk\) is a field, on \(\kk\)-point level, it is given by
        \[ ([x_0:\cdots:x_m],[y_0:\cdots:y_n]) \mapsto [\ldots:x_i y_j:\ldots], \]
        where the coordinates on the right-hand side are indexed by all \((i,j)\) with \(0\leq i \leq m\) and \(0\leq j \leq n\).
        This is called the \emph{Segre embedding} of \(\bbP^m \times \bbP^n\) into \(\bbP^{(m+1)(n+1)-1}\).
    \end{example}

    % \begin{example}\label{eg:morphism_induced_by_-K_of_Hirzebruch_surface_F_2}
    %     Let \(X=\bbF_2\) be the second Hirzebruch surface, i.e., the projective bundle \(\bbP_{\bbP^1}(\calO_{\bbP^1}\oplus \calO_{\bbP^1}(-2))\) over \(\bbP^1\).
    %     \Yang{To be continued.}
    % \end{example}

    \begin{definition}\label{def:globally_generated_line_bundle}
        A line bundle \(\calL\) on a scheme \(X\) is \emph{globally generated} if \(\Gamma(X,\calL)\) generates \(\calL\), i.e., the natural map \(\Gamma(X,\calL)\otimes \calO_X \to \calL\) is surjective.
        \Yang{To be continued.}
    \end{definition}

    \begin{example}\label{eg:multiple_of_globally_generated_line_bundle_is_globally_generated}
        Let 
    \end{example}

    \begin{example}\label{eg:product_of_pullback_along_natural_projection_is_globally_generated}
        
    \end{example}

    \begin{definition}\label{def:base_locus_and_base_idea}
        Let \(\calL\) be a line bundle on a scheme \(X\).
        \Yang{To be continued.}
    \end{definition}


    \begin{definition}\label{def:ample_line_bundle}
        A line bundle \(\calL\) on a scheme \(X\) is \emph{ample} if for every coherent sheaf \(\calF\) on \(X\), there exists \(n_0>0\) such that for all \(n\ge n_0\), \(\calF\otimes \calL^{\otimes n}\) is globally generated.
        \Yang{To be continued.}
    \end{definition}

    

    % \begin{definition}\label{def:base_locus}
    %     Let \(\calL\) be a line bundle on a scheme \(X\) and \(V\subseteq \Gamma(X,\calL)\) a subspace.
    %     The \emph{base locus} of \(V\) is the closed subset
    %     \[
    %         \Bs(V) = \{x\in X : s(x)=0, \forall s\in V\}.
    %     \]
    %     If \(\Bs(V)=\emptyset\), we say that \(V\) is \emph{base-point free}.
    %     \Yang{To be continued.}
    % \end{definition}



    % \begin{definition}\label{def:linear_system}
    %     A \emph{linear system} on a scheme \(X\) is a pair \((\calL,V)\) where \(\calL\) is a line bundle on \(X\) and \(V\subseteq \Gamma(X,\calL)\) is a subspace.
    %     The dimension of the linear system is \(\dim V - 1\).
    %     A linear system is \emph{base-point free} if \(V\) is base-point free.
    %     A linear system is \emph{complete} if \(V=\Gamma(X,\calL)\).
    %     \Yang{To be continued.}
    % \end{definition}

    \begin{theorem}\label{thm:ample_very_ample}
        Let \(X\) be a scheme of finite type over a noetherian ring \(A\) and \(\calL\) a line bundle on \(X\).
        Then the following are equivalent:
        \begin{enumerate}
            \item \(\calL\) is ample;
            \item for some \(n>0\), \(\calL^{\otimes n}\) is very ample;
            \item for all \(n \gg 0\), \(\calL^{\otimes n}\) is very ample.
        \end{enumerate}
        \Yang{To be continued.}
    \end{theorem}

    \begin{proposition}\label{prop:tensor_with_ample_very_ample_and_bpf}
        Let \(X\) be a scheme of finite type over a noetherian ring \(A\) and \(\calL\), \(\calM\) line bundles on \(X\).
        Then we have the following:
        \begin{enumerate}
            \item if \(\calL\) is ample and \(\calM\) is globally generated, then \(\calL \otimes \calM\) is ample;
            \item if \(\calL\) is very ample and \(\calM\) is globally generated, then \(\calL \otimes \calM\) is very ample;
            \item if both \(\calL\) and \(\calM\) are ample, then so is \(\calL \otimes \calM\);
            \item if both \(\calL\) and \(\calM\) are globally generated, then so \(\calL \otimes \calM\);
            \item if \(\calL\) is ample and \(\calM\) is arbitrary, then for some \(n>0\), \(\calL^{\otimes n} \ten \calM\) is ample;
        \end{enumerate}
        \Yang{To be continued.}
    \end{proposition}
    \begin{proof}
        \Yang{To be continued.}
    \end{proof}

    % \begin{proposition}\label{prop:very_ample_iff_separates_points_and_vectors}
    %     Let \(X\) be a scheme of finite type over a noetherian ring \(A\) and \(\calL\) a line bundle on \(X\).
    %     Then \(\calL\) is very ample if and only if the following two conditions hold:
    %     \begin{enumerate}
    %         \item (separate points) for any two distinct points \(x,y\in X\), there exists \(s\in \Gamma(X,\calL)\) such that \(s(x)=0\) but \(s(y)\neq 0\);
    %         \item (separate tangent vectors) for any point \(x\in X\) and non-zero tangent vector \(v\in T_xX\), there exists \(s\in \Gamma(X,\calL)\) such that \(s(x)=0\) but \(v(s)\neq 0\).
    %     \end{enumerate}
    %     \Yang{To be continued.}
    % \end{proposition}
    

\subsection{Linear systems}

    In this subsection, when work over a field, we give a more geometric interpretation of last subsection using the language of linear systems.

    \begin{definition}\label{def:geometric_complete_linear_system}
        Let \(X\) be a normal proper variety over a field \(\kk\), \(D\) a (Cartier) divisor on \(X\) and \(\calL=\calO_X(D)\) the associated line bundle.
        The \emph{complete linear system} associated to \(D\) is the set 
        \[ |D| = \{D'\in \CaDiv(X) : D'\sim D, D' \geq 0\}. \]
    \end{definition}

    There is a natural bijection between the complete linear system \(|D|\) and the projective space \(\bbP(\Gamma(X,\calL))\).
    Here the elements in \(\bbP(\Gamma(X,\calL))\) are one-dimensional subspaces of \(\Gamma(X,\calL)\).
    Consider the vector subspace \(V\subseteq \Gamma(X,\calL)\), we can define the generate linear system \(|V|\) as the image of \(V\setminus \{0\}\) in \(\bbP(\Gamma(X,\calL))\).

    % \begin{definition}\label{def:general_linear_system}
    %     A \emph{linear system} on a scheme \(X\) is a pair \((\calL,V)\) where \(\calL\) is a line bundle on \(X\) and \(V\subseteq \Gamma(X,\calL)\) is a subspace.
    %     The dimension of the linear system is \(\dim V - 1\).
    %     A linear system is \emph{base-point free} if \(V\) is base-point free.
    %     A linear system is \emph{complete} if \(V=\Gamma(X,\calL)\).
    %     \Yang{To be continued.}
    % \end{definition}

    % \begin{definition}\label{def:base_locus}
    %     Let \(\calL\) be a line bundle on a scheme \(X\) and \(V\subseteq \Gamma(X,\calL)\) a subspace.
    %     The \emph{base locus} of \(V\) is the closed subset
    %     \[
    %         \Bs(V) = \{x\in X : s(x)=0, \forall s\in V\}.
    %     \]
    %     If \(\Bs(V)=\emptyset\), we say that \(V\) is \emph{base-point free}.
    %     \Yang{To be continued.}
    % \end{definition}


\subsection{Asymptotic behavior}

    \begin{definition}\label{def:section_ring}
        Let \(X\) be a scheme and \(\calL\) a line bundle on \(X\).
        The \emph{section ring} of \(\calL\) is the graded ring
        \[
            R(X,\calL) = \bigoplus_{n\ge 0} \Gamma(X,\calL^{\otimes n}),
        \]
        with multiplication induced by the tensor product of sections.
        \Yang{To be continued.}
    \end{definition}

    \begin{definition}\label{def:semiample_line_bundle}
        A line bundle \(\calL\) on a scheme \(X\) is \emph{semiample} if for some \(n>0\), \(\calL^{\otimes n}\) is base-point free.
        \Yang{To be continued.}
    \end{definition}

    \begin{theorem}\label{thm:fibration_associated_to_semiample_line_bundle}
        Let \(X\) be a scheme over a ring \(A\) and \(\calL\) a semiample line bundle on \(X\).
        Then there exists a morphism \(f:X\to Y\) over \(A\) such that \(\calL\cong f^*\calO_Y(1)\) for some very ample line bundle \(\calO_Y(1)\) on \(Y\).
        Moreover, \(Y=\Proj R(X,\calL)\) and \(f\) is induced by the natural map \(R(X,\calL)\to \Gamma(X,\calL^{\otimes n})\).
        \Yang{To be continued.  }
        
    \end{theorem}

    \begin{definition}\label{def:big_line_bundle}
        A line bundle \(\calL\) on a scheme \(X\) is \emph{big} if the section ring \(R(X,\calL)\) has maximal growth, i.e., there exists \(C>0\) such that
        \[
            \dim \Gamma(X,\calL^{\otimes n}) \ge C n^{\dim X}
        \]
        for all sufficiently large \(n\).
        \Yang{To be continued.}
    \end{definition}

    \begin{example}\label{eg:base_locus_-K_of_Hirzebruch_surface}
        Let \(X=\bbF_2\) be the second Hirzebruch surface, i.e., the projective bundle \(\bbP(\calO_{\bbP^1}\oplus \calO_{\bbP^1}(2))\) over \(\bbP^1\).
        Let \(\pi:X\to \bbP^1\) be the projection and \(E\) the unique section of \(\pi\) with self-intersection \(-2\).
        \Yang{To be continued.}
    \end{example}

