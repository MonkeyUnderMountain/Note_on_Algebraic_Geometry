\section{Category of sheaves of modules}

\subsection{Sheaves of modules, quasi-coherent and coherent sheaves}

    % \begin{definition}\label{def:sheaf_of_modules}
    %     Let \((X,\calO_X)\) be a ringed space.
    %     A \emph{sheaf of \(\calO_X\)-modules} is a sheaf \(\calF\) of abelian groups on \(X\) such that for every open set \(U\subseteq X\), 
    %     \(\calF(U)\) is an \(\calO_X(U)\)-module, 
    %     and for every inclusion of open sets \(V\subseteq U\), the restriction map \(\res_{UV}:\calF(U)\to \calF(V)\) is \(\calO_X(U)\)-linear, 
    %     where the \(\calO_X(U)\)-module structure on \(\calF(V)\) is induced by the restriction map \(\res_{UV}:\calO_X(U)\to \calO_X(V)\).

    %     A \emph{morphism of \(\calO_X\)-modules} is a morphism of sheaves of abelian groups \(\varphi:\calF\to \calG\) such that 
    %     for every open set \(U\subseteq X\), the map \(\varphi(U):\calF(U)\to \calG(U)\) is \(\calO_X(U)\)-linear.
    % \end{definition}

    \begin{definition}\label{def:quasi-coherent_sheaf}
        Let \((X,\calO_X)\) be a ringed space.
        An \(\calO_X\)-module \(\calF\) is called \emph{quasi-coherent} if for every point \(x\in X\), there exists an open neighborhood \(U\) of \(x\) such that \(\calF|_U\) is isomorphic to the cokernel of a morphism of free \(\calO_U\)-modules, i.e., there exists an exact sequence of sheaves of \(\calO_U\)-modules
        \[
            \calO_U^{(I)} \to \calO_U^{(J)} \to \calF|_U \to 0,
        \]
        where \(I,J\) are (possibly infinite) index sets.
        % \Yang{To be checked...}
    \end{definition}

    \begin{definition}\label{def:finitely_generated_sheaf}
        Let \((X,\calO_X)\) be a ringed space.
        An \(\calO_X\)-module \(\calF\) is called \emph{finitely generated} if for every point \(x\in X\), there exists an open neighborhood \(U\) of \(x\) such that there exists a surjective morphism of sheaves of \(\calO_U\)-modules
        \[
            \calO_U^n \to \calF|_U \to 0.
        \]
    \end{definition}

    \begin{remark}\label{rmk:local_properties_of_sheaves}
        There are many versions of ``local'' properties for sheaves of \(\calO_X\)-modules.
        \Yang{To be continued.}
    \end{remark}

    \begin{definition}\label{def:coherent_sheaf}
        Let \((X,\calO_X)\) be a ringed space.
        An \(\calO_X\)-module \(\calF\) is called \emph{coherent} if it is finitely generated, and for every open set \(U\subseteq X\) and every morphism of sheaves of \(\calO_U\)-modules \(\varphi:\calO_U^n \to \calF|_U\), the kernel of \(\varphi\) is finitely generated.
        % \Yang{To be checked...}
    \end{definition}

\subsection{As abelian categories}

    \begin{theorem}\label{thm:sheaves_of_ab_mod_qcoh_coh_are_abelian_categories}
        The categories of sheaves of abelian groups, quasi-coherent sheaves, and coherent sheaves on a ringed space \((X,\calO_X)\) are all abelian categories.
        \Yang{To be checked.}
    \end{theorem}

    \begin{theorem}\label{thm:O_X_mod_has_enough_injectives}
        Let \((X,\calO_X)\) be a ringed space.
        The category of sheaves of \(\calO_X\)-modules has enough injectives.
        \Yang{To be checked.}
    \end{theorem}

    

\subsection{Relevant functors}

    \begin{definition}\label{def:hom_sheaves}
        Let \((X,\calO_X)\) be a ringed space, and let \(\calF,\calG\) be sheaves of \(\calO_X\)-modules.
        The \emph{sheaf Hom} \(\calHom_{\calO_X}(\calF,\calG)\) is the sheaf of abelian groups defined as follows:
        for an open set \(U\subseteq X\), we define
        \[
            \calHom_{\calO_X}(\calF,\calG)(U) := \Hom_{\calO_U}(\calF|_U,\calG|_U),
        \]
        where \(\Hom_{\calO_U}(\calF|_U,\calG|_U)\) is the set of morphisms of sheaves of \(\calO_U\)-modules from \(\calF|_U\) to \(\calG|_U\).
        For an inclusion of open sets \(V\subseteq U\), the restriction map
        \[
            \res_{UV}:\calHom_{\calO_X}(\calF,\calG)(U) \to \calHom_{\calO_X}(\calF,\calG)(V)
        \]
        is defined by sending a morphism \(\varphi:\calF|_U \to \calG|_U\) to its restriction \(\varphi|_V:\calF|_V \to \calG|_V\).
        \Yang{To be continued.}
    \end{definition}

    \begin{definition}\label{def:pull_back_of_sheaves}
        Let \(f:X\to Y\) be a morphism of ringed spaces.
        The \emph{pull-back functor} \(f^*:\Mod(\calO_Y)\to \Mod(\calO_X)\) is defined as follows:
        for an \(\calO_Y\)-module \(\calF\), we define
        \[
            f^*\calF := f^{-1}\calF \otimes_{f^{-1}\calO_Y} \calO_X,
        \]
        where \(f^{-1}\calF\) is the inverse image sheaf of \(\calF\).
        For a morphism of \(\calO_Y\)-modules \(\varphi:\calF\to \calG\), we define
        \[
            f^*\varphi: f^*\calF \to f^*\calG
        \]
        to be the morphism induced by the morphism of sheaves of abelian groups \(f^{-1}\varphi:f^{-1}\calF \to f^{-1}\calG\).
        \Yang{To be continued.}
    \end{definition}

    \begin{definition}\label{def:tensor_product_of_sheaves}
        Let \((X,\calO_X)\) be a ringed space, and let \(\calF,\calG\) be sheaves of \(\calO_X\)-modules.
        The \emph{tensor product} \(\calF \otimes_{\calO_X} \calG\) is the sheaf of \(\calO_X\)-modules defined as follows:
        for an open set \(U\subseteq X\), we define
        \[
            (\calF \otimes_{\calO_X} \calG)(U) := \calF(U) \otimes_{\calO_X(U)} \calG(U),
        \]
        where \(\calF(U) \otimes_{\calO_X(U)} \calG(U)\) is the tensor product of \(\calO_X(U)\)-modules.
        For an inclusion of open sets \(V\subseteq U\), the restriction map

        \Yang{To be continued.}
    \end{definition}

    

\subsection{Locally free sheaves and vector bundles}

    

\subsection{Cohomological theory}

    