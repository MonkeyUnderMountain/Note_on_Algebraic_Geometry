\section{Structure of linear algebraic groups I: commutative and solvable groups}

\Yang{In this section, everything is defined on an algebraically closed field \(\kkk\).}

\subsection{Commutative algebraic groups and character groups}

    \begin{definition}\label{def:character_group_of_linear_algebraic_group}
        Let \(G\) be a linear algebraic group over a field \(\kkk\). 
        The \emph{character group} of \(G\), denoted by \(\chi(G)\), is defined to be the group of all homomorphisms of linear algebraic groups from \(G\) to the multiplicative group \(\bbG_m\):
        \[
            \chi(G) = \Hom_{\AlgGrp_\kkk}(G, \bbG_m).
        \]
        The group operation on \(\chi(G)\) is given by pointwise multiplication of characters.
    \end{definition}

    \begin{example}\label{eg:character_group_of_G_m}
        Let us compute the character group of the multiplicative group \(\bbG_m\).
        Let \(\varphi: \bbG_m \to \bbG_m\) be a character of \(\bbG_m\). 
        Since \(\varphi\) is a morphism of algebraic varieties, it induces a homomorphism of coordinate rings \(\varphi^\sharp: \kkk[T, T^{-1}] \to \kkk[T, T^{-1}]\).
        Note that \((\kkk[T, T^{-1}])^\times = \{ c T^n \mid c \in \kkk^\times, n \in \bbZ \}\). 
        Thus, we have \(\varphi^\sharp(T) = c T^n\) for some \(c \in \kkk^\times\) and \(n \in \bbZ\).
        That is, \(\varphi(a) = c a^n\) for all \(a \in \bbG_m(\kkk) = \kkk^\times\). 
        However, since \(\varphi\) is a group homomorphism, we must have \(\varphi(1) = c = 1\). 
        Therefore, \(\varphi(a) = a^n\) for some integer \(n\). 
        Conversely, for each integer \(n\), the map \(\chi_n: \bbG_m \to \bbG_m\) defined by \(\chi_n(a) = a^n\) is indeed a character of \(\bbG_m\). 
        Hence, we have shown that \(\chi(\bbG_m) \cong \bbZ\).
    \end{example}

    \begin{example}\label{eg:character_group_of_G_a}
        The character group of the additive group \(\bbG_a\) is trivial, i.e., \(\chi(\bbG_a) = \{0\}\).
        Indeed, every morphism from \(\bbA^1\) to \(\bbA^1 \setminus \{0\}\) is constant since every regular function on \(\bbA^1\) is a polynomial, and there is no non-constant polynomial function without zeros.
    \end{example}

    \begin{definition}\label{def:torus_linear_algebraic_group}
        A linear algebraic group \(T\) over a field \(\kkk\) is called a \emph{torus} if \(T\) is isomorphic to a finite product of copies of the multiplicative group \(\bbG_m\), i.e., \(T \cong \bbG_m^n\) for some non-negative integer \(n\).
    \end{definition}

    \begin{lemma}\label{prop:subgroup_of_torus_is_torus}
        Let \(T\) be a torus and \(H\) be a connected closed algebraic subgroup of \(T\). 
        Then \(H\) is also a torus.
    \end{lemma}
    \begin{proof}
        \Yang{To be continued.}
    \end{proof}

    Recall that for a linear operator \(T:V \to V\) of finite-dimensional \(\kkk\)-vector space \(V\) is called \emph{semisimple} if it is diagonalizable, and \emph{unipotent} if \(T-\id_V\) is nilpotent.

    \begin{proposition}\label{prop:semisimple_element_and_torus}
        Let \(T \subset \GL_n(\kkk)\) be a torus. 
        Then every element of \(T(\kkk)\) is semisimple.
        Conversely, if \(g \in \GL_n(\kkk)\) is semisimple and of infinite order, then the neutral component of the algebraic subgroup generated by \(g\) is a torus.
    \end{proposition}

    \begin{proposition}\label{prop:unipotent_element_and_G_a}
        Let \(g \subset \GL_n(\kkk)\).
        Then \(g\) is unipotent if and only if the algebraic subgroup generated by \(g\) is isomorphic to \(\bbG_a\).
        \Yang{To be revised.}
    \end{proposition}

    \begin{theorem}\label{prop:structure_of_connected_commutative_linear_algebraic_groups}
        Let \(G\) be a connected commutative linear algebraic group over an algebraically closed field \(\kkk\) of characteristic zero. 
        Then \(G\) is isomorphic to \(\bbG_m^r \times \bbG_a^s\) for some non-negative integers \(r\) and \(s\).
    \end{theorem}
    \begin{proof}
        \Yang{To be continued.}
    \end{proof}


\subsection{Jordan-Chevalley Decomposition of elements}

    
    \begin{definition}\label{def:semisimple_and_unipotent_element}
        Let \(G\) be a linear algebraic group and \(g \in G(\kkk)\). 
        We say that \(g\) is \emph{semisimple} (resp. \emph{unipotent}) if its image under some (equivalently, any) faithful linear representation of \(G\) is a semisimple (resp. unipotent) linear operator.
    \end{definition}

    % \begin{proposition}\label{prop:semisimple_and_unipotent_elements_in_G_m_and_G_a}
    %     Let \(g \in G\), \(g\) is semisimple if and only if it is contained in a torus of \(G\).

    %     Similarly, \(g\) is unipotent if and only if it is contained in a subgroup of \(G\) isomorphic to the additive group \(\bbG_a\).
    %     \Yang{To be revised.}
    % \end{proposition}

    \begin{lemma}\label{lem:semisimple_and_unipotent_elements_do_not_depend_on_faithful_representation}
        The notion of semisimple and unipotent elements in \cref{def:semisimple_and_unipotent_element} does not depend on the choice of faithful linear representation.
    \end{lemma}
    \begin{proof}
        \Yang{To be added.}
    \end{proof}

    \begin{theorem}[Jordan-Chevalley Decomposition]\label{thm:jordan_chevalley_decomposition}
        Let \(G\) be a linear algebraic group and \(g \in G(\kkk)\). 
        Then there exist unique commuting elements \(g_s, g_u \in G(\kkk)\) such that \(g = g_s g_u\), where \(g_s\) is semisimple and \(g_u\) is unipotent.

        Moreover, this decomposition is functorial in the sense that for any homomorphism of linear algebraic groups \(\varphi: G \to H\), we have \(\varphi(g)_s = \varphi(g_s)\) and \(\varphi(g)_u = \varphi(g_u)\).
        \Yang{To be checked}
    \end{theorem}
    \begin{proof}
        \Yang{To be continued.}
    \end{proof}



\subsection{Solvable groups and Borel subgroups}

    \begin{definition}\label{def:solvable_group}
        A group \(G\) is said to be \emph{solvable} if there exists a finite sequence of algebraic subgroups
        \[
            G = G_0 \vartriangleright G_1 \vartriangleright G_2 \vartriangleright \cdots \vartriangleright G_n = \{e\}
        \]
        such that each \(G_{i+1}\) is normal in \(G_i\) and the quotient group \(G_i/G_{i+1}\) is commutative for all \(0 \leq i < n\).
        \Yang{to be checked.}
    \end{definition}

    \begin{theorem}\label{prop:fixed_point_of_solvable_group_acting_on_proper_varieties}
        Let \(G\) be a solvable linear algebraic group acting on a proper variety \(X\). 
        Then there exists a fixed point \(x \in X(\kkk)\) such that \(g \cdot x = x\) for all \(g \in G(\kkk)\).
    \end{theorem}

    \begin{corollary}[Lie-Kolchin Theorem]\label{prop:Lie-Kolchin_theorem}
        Let \(G < \GL_n(\kkk)\) be a solvable linear algebraic group over an algebraically closed field \(\kkk\). 
        Then there exists a basis of \(\kkk^n\) such that \(G\) is contained in the group of upper triangular matrices with respect to this basis.
    \end{corollary}

    \begin{theorem}\label{thm:classification_of_linear_alg_gp_of_dim1}
        Let \(G\) be a linear algebraic group of dimension \(1\) over an algebraically closed field \(\kkk\). 
        Then \(G\) is isomorphic to either \(\bbG_m\) or \(\bbG_a\).
    \end{theorem}



\subsection{Decomposition of linear algebraic groups}

    \begin{definition}\label{def:radical_of_linear_algebraic_group}
        Let \(G\) be a linear algebraic group over a field \(\kkk\). 
        The \emph{radical} of \(G\), denoted by \(\rad(G)\), is defined to be the unique maximal connected normal solvable subgroup of \(G\).
    \end{definition}
    \Yang{Well-defined?}

    \begin{definition}\label{def:nilradical_of_linear_algebraic_group}
        Let \(G\) be a linear algebraic group. 
        The \emph{unipotent radical} of \(G\), denoted by \(\rad_u(G)\), is defined to be the subgroup of \(\rad(G)\) consisting of all unipotent elements.
    \end{definition}
    \Yang{Why a group?}

    \begin{definition}\label{def:semisimple_linear_algebraic_group}
        Let \(G\) be a linear algebraic group over a field \(\kkk\). 
        We say that \(G\) is \emph{semisimple} if \(\rad(G)\) is trivial.
    \end{definition}

    \begin{definition}\label{def:reductive_linear_algebraic_group}
        Let \(G\) be a linear algebraic group over a field \(\kkk\). 
        We say that \(G\) is \emph{reductive} if the unipotent radical of \(G\) is trivial.
    \end{definition}

    \begin{slogan}\label{slogan:structure_of_linear_algebraic_groups}
        \[ 
            \begin{tikzcd}[row sep=small,column sep=small]
                \text{``unipotent radical''} \arrow[r,phantom,"\rightarrow\!\leftarrow"] \arrow[d,Rightarrow]   & \text{``reductive''}  \\
                \text{``solvable radical''} \arrow[r,phantom,"\rightarrow\!\leftarrow"]                         & \text{``semisimple''} \arrow[u,Rightarrow]
            \end{tikzcd}
        \]
    \end{slogan}

    \begin{theorem}[Levi Decomposition]\label{thm:levi_decomposition}
        Let \(G\) be a linear algebraic group over an algebraically closed field \(\kkk\). 
        Then there exists a reductive subgroup \(H\) of \(G\) such that the multiplication map \(\rad_u(G) \rtimes H \to G\) is an isomorphism of algebraic groups.
        Such a subgroup \(H\) is called a \emph{Levi subgroup} of \(G\).
        \Yang{To be checked.}
    \end{theorem}
    \begin{proof}
        \Yang{To be continued.}
    \end{proof}



% \subsection{Semisimple and reductive algebraic groups}


