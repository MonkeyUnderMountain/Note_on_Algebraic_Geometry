\section{Structure of linear algebraic groups}


\subsection{Jordan-Chevalley Decomposition of elements}

    Recall that for a linear operator \(T:V \to V\) of finite-dimensional \(\kkk\)-vector space \(V\) is called \emph{semisimple} if it is diagonalizable, and \emph{unipotent} if \(T-\id_V\) is nilpotent.

    \begin{definition}\label{def:semisimple_and_unipotent_element}
        Let \(G\) be a linear algebraic group and \(g \in G(\kkk)\). 
        We say that \(g\) is \emph{semisimple} (resp. \emph{unipotent}) if its image under some (equivalently, any) faithful linear representation of \(G\) is a semisimple (resp. unipotent) linear operator.
    \end{definition}

    \begin{lemma}\label{lem:semisimple_and_unipotent_elements_do_not_depend_on_faithful_representation}
        The notion of semisimple and unipotent elements in \cref{def:semisimple_and_unipotent_element} does not depend on the choice of faithful linear representation.
    \end{lemma}
    \begin{proof}
        \Yang{To be added.}
    \end{proof}

    \begin{theorem}[Jordan-Chevalley Decomposition]\label{thm:jordan_chevalley_decomposition}
        Let \(G\) be a linear algebraic group and \(g \in G(\kkk)\). 
        Then there exist unique commuting elements \(g_s, g_u \in G(\kkk)\) such that \(g = g_s g_u\), where \(g_s\) is semisimple and \(g_u\) is unipotent.

        Moreover, this decomposition is functorial in the sense that for any homomorphism of linear algebraic groups \(\varphi: G \to H\), we have \(\varphi(g)_s = \varphi(g_s)\) and \(\varphi(g)_u = \varphi(g_u)\).
        \Yang{To be checked}
    \end{theorem}
    \begin{proof}
        \Yang{To be continued.}
    \end{proof}



\subsection{Solvable groups and Borel subgroups}

    \begin{definition}\label{def:solvable_group}
        A group \(G\) is said to be \emph{solvable} if there exists a finite sequence of algebraic subgroups
        \[
            G = G_0 \vartriangleright G_1 \vartriangleright G_2 \vartriangleright \cdots \vartriangleright G_n = \{e\}
        \]
        such that each \(G_{i+1}\) is normal in \(G_i\) and the quotient group \(G_i/G_{i+1}\) is commutative for all \(0 \leq i < n\).
        \Yang{to be checked.}
    \end{definition}

    \begin{theorem}\label{prop:fixed_point_of_solvable_group_acting_on_proper_varieties}
        Let \(G\) be a solvable linear algebraic group acting on a proper variety \(X\). 
        Then there exists a fixed point \(x \in X(\kkk)\) such that \(g \cdot x = x\) for all \(g \in G(\kkk)\).
    \end{theorem}

    \begin{corollary}[Lie-Kolchin Theorem]\label{prop:Lie-Kolchin_theorem}
        Let \(G < \GL_n(\kkk)\) be a solvable linear algebraic group over an algebraically closed field \(\kkk\). 
        Then there exists a basis of \(\kkk^n\) such that \(G\) is contained in the group of upper triangular matrices with respect to this basis.
    \end{corollary}

    \begin{theorem}\label{thm:classification_of_linear_alg_gp_of_dim1}
        Let \(G\) be a linear algebraic group of dimension \(1\) over an algebraically closed field \(\kkk\). 
        Then \(G\) is isomorphic to either \(\bbG_m\) or \(\bbG_a\).
    \end{theorem}



\subsection{Decomposition of linear algebraic groups}

    \begin{definition}\label{def:radical_of_linear_algebraic_group}
        Let \(G\) be a linear algebraic group over a field \(\kkk\). 
        The \emph{radical} of \(G\), denoted by \(\rad(G)\), is defined to be the unique maximal connected normal solvable subgroup of \(G\).
    \end{definition}
    \Yang{Well-defined?}

    \begin{definition}\label{def:nilradical_of_linear_algebraic_group}
        Let \(G\) be a linear algebraic group. 
        The \emph{unipotent radical} of \(G\), denoted by \(\rad_u(G)\), is defined to be the subgroup of \(\rad(G)\) consisting of all unipotent elements.
    \end{definition}
    \Yang{Why a group?}

    \begin{definition}\label{def:semisimple_linear_algebraic_group}
        Let \(G\) be a linear algebraic group over a field \(\kkk\). 
        We say that \(G\) is \emph{semisimple} if \(\rad(G)\) is trivial.
    \end{definition}

    \begin{definition}\label{def:reductive_linear_algebraic_group}
        Let \(G\) be a linear algebraic group over a field \(\kkk\). 
        We say that \(G\) is \emph{reductive} if the unipotent radical of \(G\) is trivial.
    \end{definition}

    \begin{slogan}\label{slogan:structure_of_linear_algebraic_groups}
        \[ 
            \begin{tikzcd}[row sep=small,column sep=small]
                \text{``unipotent radical''} \arrow[r,phantom,"\rightarrow\!\leftarrow"] \arrow[d,Rightarrow]   & \text{``reductive''}  \\
                \text{``solvable radical''} \arrow[r,phantom,"\rightarrow\!\leftarrow"]                         & \text{``semisimple''} \arrow[u,Rightarrow]
            \end{tikzcd}
        \]
    \end{slogan}

    \begin{theorem}[Levi Decomposition]\label{thm:levi_decomposition}
        Let \(G\) be a linear algebraic group over an algebraically closed field \(\kkk\). 
        Then there exists a reductive subgroup \(H\) of \(G\) such that the multiplication map \(\rad_u(G) \rtimes H \to G\) is an isomorphism of algebraic groups.
        Such a subgroup \(H\) is called a \emph{Levi subgroup} of \(G\).
        \Yang{To be checked.}
    \end{theorem}
    \begin{proof}
        \Yang{To be continued.}
    \end{proof}



\subsection{Semisimple and reductive algebraic groups}


