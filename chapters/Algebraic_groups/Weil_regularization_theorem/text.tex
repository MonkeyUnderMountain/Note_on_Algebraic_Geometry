\section{Weil regularization theorem}


    % \begin{definition}\label{def:rational_group_actions}
    %     An \emph{rational action} of an algebraic group \(G\) on a variety \(X\) is a rational map
    %     \[
    %         \sigma: G \times X \dashrightarrow X
    %     \]
    %     such that the following diagrams commute wherever the maps are defined:
    %     \[
    %         \begin{tikzcd}
    %             G \times G \times X \arrow{r}{\mu \times \id_X} \arrow[dashed]{d}{\id_G \times \sigma} & G \times X \arrow[dashed]{d}{\sigma} \\
    %             G \times X \arrow[dashed]{r}{\sigma} & X     
    %         \end{tikzcd}
    %         \quad
    %         \begin{tikzcd}
    %             \Spec \kk \times X \arrow{r}{\varepsilon \times \id_X} \arrow{dr}[swap]{\cong} & G \times X \arrow[dashed]{d}{\sigma} \\
    %             & X
    %         \end{tikzcd}
    %     \]
    %     where \(\mu\) is the group law of \(G\) and \(\varepsilon\) is the identity element of \(G\).
    %     \Yang{To be checked.}
    % \end{definition}


    % \begin{theorem}[Weil's Regularization Theorem]\label{thm:Weil_regularization}
    %     Let \(G\) be an algebraic group acting on a variety \(X\).
    %     Then there exists a variety \(Y\) with a regular action of \(G\) and a \(G\)-equivariant birational isomorphism \(X \dashrightarrow Y\).
    %     % In other words, any rational action of an algebraic group on a variety can be regularized.
    %     \Yang{To be checked.}
    % \end{theorem}