\section{First properties of algebraic groups}

Let \(\kk\) be a field and \(\kkk\) its algebraic closure. 
All varieties are defined over \(\kk\) unless otherwise specified.

\subsection{Basic concepts}

    \begin{definition}\label{def:group_scheme}
        A \emph{group scheme} over \(S\) is an \(S\)-scheme \(G\) together with morphisms \emph{multiplication} \(\mu: G \times G \to G\), \emph{identity} \(\varepsilon: S \to G\) and \emph{inversion} \(\iota: G \to G\) over \(S\) such that the following diagrams commute:
        \begin{enumerate}
            \item (Associativity)
            \[
                \begin{tikzcd}
                    & G \times G \times G \arrow[rd, "\mu \times \id_G"] \arrow[ld, "\id_G \times \mu"'] & \\
                    G \times G \arrow[rd, "\mu"] & & G \times G \arrow[ld, "\mu"'] \\
                     & G & 
                \end{tikzcd};
            \]
            \item (Identity)
            \[
                \begin{tikzcd}
                    G \times S \arrow[r, "\id_G \times \varepsilon"] \arrow[rd, "\cong"'] & G \times G \arrow[d, "\mu"] & S \times G \arrow[l, "\varepsilon \times \id_G"'] \arrow[ld, "\cong"] \\
                    & G &
                \end{tikzcd};
            \]
            \item (Inversion)
            \[
                \begin{tikzcd}
                    & G \arrow[ld, "\id_G \times i"'] \arrow[rd, "i \times \id_G"] \arrow[d] & \\
                    G \times G \arrow[rd, "\mu"'] & S \arrow[d, "\varepsilon"] & G\times G\arrow[ld, "\mu"]\\
                    & G &
                \end{tikzcd}.
            \]
        \end{enumerate}
        In other words, a group scheme is a group object in the category of schemes.
        % \Yang{To be continued...}
    \end{definition}

    \begin{definition}\label{def:algebraic_group}
        An \emph{algebraic group} is a \(\kk\)-group scheme \(G\) which is reduced, separated and of finite type over a field \(\kk\).
    \end{definition}

    \begin{definition}\label{def:transition_morphism}
        Let \(G\) be an algebraic group and \(x \in G(\kk)\) a \(\kk\)-point.
        The \emph{left translation} by \(x\) is the morphism
        \[
            l_x: G \xrightarrow{\cong} \Spec \kk \times G \xrightarrow{x \times \id_G} G \times G \xrightarrow{\mu} G.
        \]
        Similar definition applies to the right translation \(r_x\).
    \end{definition}

    \begin{remark}\label{rmk:notation_of_multiplication_and_inverse}
        In the context of algebraic groups, we often use multiplicative notation for the group law.
        That is, for \(g,h \in G(\kk)\), we write \(gh\) instead of \(\mu(g,h)\) and \(g^{-1}\) instead of \(\iota(g)\).
        The identity element \(\varepsilon\) is often denoted by \(e\).
        
        Sometimes we also abuse the notation by \(\mu:G \times \cdots \times G \to G\) to denote the multiplication of multiple elements, i.e. \(\mu(g_1,\ldots,g_n) = g_1 \cdots g_n\) for \(g_1,\ldots,g_n \in G(\kk)\).
    \end{remark}

    \begin{remark}\label{rmk:algebraic_group_and_its_closed_points}
        Since algebraic groups are almost varieties over an arbitrary field \(\kk\), we often identify an algebraic group \(G\) with its set of closed points \(G(\kkk)\) when there is no confusion.
    \end{remark}

    \begin{proposition}\label{prop:algebraic_group_is_smooth}
        Let \(G\) be an algebraic group.
        Then \(G\) is smooth over \(\kk\).
    \end{proposition}
    \begin{proof}
        Since \(G\) is reduced and of finite type over a field, it is generically regular.
        Let \(g \in G(\kkk)\) be a regular point.
        Then the left translation \(l_{gh^{-1}}: G \to G\) is an isomorphism, hence \(G\) is regular at \(h \in G(\kkk)\).
        It follows that \(G\) is regular at every \(\kkk\)-point, hence \(G\) is smooth over \(\kk\).
    \end{proof}

    \begin{remark}\label{rmk:irreducible_and_connected_components_of_algebraic_group}
        Let \(G\) be an algebraic group.
        Then the irreducible components of \(G\) coincide with the connected components of \(G\).
        We will use the term ``connected'' to refer to both concepts since ``irreducible'' has other meanings in the theory of representations.
    \end{remark}

    \begin{example}\label{eg:additive_group}
        The \emph{additive group} $\bbG_a$ is defined to be the affine line $\mathbb{A}^1$ with the group law given by addition.
        Concretely, we can write $\bbG_a = \Spec \kk[T]$ with the group law given by the morphism
        \begin{align*}
            \mu:\bbG_a \times \bbG_a \to \bbG_a, \quad & (x,y) \mapsto x+y, \\
            \iota:\bbG_a \to \bbG_a, \quad & x \mapsto -x, \\
            \varepsilon:\Spec \kk \to \bbG_a, \quad & * \mapsto 0.
        \end{align*}
    \end{example}

    \begin{example}\label{eg:multiplicative_group}
        The \emph{multiplicative group} $\bbG_m$ is defined to be the affine variety $\mathbb{A}^1 \setminus \{0\}$ with the group law given by multiplication.
        Concretely, we can write $\bbG_m = \Spec \kk[T,T^{-1}]$ with the group law given by the morphism
        \begin{align*}
            \mu: \bbG_m \times \bbG_m \to \bbG_m, \quad & (x,y) \mapsto xy, \\
            \iota: \bbG_m \to \bbG_m, \quad & x \mapsto x^{-1}, \\
            \varepsilon: \Spec \kk \to \bbG_m, \quad & * \mapsto 1.
        \end{align*}
    \end{example}

    \begin{example}\label{eg:general_linear_group}
        The \emph{general linear group} $\GL_n$ is defined to be the open subvariety of $\mathbb{A}^{n^2}$ consisting of invertible matrices, with the group law given by matrix multiplication.
        Concretely, we can write \(\GL_n = \Spec \kk[T_{ij}, \det(T_{ij})^{-1}]\) where \(1 \leq i,j \leq n\) and the group law is given by the morphism
        \begin{align*}
            \mu: & \GL_n \times \GL_n \to \GL_n, \quad (A,B) \mapsto AB, \\
            \iota: & \GL_n \to \GL_n, \quad A \mapsto A^{-1}, \\
            \varepsilon: & \Spec \kk \to \GL_n, \quad * \mapsto I_n.
        \end{align*}
    \end{example}

    \begin{example}\label{eg:abelian_varieties_as_algebraic_groups}
        An abelian variety is an algebraic group that is also a proper variety.
    \end{example}

    \begin{example}\label{eg:product_of_algebraic_groups}
        Let \(G\) and \(H\) be algebraic groups.
        The \emph{product} \(G \times H\) is an algebraic group with the group law defined by
        \begin{align*}
            \mu_{G \times H} = \mu_G \times \mu_H &: (G \times H) \times (G \times H) \cong (G \times G) \times (H \times H) \to G \times H, \\   
            \varepsilon_{G \times H} = \varepsilon_G \times \varepsilon_H &: \Spec \kk \cong \Spec \kk \times \Spec \kk \to G \times H, \\
            \iota_{G \times H} = \iota_G \times \iota_H &: G \times H \to G \times H.
        \end{align*}
    \end{example}

    \begin{example}\label{eg:base_change_of_algebraic_groups}
        Let \(G\) be an algebraic group over \(\kk\) and \(\KK/\kk\) a field extension.
        The base change \(G_\KK = G \times_{\Spec \kk} \Spec \KK\) is an algebraic group over \(\KK\) with the group law defined by the base change of the original group law of \(G\) to \(\KK\).
    \end{example}

    \begin{definition}\label{def:homomorphism_of_algebraic_groups}
        A \emph{homomorphism} of algebraic groups is a morphism of schemes that is also a group homomorphism.
        Explicitly, a morphism \(f: G \to H\) between algebraic groups \(G\) and \(H\) is a homomorphism if the following diagram commutes:
        \[
            \begin{tikzcd}
                G \times G \arrow{r}{\mu_G} \arrow[d, "f\times f" swap] & G \arrow{d}{f} \\
                H \times H \arrow{r}{\mu_H} & H
            \end{tikzcd}
        \]
        where \(\mu_G\) and \(\mu_H\) are the group laws of \(G\) and \(H\), respectively.
    \end{definition}

    \begin{definition}\label{def:algebraic_subgroup}
        An \emph{algebraic subgroup} of an algebraic group \(G\) is a closed subscheme \(H \subseteq G\) that is also a subgroup of \(G\).
        More precisely, \(H\) is an algebraic subgroup and the inclusion morphism \(H \hookrightarrow G\) is compatible with the group laws.
    \end{definition}

    \Yang{I need the definition of normal subgroup here.}

    \begin{example}\label{eg:special_linear_group}
        The \emph{special linear group} $\SL_n$ is defined to be the closed subvariety of $\GL_n$ defined by the equation \(\det = 1\).
        It is an algebraic subgroup of \(\GL_n\).
    \end{example}

    \begin{definition}\label{def:neutral_component}
        Let \(G\) be an algebraic group.
        The \emph{neutral component} \(G^0\) is the connected component of \(G\) containing the identity element \(\varepsilon\).
        % It is a closed, normal algebraic subgroup of \(G\).
    \end{definition}

    \begin{proposition}\label{prop:neutral_component_is_subgroup}
        The neutral component \(G^0(\kkk)\) is a closed, normal algebraic subgroup of \(G(\kkk)\) of finite index.
        Moreover, each closed subgroup \(H\) of finite index contains \(G^0(\kkk)\).
    \end{proposition}
    \begin{proof}
        \Yang{To be continued...}
    \end{proof}

    \begin{proposition}\label{prop:closure_of_subgroup_is_subgroup}
        Let \(G\) be an algebraic group and \(H \subseteq G(\kkk)\) a subgroup (not necessarily closed).
        Then the Zariski closure \(\overline{H}\) of \(H\) in \(G\) is an algebraic subgroup of \(G\).
        If \(H \subset G(\kkk)\) is constructible, then \(H = \overline{H}(\kkk)\).
    \end{proposition}
    \begin{proof}
        \Yang{To be continued...}
    \end{proof}

    % \begin{remark}\label{rmk:closure_of_subgroup_is_subgroup}
    %     The assumption that \(H\) is a subgroup is essential.
    %     \Yang{To be continued...}
    % \end{remark}

    \begin{example}\label{eg:product_of_closed_algebraic_subgroups_which_is_not_closed}
        Let \(G = \SL_2\) over \(\kkk\), \(T = \{\diag(t,t^{-1})|t \in \kkk^\times\}\) and \(g = \begin{pmatrix}
            1 & 1 \\
            0 & 1
        \end{pmatrix}\).
        Set \(S = gTg^{-1}\).
        Then both \(T\) and \(S\) are closed algebraic subgroups of \(G(\kkk)\), but the product \(TS\) is not closed in \(G(\kkk)\).
        By direct computation, we have
        \[ S = \left\{ \begin{pmatrix}
            s & s^{-1}-s \\
            0 & s^{-1}
        \end{pmatrix} \;\middle|\; s \in \kkk^\times \right\}.
        \]
        Then 
        \[ TS = \left\{ \begin{pmatrix}
            ts & t(s^{-1}-s) \\
            0 & (ts)^{-1}
        \end{pmatrix} \;\middle|\; t,s \in \kkk^\times \right\}.
        \]
        We have 
        \[ TS \cap \left\{ \begin{pmatrix}
            1 & a \\
            0 & 1
        \end{pmatrix} \;\middle|\; a \in \kkk \right\} = \left\{ \begin{pmatrix}
            1 & s^{-2}-1 \\
            0 & 1
        \end{pmatrix} \;\middle|\; s \in \kkk^\times \right\}. \]
        The right hand side is not closed in \(\SL_2(\kkk)\) since it does not contain the matrix \(\begin{pmatrix}
            1 & -1 \\
            0 & 1
        \end{pmatrix}\).
        Hence \(TS\) is not closed in \(G(\kkk)\).
    \end{example}

    \begin{proposition}\label{prop:generate_subgroup_by_irred_constructible_subset}
        Let \(G\) be an algebraic group, \(X_i\) varieties over \(\kk\) and \(f_i: X_i \to G\) morphisms for \(i=1,\ldots,n\) with images \(Y_i = f_i(X_i)\).
        Suppose that \(Y_i\) pass through the identity element of \(G\).
        Let \(H\) be the closed subgroup of \(G\) generated by \(Y_1,\ldots,Y_n\), i.e. the smallest closed subgroup of \(G\) containing \(Y_1,\ldots,Y_n\).
        Then \(H\) is connected and \(H = Y_{a_1}^{e_1} \cdots Y_{a_m}^{e_m}\) for some \(a_1,\ldots,a_m \in \{1,\ldots,n\}\) and \(e_1,\ldots,e_m \in \{\pm 1\}\).
    \end{proposition}
    \begin{proof}
        \Yang{To be continued...}
    \end{proof}

    \begin{remark}\label{rmk:generated_subgroup_by_irred_constructible_subset}
        We can take \(m \leq 2 \dim G\) in \cref{prop:generate_subgroup_by_irred_constructible_subset}.
        % \Yang{To be continued...}
    \end{remark}




\subsection{Action and representations}

    \begin{definition}\label{def:action_of_algebraic_group}
        An \emph{action} of an algebraic group \(G\) on a variety \(X\) is a morphism
        \[
            \sigma: G \times X \to X
        \]
        such that the following diagrams commute:
        \[
            \begin{tikzcd}
                G \times G \times X \arrow{r}{\mu \times \id_X} \arrow{d}{\id_G \times \sigma} & G \times X \arrow{d}{\sigma} \\
                G \times X \arrow{r}{\sigma} & X
            \end{tikzcd}
            \quad
            \begin{tikzcd}
                \Spec \kk \times X \arrow{r}{\varepsilon \times \id_X} \arrow{dr}[swap]{\cong} & G \times X \arrow{d}{\sigma} \\
                & X
            \end{tikzcd}
        \]
        where \(\mu\) is the group law of \(G\) and \(\varepsilon\) is the identity element of \(G\).
        In other words, for any \(\kk\)-scheme \(S\), the induced map \(G(S) \times X(S) \to X(S)\) defines a group action of the abstract group \(G(S)\) on the set \(X(S)\).
        % We say that \(X\) is a \emph{\(G\)-variety}.
        % \Yang{To be checked.}
    \end{definition}

    % \begin{definition}\label{def:basic_concepts_in_group_action}
    %     Let \(G\) be an algebraic group acting on a variety \(X\).
    %     For any field extension \(K/\kk\) and \(x \in X(K)\), the \emph{orbit} of \(x\) is the subset
    %     \[
    %         G(K) \cdot x = \{\sigma(g,x) | g \in G(K)\} \subseteq X(K).
    %     \]
    %     The \emph{stabilizer} of \(x\) is the subgroup
    %     \[
    %         G_x(K) = \{g \in G(K) | \sigma(g,x) = x\} \subseteq G(K).
    %     \]
    %     The action is called \emph{transitive} if for any field extension \(K/\kk\), the induced map \(G(K) \times X(K) \to X(K)\) is transitive.
    %     The action is called \emph{faithful} if for any field extension \(K/\kk\), the induced map \(G(K) \times X(K) \to X(K)\) is faithful.
    %     \Yang{To be checked.}
    % \end{definition}

    % \begin{example}\label{eg:linear_representation}
    %     A \emph{linear representation} of an algebraic group \(G\) on a finite-dimensional vector space \(V = \bbA_\kk^k\) over \(\kk\) is an action of \(G\) on the affine space associated to \(V\), i.e. a morphism
    %     \[
    %         \rho: G \times V \to V
    %     \]
    %     such that for any field extension \(K/\kk\), the induced map \(G(K) \times V(K) \to V(K)\) defines a group homomorphism from the abstract group \(G(K)\) to the general linear group of the vector space \(V(K)\).
    %     In other words, for any \(g \in G(K)\), the map \(\rho_g: V(K) \to V(K)\) defined by \(\rho_g(v) = \rho(g,v)\) is a linear automorphism of \(V(K)\).
    %     We say that \(V\) is a \(G\)-module.
    %     \Yang{To be checked.}
    % \end{example}

    \begin{definition}\label{def:rational_group_actions}
        An \emph{rational action} of an algebraic group \(G\) on a variety \(X\) is a rational map
        \[
            \sigma: G \times X \dashrightarrow X
        \]
        such that the following diagrams commute wherever the maps are defined:
        \[
            \begin{tikzcd}
                G \times G \times X \arrow{r}{\mu \times \id_X} \arrow[dashed]{d}{\id_G \times \sigma} & G \times X \arrow[dashed]{d}{\sigma} \\
                G \times X \arrow[dashed]{r}{\sigma} & X     
            \end{tikzcd}
            \quad
            \begin{tikzcd}
                \Spec \kk \times X \arrow{r}{\varepsilon \times \id_X} \arrow{dr}[swap]{\cong} & G \times X \arrow[dashed]{d}{\sigma} \\
                & X
            \end{tikzcd}
        \]
        where \(\mu\) is the group law of \(G\) and \(\varepsilon\) is the identity element of \(G\).
        \Yang{To be checked.}
    \end{definition}

    \begin{definition}\label{def:orbit_of_algebraic_group_action}
        Let \(G\) be an algebraic group acting on a variety \(X\).
        For any \(x \in X(\kk)\), the \emph{orbit} of \(x\) is the locally closed subvariety \(G \cdot x = \sigma(G \times \{x\})\) of \(X\).
        % Note that this definition is different from the one in \cref{def:basic_concepts_in_group_action}.
        \Yang{To be checked.}
    \end{definition}

    % \begin{theorem}[Weil's Regularization Theorem]\label{thm:Weil_regularization}
    %     Let \(G\) be an algebraic group acting on a variety \(X\).
    %     Then there exists a variety \(Y\) with a regular action of \(G\) and a \(G\)-equivariant birational isomorphism \(X \dashrightarrow Y\).
    %     % In other words, any rational action of an algebraic group on a variety can be regularized.
    %     \Yang{To be checked.}
    % \end{theorem}

    \begin{proposition}\label{prop:orbit_is_locally_closed}
        Let \(G\) be an algebraic group acting on a variety \(X\).
        Then for any \(x \in X(\kk)\), the orbit \(G \cdot x\) is a locally closed subvariety of \(X\), and \(\overline{G \cdot x} \setminus G \cdot x \) is a union of orbits of strictly smaller dimension.
    \end{proposition}
    \begin{proof}
        \Yang{To be continued...}
    \end{proof}

% \subsection{Affine algebraic groups}

    Let \(G\) be an algebraic group acting on an affine variety \(X = \Spec A\).
    For \(x \in G(\kk)\), we have the left translation of functions \(\tau_x: A \to A\) defined by \(\tau_x(f)(y) = f(x^{-1}y)\) for \(y \in X(\kk)\).

    \begin{lemma}\label{lem:finite_dimensional_invariant_subspace}
        Let \(G\) be an algebraic group acting on an affine variety \(X = \Spec A\).
        For any finite-dimensional subspace \(V \subseteq A\), there exists a finite-dimensional \(G\)-invariant subspace \(W \subseteq A\) containing \(V\).
    \end{lemma}
    \begin{proof}
        \Yang{To be continued...}
    \end{proof}

    \begin{theorem}\label{thm:affine_algebraic_group_is_linear}
        Any affine algebraic group is isomorphic to a closed algebraic subgroup of some \(\GL_n\).
    \end{theorem}
    \begin{proof}
        \Yang{To be continued...}
    \end{proof}

    % \begin{theorem}\label{thm:Chavellaye_decomposition}
    %     Let \(G\) be an algebraic group.
    %     Then there exists a unique maximal connected affine normal algebraic subgroup \(G_{\text{aff}}\) of \(G\) such that the quotient \(G/G_{\text{aff}}\) is an abelian variety.
    %     This subgroup is called the \emph{affine part} of \(G\).
    %     \Yang{To be continued...}
    % \end{theorem}

    % \begin{theorem}\label{thm:Rosenlicht_decomposition}
    %     Let \(G\) be an algebraic group.
    %     Then there exists a smallest normal connected algebraic subgroup \(G_{\text{ant}}\) of \(G\) such that the quotient \(G/G_{\text{ant}}\) is affine.
    %     This subgroup is called the \emph{anti-affine part} of \(G\).
    %     Moreover, \(G_{\text{ant}}\) is contained in the center of \(G^0\) and is smooth and connected.
    %     \Yang{To be continued...}
    % \end{theorem}


\subsection{Lie algebra of an algebraic group}

    Let \(G\) be an algebraic group.
    The \emph{Lie algebra} of \(G\) is defined to be the tangent space of \(G\) at the identity element \(\varepsilon\):
    \[
        \Lie(G) = T_\varepsilon G.
    \]
    It is a finite-dimensional vector space over \(\kk\).

    \begin{proposition}\label{prop:multiplication_of_group_induces_plus_on_Lie_algebra}
        The group law \(\mu: G \times G \to G\) induces the plus map on \(\Lie(G)\):
        \[
            \upd\mu_{(\varepsilon,\varepsilon)}: T_{(\varepsilon,\varepsilon)}(G \times G) \cong T_\varepsilon G \oplus T_\varepsilon G \to T_\varepsilon G, \quad (v,w) \mapsto v + w.
        \]
    \end{proposition}
    \begin{proof}
        We have 
        \[ \upd \mu_{(\varepsilon,\varepsilon)}(v,w) = \upd \mu_{(\varepsilon,\varepsilon)}(v,0) + \upd \mu_{(\varepsilon,\varepsilon)}(0,w) = (\upd \mu \circ (\id_G \times \varepsilon))_\varepsilon(v) + (\upd \mu \circ (\varepsilon \times \id_G))_\varepsilon(w) = v + w. \]
        % \Yang{To be filled.}
    \end{proof}