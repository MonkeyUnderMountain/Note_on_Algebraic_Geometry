\section{First properties of algebraic groups}

Let \(\kk\) be a field and \(\kkk\) its algebraic closure. 
All varieties are defined over \(\kk\) unless otherwise specified.

\subsection{Basic concepts}

    \begin{definition}\label{def:algebraic_group}
        An \emph{algebraic group} is a group object in the category of algebraic varieties, i.e. an algebraic variety $G$ together with morphisms
        \Yang{To be continued...}
    \end{definition}

    \begin{proposition}\label{prop:algebraic_group_is_smooth}
        Let \(G\) be an algebraic group.
        Then \(G\) is a smooth variety over \(\kk\).
    \end{proposition}

    \begin{example}\label{eg:additive_group}
        The \emph{additive group} $\bbG_a$ is defined to be the affine line $\mathbb{A}^1$ with the group law given by addition.
        Concretely, we can write $\bbG_a = \Spec \kk[T]$ with the group law given by the morphism
        \begin{align*}
            \mu:\bbG_a \times \bbG_a \to \bbG_a \quad & \kk[T] \to \kk[T]\ten_\kk \kk[T], \quad T \mapsto T\ten 1 + 1 \ten T. \\
            \iota:\bbG_a \to \bbG_a \quad & \kk[T] \to \kk[T], \quad T \mapsto -T. \\
            \varepsilon:\Spec \kk \to \bbG_a \quad & \kk[T] \to \kk, \quad T \mapsto 0.
        \end{align*}
        \Yang{To be continued...}
    \end{example}

    \begin{example}\label{eg:multiplicative_group}
        The \emph{multiplicative group} $\bbG_m$ is defined to be the affine variety $\mathbb{A}^1 \setminus \{0\}$ with the group law given by multiplication.
        Concretely, we can write $\bbG_m = \Spec \kk[T,T^{-1}]$ with the group law given by the morphism
        \begin{align*}
            \mu: \bbG_m \times \bbG_m \to \bbG_m & \leadsto \kk[T,T^{-1}] \to \kk[T,T^{-1}]\ten_\kk \kk[T,T^{-1}], \quad T \mapsto T\ten T. \\
            \iota: \bbG_m \to \bbG_m & \leadsto \kk[T,T^{-1}] \to \kk[T,T^{-1}], \quad T \mapsto T^{-1}. \\
            \varepsilon: \Spec \kk \to \bbG_m & \leadsto \kk[T,T^{-1}] \to \kk, \quad T \mapsto 1.
        \end{align*}
        \Yang{To be continued...}
    \end{example}

    \begin{example}\label{eg:general_linear_group}
        The \emph{general linear group} $\GL_n$ is defined to be the open subvariety of $\mathbb{A}^{n^2}$ consisting of invertible matrices, with the group law given by matrix multiplication.
        Concretely, we can write \(\GL_n = \Spec \kk[T_{ij}, \det(T_{ij})^{-1}]\) where \(1 \leq i,j \leq n\) and the group law is given by the morphism
        \Yang{To be continued...}
    \end{example}

    % \begin{example}\label{eg:special_linear_group}
    %     The \emph{special linear group} $\SL_n$ is defined to be the closed subvariety of $\GL_n$ consisting of matrices with determinant equal to \(1\), with the group law given by matrix multiplication.
        
    %     \Yang{To be continued...}
    % \end{example}

    \begin{example}\label{eg:elliptic_curve}
        An \emph{elliptic curve} is a smooth projective curve of genus \(1\) with a specified point \(O\).
        Given an elliptic curve \(E\), we can define a group law on \(E\) using the chord-tangent process.
        Concretely, for any two points \(P,Q \in E\), we can define their sum \(P+Q\) as follows:
        \begin{itemize}
            \item If \(P \neq Q\), then let \(L\) be the line passing through \(P\) and \(Q\).
                Since \(E\) is a cubic curve, \(L\) intersects \(E\) at a third point \(R\).
                Then we define \(P+Q\) to be the point obtained by reflecting \(R\) across the x-axis (i.e., if \(R=(x,y)\), then \(P+Q=(x,-y)\)).
            \item If \(P = Q\), then let \(L\) be the tangent line to \(E\) at \(P\).
                Again, since \(E\) is a cubic curve, \(L\) intersects \(E\) at a second point \(R\).
                Then we define \(2P = P+P\) to be the point obtained by reflecting \(R\) across the x-axis.
            \item The identity element is the specified point \(O\).
            \item The inverse of a point \(P=(x,y)\) is given by \(-P=(x,-y)\).
        \end{itemize}
        This group law makes \(E\) into an algebraic group.
        We can identify the \(\kkk\)-points of \(E\) with the set of points on the elliptic curve defined over \(\kkk\).
        \Yang{To be continued...}
    \end{example}

    \begin{definition}\label{def:transition_morphism}
        Let \(G\) be an algebraic group and \(x \in G(\kk)\) a \(\kk\)-point.
        The \emph{left translation} by \(x\) is the morphism
        \[
            L_x: G \xrightarrow{\cong} \Spec \kk \times G \xrightarrow{x \times \id_G} G \times G \xrightarrow{\mu} G,
        \]
    \end{definition}

    \begin{definition}\label{def:algebraic_subgroup}
        An \emph{algebraic subgroup} of an algebraic group \(G\) is a closed subvariety \(H \subseteq G\) that is also a subgroup of \(G\).
        In other words, the inclusion morphism \(H \hookrightarrow G\) is a morphism of algebraic groups.
    \end{definition}

    \begin{example}\label{eg:special_linear_group}
        The \emph{special linear group} $\SL_n$ is defined to be the closed subvariety of $\GL_n$ consisting of matrices with determinant equal to \(1\), with the group law given by matrix multiplication.
        Concretely, we can write \(\SL_n = \Spec \kk[T_{ij}]/(\det(T_{ij}) - 1)\) where \(1 \leq i,j \leq n\) and the group law is given by the morphism
        \Yang{To be continued...}
    \end{example}

    \begin{definition}\label{def:product_of_algebraic_groups}
        Let \(G\) and \(H\) be algebraic groups.
        The \emph{product} \(G \times H\) is an algebraic group with the group law defined by
        \[
            \mu_{G \times H} = (\mu_G, \mu_H): (G \times H) \times (G \times H) \cong (G \times G) \times (H \times H) \to G \times H,
        \]
        where \(\mu_G\) and \(\mu_H\) are the group laws of \(G\) and \(H\), respectively.
        \Yang{To be continued...}
    \end{definition}

    \begin{definition}\label{def:neutral_component}
        Let \(G\) be an algebraic group.
        The \emph{neutral component} \(G^0\) is the connected component of \(G\) containing the identity element \(\varepsilon\).
        % It is a closed, normal algebraic subgroup of \(G\).
        \Yang{To be continued...}
    \end{definition}

    \begin{proposition}\label{prop:neutral_component_is_subgroup}
        The neutral component \(G^0\) is a closed, normal algebraic subgroup of \(G\) of finite index.
        Moreover, each closed subgroup \(H\) of finite index contains \(G^0\).
    \end{proposition}
    \begin{proof}
        \Yang{To be continued...}
    \end{proof}

    \begin{definition}\label{def:homomorphism_of_algebraic_groups}
        A \emph{homomorphism} of algebraic groups is a morphism of varieties that is also a group homomorphism.
        In other words, a morphism \(f: G \to H\) between algebraic groups \(G\) and \(H\) is a homomorphism if the following diagram commutes:
        \[
            \begin{tikzcd}
                G \times G \arrow{r}{\mu_G} \arrow{d}{f \times f} & G \arrow{d}{f} \\
                H \times H \arrow{r}{\mu_H} & H
            \end{tikzcd}
        \]
        where \(\mu_G\) and \(\mu_H\) are the group laws of \(G\) and \(H\), respectively.
        \Yang{To be continued...}
    \end{definition}

    \begin{proposition}\label{prop:closure_of_subgroup_is_subgroup}
        Let \(G\) be an algebraic group and \(H \subseteq G\) a subgroup (not necessarily closed).
        Then the Zariski closure \(\overline{H}\) of \(H\) in \(G\) is an algebraic subgroup of \(G\).
        \Yang{To be continued...}
    \end{proposition}
    \begin{proof}
        \Yang{To be continued...}
    \end{proof}

    \begin{proposition}\label{prop:generate_subgroup_by_irred_constructible_subset}
        Let \(G\) be an algebraic group, \(Y_i\) irreducible constructible subsets of \(G\) containing the identity element for \(i=1,\ldots,n\).
        Then the closed subgroup \Yang{To be continued...}
    \end{proposition}
    \begin{proof}
        \Yang{To be continued...}
    \end{proof}

    \begin{remark}\label{rmk:generated_subgroup_by_irred_constructible_subset}
        We can take \(n \leq 2 \dim G\).
        \Yang{To be continued...}
    \end{remark}


\subsection{Action and representations}

    \begin{definition}\label{def:action_of_algebraic_group}
        An \emph{action} of an algebraic group \(G\) on a variety \(X\) is a morphism
        \[
            \sigma: G \times X \to X
        \]
        such that the following diagrams commute:
        \[
            \begin{tikzcd}
                G \times G \times X \arrow{r}{\mu \times \id_X} \arrow{d}{\id_G \times \sigma} & G \times X \arrow{d}{\sigma} \\
                G \times X \arrow{r}{\sigma} & X
            \end{tikzcd}
            \quad
            \begin{tikzcd}
                \Spec \kk \times X \arrow{r}{\varepsilon \times \id_X} \arrow{dr}[swap]{\cong} & G \times X \arrow{d}{\sigma} \\
                & X
            \end{tikzcd}
        \]
        where \(\mu\) is the group law of \(G\) and \(\varepsilon\) is the identity element of \(G\).
        In other words, for any field extension \(K/\kk\), the induced map \(G(K) \times X(K) \to X(K)\) defines a group action of the abstract group \(G(K)\) on the set \(X(K)\).
        We say that \(X\) is a \(G\)-variety.
        \Yang{To be continued...}
    \end{definition}

    \begin{example}\label{eg:linear_representation}
        A \emph{linear representation} of an algebraic group \(G\) on a finite-dimensional vector space \(V\) over \(\kk\) is an action of \(G\) on the affine space associated to \(V\), i.e. a morphism
        \[
            \rho: G \times V \to V
        \]
        such that for any field extension \(K/\kk\), the induced map \(G(K) \times V(K) \to V(K)\) defines a group homomorphism from the abstract group \(G(K)\) to the general linear group of the vector space \(V(K)\).
        In other words, for any \(g \in G(K)\), the map \(\rho_g: V(K) \to V(K)\) defined by \(\rho_g(v) = \rho(g,v)\) is a linear automorphism of \(V(K)\).
        We say that \(V\) is a \(G\)-module.
        \Yang{To be continued...}
    \end{example}