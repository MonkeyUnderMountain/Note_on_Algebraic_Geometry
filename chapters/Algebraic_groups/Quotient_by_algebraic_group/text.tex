\section{Quotient by algebraic group}

Everything in this section is over an arbitrary field \(\kk\) unless otherwise specified.

\subsection{Quotient}

    \begin{definition}\label{def:quotient_of_variety_by_group_action}
        Let \(G\) be an algebraic group acting on a variety \(X\).
        A \emph{quotient} of \(X\) by \(G\) is a variety \(Y\) together with a morphism \(\pi: X \to Y\) such that
        \begin{enumerate}
            \item \(\pi\) is \(G\)-invariant, i.e., \(\pi(g \cdot x) = \pi(x)\) for all \(g \in G\) and \(x \in X\).
            \item For any variety \(Z\) and any \(G\)-invariant morphism \(f: X \to Z\), there exists a unique morphism \(\overline{f}: Y \to Z\) such that \(f = \overline{f} \circ \pi\).
        \end{enumerate}
        In other words, the following diagram commutes:
        \[
            \begin{tikzcd}
                X \arrow[r, "\pi"] \arrow[rd, "f"'] & Y \arrow[d, "\overline{f}"] \\
                & Z
            \end{tikzcd}
        \]
        If a quotient exists, it is unique up to a unique isomorphism.
        \Yang{To be continued...}
    \end{definition}


\subsection{Passage to projective space}

    \begin{theorem}\label{thm:closed_subgroup_of_affine_algebraic_group_can_be_realized_as_stabilizer_of_line}
        Let \(G\) be an affine algebraic group and \(H\) a closed subgroup.
        Then there exists a finite-dimensional linear representation \(V\) of \(G\) and a one-dimensional subspace \(L \subseteq V\) such that \(H\) is the stabilizer of \(L\).
    \end{theorem}
    \begin{proof}
        \Yang{To be filled.}
    \end{proof}


\subsection{More general quotients}

    \begin{theorem}\label{thm:existence_of_quotient_by_algebraic_group}
        Let \(G\) be an affine algebraic group acting on a variety \(X\).
        Then there exists a variety \(Y\) and a rational morphism \(\pi: X \ratmap Y\) with commutative diagram
        \[
            \begin{tikzcd}
                X \arrow[r, dashed, "\pi"] \arrow[rd, "f"'] & Y \arrow[d, "\overline{f}"] \\
                & Z
            \end{tikzcd}
        \]
        satisfying the following universal property:
        If a quotient exists, it is unique up to a unique isomorphism.

        Furthermore, if all orbits of \(G\) in \(X\) are closed, then \(\pi\) is a morphism (i.e., defined everywhere).
        \Yang{To be continued...}
    \end{theorem}