\section{Application: birational group of varieties of general type}

    In this section, we apply the results from the previous sections to study the birational automorphism groups of varieties of general type.

    \begin{theorem}\label{thm:birational_group_of_varieties_of_general_type}
        Let \(X\) be a projective variety of general type over an algebraically closed field \(\kkk\) of characteristic zero. 
        Then the group of birational automorphisms \(\Bir(X)\) is finite.
    \end{theorem}
    \begin{proof}
        We will prove this theorem in several steps.
        By replacing \(X\) with its resolution of singularities, we may assume that \(X\) is smooth.

        \begin{step}\label{step_in_thm:birational_group_of_varieties_of_general_type:pluricanonical_representation}
            For every \(m \geq 1\), \(\Bir(X)\) linearly acts on \(H^0(X, mK_X)\) via pull-back of functions (as abstract group).
        \end{step}

        Let \(\fkk(X)\) be the function field of \(X\).
        Then for every \(g \in \Bir(X)\), \(g\) induces an automorphism of \(\fkk(X)\) over \(\kkk\), which we denote by \(g^*\).
        In particular we know that \(g^*\) is injective and \(\kkk\)-linear.
        By definition, \(H^0(X,mK_X) = \{ s \in \fkk(X) \mid \divisor(s) + mK_X \geq 0 \} \).
        We only need to show that for every \(s \in H^0(X,mK_X)\), \(g^*(s) \in H^0(X,mK_X)\) since \(\dim_\kkk H^0(X,mK_X) < \infty\).
        Consider the commutative diagram
        \[ \begin{tikzcd}
            \Gamma \arrow[d, "p"'] \arrow[rd, "q"] & \\
            X \arrow[r, dashed, "g"] & X
        \end{tikzcd} \]
        with \(\Gamma\) smooth and \(p,q\) birational morphisms.
        Then we have
        \[ K_\Gamma = p^*K_X + E_p = q^*K_X + E_q, \]
        where \(E_p\) and \(E_q\) are \(p\)- and \(q\)-exceptional divisors respectively.
        Moreover, \(E_p\) and \(E_q\) are effective since \(X\) is smooth.
        For every \(s \in H^0(X,mK_X)\), we have
        \[ \divisor (q^*s) + mK_\Gamma = q^*(\divisor(s) + mK_X) + mE_q \geq 0. \]
        Then
        \begin{align*}
            \divisor (g^*s) + mK_X &= p_* p^*(\divisor (g^*s) + mK_X) \\
            &= p_* \left(\divisor (q^*s) + mK_\Gamma - mE_p\right) \\
            &= p_* \left(\divisor (q^*s) + mK_\Gamma\right) \geq 0.
        \end{align*}
        It follows that \(g^*(s) \in H^0(X,mK_X)\).

        Note this action \(g \mapsto g^*\) is contravariant, i.e., for every \(g_1,g_2 \in \Bir(X)\), we have \((g_1 \circ g_2)^* = g_2^* \circ g_1^*\).

        \begin{step}\label{step_in_thm:birational_group_of_varieties_of_general_type:Bir_X_is_linear}
            The group \(\Bir(X)\) is a linear algebraic group by identifying it with a closed subgroup of \(\Aut(\bbP(V))\) for some finite-dimensional \(\kkk\)-vector space \(V\) (subspace of \(H^0(X,mK_X)\) for some \(m > 0\)).
            Moreover, its rational action on \(X\) is algebraic.
        \end{step}

        By \cref{thm:Iitaka_fibration}, there exists an integer \(m > 0\) such that the map \(\psi: X \dashrightarrow \bbP(H^0(X,mK_X))\) is birational onto its image \(Y\).
        Let \(V\) be the subspace of \(H^0(X,mK_X)\) spanned by the affine cone over \(Y\).
        % Then we have that \(h \in \PGL(V)\) is the identity whenever \(h|_Y\) is the identity.
        Since \(\Bir(X)\) linearly acts on \(H^0(X,mK_X)\) by \cref{step_in_thm:birational_group_of_varieties_of_general_type:pluricanonical_representation}, it also linearly acts on \(V\).
        we have a commutative diagram
        \[ \begin{tikzcd}
            X \arrow[d, dashed, "\psi"] \arrow[r, dashed, "g"] & X \arrow[d, dashed, "\psi"] \\
            Y \arrow[r, "\varphi_g|_Y"] \arrow[d, hook] & Y \arrow[d, hook] \\
            \bbP(V) \arrow[r, "\varphi_g"] & \bbP(V)
        \end{tikzcd} \]
        for every \(g \in \Bir(X)\), where \(\varphi_g\) is the induced automorphism of \(\bbP(V)\).
        
        Since \(\psi\) is birational, the map \(g \mapsto \varphi_g\) defines an injective group homomorphism from \(\Bir(X)\) to \(\Aut(\bbP(V))\).
        Consider the natural algebraic group structure on \(\Aut(\bbP(V))\) and let \(G\) be the Zariski closure of the image of \(\Bir(X)\) in \(\Aut(\bbP(V))\).
        Note that \(\Bir(X)\) fixes \(Y\).
        Thus \(G\) also fixes \(Y\).
        Since the affine cone over \(Y\) spans \(V\), we conclude that any element \(g \in G\) is uniquely determined by its restriction to \(Y\).
        In particular, we have \(G = \Bir(X)\).
        Note that \(\Aut(\bbP(V))\) is a linear algebraic group and so is its closed subgroup \(\Bir(X)\).

        \begin{step}\label{step_in_thm:birational_group_of_varieties_of_general_type:Bir_X_is_finite}
            If \(\dim \Bir(X) > 0\), then it contains \(\bbG_a\) or \(\bbG_m\) as a subgroup.
            We show that the action of \(\bbG_a\) or \(\bbG_m\) on \(X\) leads to \(X\) being uniruled, which contradicts the assumption that \(X\) is of general type.
        \end{step}

        By \cref{lem:any_linear_alg_gp_has_one_dimensional_alg_subgroup} and \cref{thm:classification_of_linear_alg_gp_of_dim1}, if \(\dim \Bir(X) > 0\), then \(\Bir(X)\) contains either \(\bbG_a\) or \(\bbG_m\) as a subgroup.
        Note that both \(\bbG_a\) and \(\bbG_m\) are rational varieties, without loss of generality, we may assume that \(\Bir(X)\) contains \(\bbG_m\) as a subgroup.
        % By replacing \(X\) by \(Y\) in the above diagram if necessary, we may assume that \(\Bir(X)\) acts on \(X\) morphismically.
        Then we have a rational map
        \[ \Phi: \bbG_m \times X \ratmap X. \]
        % Note that \(\{x \in X \mid \Phi|_{\bbG_m \times \{x\}} \text{ is constant}\}\) is a closed subset of \(X\).
        % Then there is a non-empty Zariski open subset \(U \subseteq X\) such that \(\forall x \in U\), \(\Phi|_{\bbG_m \times \{x\}}: \bbG_m \to X\) is not constant.
        % We get a morphism
        % \[ \Phi: \bbG_m \times U \to U. \]

        Fix \(x \in X\) such that \(\Phi|_{\bbG_m \times \{x\}}: \bbG_m \to X\) is not constant.
        Choose \(Z \subset X\) a closed subvariety of codimension \(1\) passing through \(x\) such that \(\bbG_m.x \not\subseteq Z\).
        Then the closure of \(\Phi(\bbG_m \times Z)\) in \(X\) has dimension at least \(\dim Z + 1 = \dim X\).
        Hence we have a dominant rational map
        \[ \Phi: \bbP^1 \times Z \ratmap X. \]
        This contradicts \cref{thm:K_X_not_pseudo_effective_iff_uniruled} and the assumption that \(X\) is of general type.
        Therefore, we must have \(\dim \Bir(X) = 0\), i.e., \(\Bir(X)\) is finite.
    \end{proof}

    \begin{remark}\label{rmk_on_finite_birational_group_of_general_type:projection_in_the_sense_of_Grothendieck_and_its_dual}
        In the proof of \cref{thm:birational_group_of_varieties_of_general_type}, by \(\bbP(V)\) we mean the projective space associated to the vector space \(V\) in the sense of Grothendieck, i.e., \(\bbP(V) = \Proj(\bigoplus_{k \geq 0} \Sym^k V)\).
        Hence if one have a linear map \(f: V \to W\) between two finite-dimensional \(\kkk\)-vector spaces, then it induces a morphism \(\bbP(W) \to \bbP(V)\) (not \(\bbP(V) \to \bbP(W)\)).
    \end{remark}

    \begin{corollary}\label{cor:regularization_of_birational_group_of_varieties_of_general_type}
        Let \(X\) be a projective variety of general type over an algebraically closed field \(\kkk\) of characteristic zero. 
        Then there exists a projective variety \(Y\) birational to \(X\) such that \(\Bir(Y) = \Aut(Y)\).
    \end{corollary}

    \begin{corollary}\label{cor:automorphism_group_of_Fano_varieties_is_linear}
        Let \(X\) be a smooth projective Fano variety over an algebraically closed field \(\kkk\) of characteristic zero. 
        Then the group of automorphisms \(\Aut(X)\) is a linear algebraic group.
    \end{corollary}
    \begin{proof}
        Note that for every \(g \in \Aut(X)\), \(g\) induces an automorphism of \(H^0(X, -mK_X)\) for every integer \(m \geq 1\) via pull-back of functions.
        Then the same argument as in \cref{step_in_thm:birational_group_of_varieties_of_general_type:Bir_X_is_linear} shows that \(\Aut(X)\) is a linear algebraic group.
    \end{proof}

    \begin{lemma}\label{lem:any_linear_alg_gp_has_one_dimensional_alg_subgroup}
        Let \(G\) be a linear algebraic group over an algebraically closed field \(\kkk\). 
        Then \(G\) has a one-dimensional algebraic subgroup.
    \end{lemma}

    % \begin{example}\label{eg:graph_of_birational_self_map_of_x_mapsto_1_over_x_on_P2}
    %     Here is an example to illustrate the construction in \cref{step_in_thm:birational_group_of_varieties_of_general_type:pluricanonical_representation}.
    %     Consider the birational self-map \(g: \bbP^2 \dashrightarrow \bbP^2\) defined by \([x:y:z] \mapsto [yz:xz:xy]\).
    %     The graph \(\Gamma\) of \(g\) is given by the blow-up of \(\bbP^2\) at the three points \([1:0:0]\), \([0:1:0]\) and \([0:0:1]\) and \(p,q\) contract different exceptional divisors as follows.
    %     \begin{center}
    %         \begin{tikzpicture}
    %               % Top part: hexagon with six lines, three red and three blue
    %                 \draw[red, thick] (1,0.866) -- (-1,0.866);
    %                 \draw[blue, thick] (-0.35,1.1258) -- (-1.15,-0.2598);
    %                 \draw[red, thick] (-1.15,0.2598) -- (-0.35,-1.1258);
    %                 \draw[blue, thick] (-1,-0.866) -- (1,-0.866);
    %                 \draw[red, thick] (0.35,-1.1258) -- (1.15,0.2598);
    %                 \draw[blue, thick] (1.15,-0.2598) -- (0.35,1.1258);
    
    %             % Bottom left: triangle from contracting red lines
    
    %             % Bottom right: triangle from contracting blue lines
    
    %             % Labels
    %             \node at (0,1.5) {\(\Gamma\)};
    %         \end{tikzpicture}
    %     \end{center}
    % \end{example}