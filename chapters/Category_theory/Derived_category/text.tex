\section{Vanishing in positive characteristic}


\subsection{Preliminaries}

    Let \(\bfA\) be an abelian category.
    The category \(\sfC(\bfA)\) of complexes in \(\bfA\) is defined as follows:
    the objects are complexes \(X^\bullet\) in \(\bfA\), and the morphisms are morphisms of complexes.
    For every \(X^\bullet \in \Obj(\sfC(\bfA))\), the object \(X^n\) is the \(n\)-th component of the complex, and the morphism \(\d^n: X^n \to X^{n+1}\) is the differential.
    
    We denote \(X[k]\) by the complex obtained by shifting \(X^\bullet\) by \(k\), that is,
    \[ X[k]^n = X^{n+k}, \quad \d^n_{X[k]} = (-1)^k \d^{n+k}_{X^\bullet}. \]
    
    Given a morphism \(f: X^\bullet \to Y^\bullet\) in \(\sfC(\bfA)\), we define the map cone \(\Cone(f)^\bullet \in \sfC(\bfA)\) by
    \[ \Cone(f)^n = X^{n+1} \oplus Y^n, \quad \d^n_{\Cone(f)} = \mat{\d^{n+1}_X}{}{f^{n+1}}{\d^n_Y}, \]
    Using the notation of shifting, we can also write
    \[ \Cone(f)^\bullet = \left( X[1]^\bullet \oplus Y^\bullet, \mat{\d_{X[1]}}{}{f[1]}{\d_Y} \right). \]
    \Yang{Check that the cone is a complex.}


    The category \(\sfK(\bfA)\) is defined by 
    \[ \Obj(\sfK(\bfA)) = \Obj(\sfC(\bfA)),\quad \Hom_{\sfK(\bfA)}(X^\bullet, Y^\bullet) = \Hom_{\sfC(\bfA)}(X^\bullet, Y^\bullet)/\{\text{homotopy}\}. \]

    A homomorphism \(f^\bullet: X^\bullet \to Y^\bullet\) is said to be a \emph{quasi-isomorphism} if the induced map \(H^n(f^\bullet): H^n(X^\bullet) \to H^n(Y^\bullet)\) is an isomorphism for all \(n\).

    \begin{example}\label{eg:resolution_as_quasi_isomorphism}
        Let \(\bfA\) be an abelian category and \(A\) an object in \(\bfA\).
        Let \(A \xrightarrow{i} I^\bullet\) be an injective resolution of \(A\).
        Then the complex \(I^\bullet\) is a complex in \(\sfC(\bfA)\), and the morphism 
        \[
            \begin{tikzcd}
                \cdots \arrow[r] & 0 \arrow[r] & A \arrow[d, "i"] \arrow[r] & 0 \arrow[r] & 0 \arrow[r] & \cdots \\
                \cdots \arrow[r] & 0 \arrow[r] & I^0 \arrow[r, "d^0"] & I^1 \arrow[r, "d^1"] & I^2 \arrow[r] & \cdots 
            \end{tikzcd}
        \]
        is a quasi-isomorphism in \(\sfK(\bfA)\).
    \end{example}

    \begin{definition}\label{def:triangle}
        A \emph{triangle} in \(\sfK(\bfA)\) (or \(\sfC(\bfA)\)) is a diagram of the form
        \[
            X^\bullet \xrightarrow{f} Y^\bullet \xrightarrow{g} Z^\bullet \xrightarrow{h} X[1]^\bullet
        \]
        such that \(f\), \(g\), and \(h\) are morphisms of complexes.
    \end{definition}

    For every \(f^\bullet: X^\bullet \to Y^\bullet\) in \(\sfC(\bfA)\), we can construct a triangle
    \[ X^\bullet \xrightarrow{f^\bullet} Y^\bullet \to \Cone(f^\bullet)^\bullet \to X[1]^\bullet, \]
    where the morphism \(Y^\bullet \to \Cone(f^\bullet)^\bullet\) is the natural inclusion, 
    and the morphism \(\Cone(f^\bullet)^\bullet \to X[1]^\bullet\) is the natural projection.
    The triangle which is isomorphic to the above triangle in \(\sfK(\bfA)\) is called \emph{distinguished triangle}.

    \begin{definition}[Truncation functor]\label{def:truncation_functor}
        The \emph{truncated functor} \(\tau^{>0}: \sfK(\bfA) \to \sfK(\bfA)\) is defined by
        \[ \tau^{>0}(X^\bullet)^n = \left( \cdots \to 0 \to \coker d^0 \to X^1 \to X^2 \to \cdots \right). \]
    \end{definition}
    \Yang{On cohomological level, we have}
    \[ H^n(\tau^{>0}(X^\bullet)) = \begin{cases}
        0, & n \leq 0, \\
        H^n(X^\bullet), & n > 0.
    \end{cases} \]


    \begin{definition}[Derived category]\label{def:derived_category}
        Let \(\bfA\) be an abelian category. The \emph{derived category} \(\sfD(\bfA)\) is defined by the following universal property: 
        for any
    \end{definition}

    \begin{proposition}\label{prop:existence_of_injective_resolution}
        Let \(\bfA\) be an abelian category with enough injectives.
        Then for every object \(X \in \sfD^+(\bfA)\), there exists an isomorphism \(X \to I\) in \(\sfD^+(\bfA)\) such \(I^n\) is an injective object in \(\bfA\) for all \(n\).
    \end{proposition}

    \begin{definition}
        Such an isomorphism \(X \to I\) is called an \emph{injective resolution} of \(X\).
    \end{definition}

    \begin{definition}[Right Derived functor]\label{def:derived_functor}
        Let \(\bfA\) and \(\bfB\) be abelian categories and \(F: \bfA \to \bfB\) a left exact functor. 
        The \emph{right derived functor} of \(F\) is a datum \((T,\alpha)\) fitting into the following diagram
        \[
            \begin{tikzcd}
                \sfK^+(\bfA) \arrow[r, "K^+(F)"] \arrow[d] & \sfK^+(\bfB) \arrow[d] \\
                \sfD^+(\bfA) \arrow[r, "T"] \arrow[ur,Rightarrow, "\alpha"] & \sfD^+(\bfB) 
            \end{tikzcd}
        \]
        that satisfies for every additive functor \(G: \sfD^+(\bfA) \to \sfD^+(\bfB)\) preserving distinguish triangles and the shifting \(X \mapsto X[1]\), the map
        \[ 
            \begin{tikzcd}
                \sfD^+(\bfA) \xrightarrow{T} \sfD^+(\bfB) \arrow[d, "\beta", Rightarrow] \\
                \sfD^+(\bfA) \xrightarrow{G}\sfD^+(\bfB)
            \end{tikzcd} 
            \quad \mapsto \quad
            \begin{tikzcd}
                \sfK^+(\bfA) \xrightarrow{K^+(F)} \sfK^+(\bfB) \to \sfD^+(\bfB) \arrow[d, "\alpha", Rightarrow] \\
                \sfK^+(\bfA) \to \sfD^+(\bfA) \xrightarrow{T} \sfD^+(\bfB) \arrow[d, "\beta\circ_h \id", Rightarrow] \\
                \sfK^+(\bfA) \to \sfD^+(\bfA) \xrightarrow{G} \sfD^+(\bfB)
            \end{tikzcd}
        \]
        is bijective.
    \end{definition}

    Such functor is unique up to isomorphism, and denoted by \(\sfR F\).

    \begin{proposition}\label{prop:derived_functor_by_resolution}
        Let \(\bfA\) be an abelian category with enough injectives, and \(F: \bfA \to \bfB\) a left exact functor.
        Then the right derived functor \(\sfR F\) is given by
        \[ \sfR F(X^\bullet) = F(I^\bullet), \]
        where \(I^\bullet\) is an injective resolution of \(X^\bullet\).
    \end{proposition}


\subsection{An example}

    Fix a base ring \(T = \bbZ_p[[u]]\) for some prime \(p>0\) and let \(x = (p,T)\) be the maximal ideal of \(T\).
    Let \(Z = \bbP^1_T\) be the projective line over \(T\).
    Choose a covering of \(Z\) by two affine open subschemes \(U_0 = \Spec(T[v])\) and \(U_1 = \Spec(T[1/v])\).
    Let \(I = (p,T,v) \subset T[v]\) be the ideal of the closed point \(z \in U_0 \subset Z\).

    Let \(\pi: X = \Bl_{p} Z \to Z\) be the blow-up of \(Z\) at the point \(z\).
    We try to describe it explicitly.
    Consider the blow-up \(\Proj T[v][pW,uW,vW]\) of \(U_0\) at the point \(z\), 
    where \(W\) is a formal variable to denote grading.
    It is covered by 
    \begin{align*}
        U_{01} &= \Spec\left(T[v]\left[\frac{uW}{pW},\frac{vW}{pW}\right]\right) \cong , \\
        U_{02} &= \Spec\left(T[v]\left[\frac{pW}{uW},\frac{vW}{uW}\right]\right) \cong , \\
        U_{03} &= \Spec\left(T[v]\left[\frac{pW}{vW},\frac{uW}{vW}\right]\right) \cong . 
    \end{align*}
    Reduce to the special fiber, they become
    \begin{align*}
        U_{01,x} &= \Spec\left(\bbF_p\left[\frac{uW}{pW},\frac{vW}{pW}\right]\right), \\
        U_{02,x} &= \Spec\left(\bbF_p\left[\frac{pW}{uW},\frac{vW}{uW}\right]\right), \\
        U_{03,x} &= \Spec\left(\bbF_p[v]\left[\frac{pW}{vW},\frac{uW}{vW}\right]/(v\frac{pW}{vW},v\frac{uW}{vW})\right) \cong \Spec\left(\bbF_p[v,\alpha,\beta]/(v\alpha,v\beta)\right). 
    \end{align*}
    Glue these three affine schemes and \(U_{1,x}\) together, we obtain the special fiber \(X_{x}\), which consists of two components \(\bbP_{\bbF_p}^1\) and \(\bbP_{\bbF_p}^2\) meeting at one point.
    It follows that the exceptional divisor \(E\) of the blow-up \(\pi: X \to Z\) is isomorphic to \(\bbP_{\bbF_p}^2\).

    Reduce to the fiber \(p=0\), we have 
    \begin{align*}
        U_{01,p} &= \Spec\left(\bbF_p\left[\frac{uW}{pW},\frac{vW}{pW}\right]\right), \\
        U_{02,p} &= \Spec\left(\bbF_p[[u]]\left[\frac{pW}{uW},\frac{vW}{uW}\right]/\left(u\frac{pW}{uW}\right)\right), \\
        U_{03,p} &= \Spec\left(\bbF_p[[u]]\left[v, \frac{pW}{vW},\frac{uW}{vW}\right]/\left(v\frac{pW}{vW},v\frac{uW}{vW}-u\right)\right).
    \end{align*}

    Let \(\calL := \pi^* \calO_Z(1)\) be the pullback of the line bundle \(\calO_Z(1)\) on \(Z\).
    \Yang{Then \(\calL\) is nef and big.}
    We use this example to compute 
    \[ \sfR \Gamma_x(\sfR \Gamma(X_{p=0}, \calL)) \]


    \begin{definition}
        Let \(X\) be a scheme and \(\calF\) a coherent sheaf on \(X\).
        For \(s \in \Gamma(X,\calF)\), we define the \emph{support} of \(s\) to be the closed subset \(\{x \in X \mid s_x \neq 0\}\).
        Let \(Y \subset X\) be a closed subset.
        The section with support in \(Y\) is defined to be the set
        \[ \Gamma_Y(X,\calF) = \{ s \in \Gamma(X,\calF) \mid \Supp s \subset Y \}. \]
    \end{definition}

    \paragraph{Compute \(\sfR \Gamma_x(\sfR \Gamma(Z_{p=0}, \calO_Z(1)))\)}

    
