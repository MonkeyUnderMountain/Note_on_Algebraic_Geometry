\section{Birational geometry on surfaces}

Let \(\kkk\) be an algebraically closed field of arbitrary characteristic.
Unless otherwise specified, all varieties are defined over \(\kkk\).

\subsection{Birational morphisms on surfaces}

    Let \(X\) be a smooth projective surface, \(0 \in X(\kkk)\) and \(\pi:\tilde{X} = \Bl_0 X \to X\) the blow-up of \(X\) at \(0\).
    Denote by \(E\) the exceptional divisor of \(\pi\).
    
    \begin{proposition}\label{prop:picard_group_after_blowing_up_on_surface}
        We have \(\Pic \tilde{X} = \Pic X \oplus \bbZ \cdot [E]\) and \(E^2 = -1\).
    \end{proposition}
    \begin{proof}
        \Yang{To be continued}
    \end{proof}

    \begin{proposition}\label{prop:canonical_divisor_after_blowing_up_on_surface}
        We have \(K_{\tilde{X}} = \pi^* K_X + E\).
    \end{proposition}
    \begin{proof}
        We have the exact sequence
        \[ \Omega_{\tilde{X}} \to \pi^* \Omega_X \to \Omega_{\tilde{X}/X} \to 0. \]
        Since both \(\tilde{X}\) and \(X\) are smooth, \(\Omega_{\tilde{X}}\) and \(\Omega_X\) are locally free sheaves of rank \(2\).
        The kernel of the first map is of rank \(0\) and torsion, thus it is zero.
        Therefore, we have the short exact sequence
        \[ 0 \to \Omega_{\tilde{X}} \to \pi^* \Omega_X \to \Omega_{\tilde{X}/X} \to 0. \]
        By taking \(c_1\), we only need to show that \(c_1(\Omega_{\tilde{X}/X}) = E\).

        For \(\eta \in \tilde{X}\) of codimension \(1\), if \(\eta \notin E\), then \((\Omega_{\tilde{X}/X})_\eta = \Omega_{\calO_{\tilde{X},\eta}/\calO_{X,\pi(\eta)}} = 0\).
        Hence we only need to consider the case \(\overline{\{\eta\}} = E\).
        \Yang{To be continued}
    \end{proof}


    \begin{corollary}\label{cor:self-intersection_number_of_canonical_divisor_after_blowing_up_on_surface}
        We have \(K_{\tilde{X}}^2 = K_X^2 - 1\).
    \end{corollary}
    \begin{proof}
        By \cref{prop:canonical_divisor_after_blowing_up_on_surface}, we have
        \[ K_{\tilde{X}}^2 = (\pi^* K_X + E)^2 = (\pi^* K_X)^2 + 2 \pi^* K_X \cdot E + E^2 = K_X^2 + 0 - 1 = K_X^2 - 1. \]
    \end{proof}
    
    \begin{proposition}\label{prop:pull_back_of_curves_pass_through_center_along_blow_up}
        Let \(C \subseteq X\) be an irreducible curve passing through \(0\) with multiplicity \(m = \mult_0 C\).
        Then the total transform of \(C\) along \(\pi:\tilde{X} \to X\) is given by
        \[ \pi^* C = \tilde{C} + m E, \]
        where \(\tilde{C}\subseteq \tilde{X}\) is the strict transform of \(C\).
        \Yang{To be revised.}
    \end{proposition}

    \begin{theorem}\label{thm:birational_map_decomposition_on_surfaces}
        Let \(f:X \to Y\) be a birational morphism between two smooth projective surfaces.
        Then \(f\) can be decomposed as a finite sequence of blow-ups at points.
        % \Yang{To be checked.}
    \end{theorem}
    \begin{proof}
        \Yang{To be continued}
    \end{proof}

\subsection{Castelnuovo's Theorem}

    \begin{definition}\label{def:-1-curve}
        A \emph{\((-1)\)-curve} on a smooth projective surface \(X\) is an irreducible curve \(C\subseteq X\) such that \(C\cong \bbP^1\) and \(C^2=-1\).
    \end{definition}

    \begin{remark}\label{rmk:-1-curve_and_extremal_ray_of_Psef_1}
        Let \(C\) be a \((-1)\)-curve on a smooth projective surface \(X\).
        Then its numerical class \([C]\in \rmN_1(X)\) spans an extremal ray of \(\Psef_1(X)\) such that \(K_X\cdot C < 0\).
        \Yang{To be revised.}
    \end{remark}

    \begin{theorem}[Castelnuovo's contractibility criterion]\label{thm:castelnuovo_contractibility_criterion}
        Let \(X\) be a smooth projective surface and \(C\subseteq X\) an irreducible curve.
        Then there exists a birational morphism \(f:X\to Y\) contracting \(C\) to a smooth point if and only if \(C\) is a \((-1)\)-curve.
    \end{theorem}
    \begin{proof}
        \Yang{To be continued}
    \end{proof}

    \begin{definition}\label{def:minimal_surface}
        A \emph{minimal surface} is a smooth projective surface that does not contain any \((-1)\)-curves.
        \Yang{To be checked.}
    \end{definition}

\subsection{Resolution of singularities on surfaces}

    \begin{definition}\label{def:resolution_of_singularities}
        A \emph{resolution of singularities} of a projective surface \(X\) is a smooth projective surface \(\widetilde{X}\) together with a birational and proper morphism \(\pi:\widetilde{X}\to X\) such that \(\pi\) is an isomorphism over the smooth locus of \(X\).
        \Yang{To be checked.}
    \end{definition}

    \begin{theorem}[Resolution of singularities on surfaces]\label{thm:resolution_of_singularities_on_surfaces}
        Let \(X\) be a projective surface over an algebraically closed field \(\kkk\).
        Then \(X\) admits a resolution of singularities.
        \Yang{To be checked.}
    \end{theorem}

    \begin{definition}\label{def:minimal_resolution}
        Let \(X\) be a projective surface.
        A \emph{minimal resolution} of \(X\) is a resolution of singularities \(\pi:\widetilde{X}\to X\) such that for any other resolution of singularities \(\pi':\widetilde{X}'\to X\), there exists a morphism \(f:\widetilde{X}'\to \widetilde{X}\) such that \(\pi'\) factors as \(\pi' = \pi \circ f\).
    \end{definition}

    \begin{proposition}\label{prop:minimal_resolution_on_surfaces}
        Let \(X\) be a projective surface.
        Then \(X\) admits a unique minimal resolution of singularities.
    \end{proposition}
    \begin{proof}
        \Yang{To be continued}
    \end{proof}