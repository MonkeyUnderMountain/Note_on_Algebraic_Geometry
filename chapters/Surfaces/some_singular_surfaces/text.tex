\section{Some Singular Surfaces}

In this section, fix an algebraically closed field \(\kkk\).
Everything is over \(\kkk\) unless otherwise specified.

\subsection{Projective cone over smooth projective curve}

    Let \(C \subset \bbP^n\) be a smooth projective curve.
    The \emph{projective cone} over \(C\) is the projective variety \(X \subset \bbP^{n+1}\) defined by the same homogeneous equations as \(C\).
    The variety \(X\) is singular at the vertex of the cone, which corresponds to the point \([0:\cdots:0:1]\in \bbP^{n+1}\).


\subsection{Du Val singularites}


    Du Val singularities (also known as rational double points, or ADE singularities) are a class of surface singularities that arise in algebraic geometry and complex surface theory.
    They are characterized by their resolution properties and can be classified according to the ADE classification of simply laced Dynkin diagrams.

    A Du Val singularity can be locally described by one of the following equations in \(\bbC^3\):

    \begin{itemize}
        \item \(A_n\) singularity: \(x^2 + y^2 + z^{n+1} = 0\) for \(n \geq 1\)
        \item \(D_n\) singularity: \(x^2 + y^{n-1} + yz^2 = 0\) for \(n \geq 4\)
        \item \(E_6\) singularity: \(x^2 + y^3 + z^4 = 0\)
        \item \(E_7\) singularity: \(x^2 + y^3 + yz^3 = 0\)
        \item \(E_8\) singularity: \(x^2 + y^3 + z^5 = 0\)
    \end{itemize}

    These singularities are important in the study of algebraic surfaces, particularly in the context of minimal models and the classification of surfaces.
    They also appear in various areas of mathematics and theoretical physics, including string theory and mirror symmetry.

    {\Large \Yang{Above by ai}}

\subsection{Quotient singularities}

    Consider \(\mu_n\), the group of \(n\)-th roots of unity, acting on \(\bbA^2\) by
    \[ \zeta \cdot (x,y) = (\zeta x, \zeta^m y) \]
    for a fixed integer \(m\) with \(\gcd(m,n) = 1\).

    \((a_1,...,a_n)/r\)-singularity is the singularity obtained by taking the quotient of \(\bbA^n\) by the action of \(\mu_r\) defined by
    \[ \zeta \cdot (x_1,...,x_n) = (\zeta^{a_1} x_1, ..., \zeta^{a_n} x_n) \]
    where \(\zeta\) is a primitive \(r\)-th root of unity.

    \(A_n\) singularity is the quotient singularity of type \((1,-1)/ (n+1)\).

    Its minimal resolution has exceptional locus consisting of a chain of \(n\) smooth rational curves, each with self-intersection \(-2\).
    Looks like:
    \[ \tikzcdset{every arrow/.append style={dash}}
    \begin{tikzcd}
        \bullet \arrow[r] & \bullet \arrow[r] & \bullet \arrow[r] & \cdots \arrow[r] & \bullet
    \end{tikzcd} \]

    Du Val singularities can be got by deforming \(A_n\) singularities.(general fiber \(A_n\), special fiber Du Val).