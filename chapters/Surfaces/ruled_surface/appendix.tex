\section{Appendix}

    Let \(S\) be a variety over \(\kkk\) and \(\calE\) a vector bundle of rank \(r+1\) on \(S\).

    \begin{proposition}\label{prop:isomorphic_projective_bundle_iff_twist_by_line_bundle}
        The \(S\)-varieties \(\bbP_X(\calE) \cong \bbP_X(\calE')\) if and only if \(\calE \cong \calE' \otimes \calL\) for some line bundle \(\calL\) on \(S\).
    \end{proposition}

    \begin{theorem}\label{thm:Eulur_sequence_for_projective_bundle}
        Let \(\pi: X=\bbP_S(\calE) \to S\) be the projective bundle associated to a vector bundle \(\calE\) of rank \(r+1\) on \(S\). 
        Then there is an exact sequence of vector bundles on \(\bbP_S(\calE)\)
        \[
            0 \to \Omega_{\bbP_S(\calE)/S} \to \pi^*(\calE)(-1) \to \calO_{\bbP_S(\calE)} \to 0.
        \]
        In particular, \(K_X \sim \pi^*(K_S + \det \calE) - (r+1)\calO_{\bbP_S(\calE)}(1)\).
        \Yang{To be continued...}
    \end{theorem}

    \begin{theorem}[Tsen's Theorem, {\cite[Tag 03RD]{Stacks}}]\label{thm:Tsen_theorem}
        Let \(C\) be a smooth curve over an algebraically closed field \(\kkk\). 
        Then \(\KK=\kkk(C)\) is a \(C_1\) field, i.e., every degree \(d\) hypersurface in \(\bbP^n_{\KK}\) has a \(\KK\)-rational point provided \(d \leq n\).
        % \Yang{Need a reference.}
    \end{theorem}

    % \begin{theorem}[Cohomology and Base Change, {\cite[Theorem 12.11]{Har77}}]\label{thm:cohomology_and_base_change}
    %     Let \(f:X \to S\) be a projective morphism of noetherian schemes and \(\calF\) a coherent sheaf on \(X\) which is flat over \(S\). 
    %     Then for each \(i \geq 0\) and each point \(s \in S\) there is a natural base change homomorphism
    %     \[
    %         \varphi_s^i: \sfR^i f_*\calF \ten_{\calO_S} \rkk(s) \to H^i(X_s,\calF_s).
    %     \]
    %     Suppose that \(\varphi_s^i\) is surjective. 
    %     Then
    %     \begin{enumerate}
    %         \item there exists an open neighborhood \(U\) of \(s\) such that \(\varphi_{s'}^i\) is an isomorphism for all \(s' \in U\);
    %         \item TFAE:
    %             \begin{enumerate}
    %                 \item \(\varphi_{s}^{i-1}\) is surjective;
    %                 \item \(\sfR^i f_*\calF\) is locally free on an open neighborhood of \(s\).
    %             \end{enumerate}
    %     \end{enumerate}
    % \end{theorem}

    \begin{theorem}[Grauert's Theorem, {\cite[Corollary 12.9]{Har77}}]\label{thm:Grauert_theorem}
        Let \(f:X \to S\) be a projective morphism of noetherian schemes and \(\calF\) a coherent sheaf on \(X\) which is flat over \(S\).
        Suppose that \(S\) is integral and the function \(s \mapsto \dim_{\rkk(s)} H^i(X_s,\calF_s)\) is constant on \(S\) for some \(i \geq 0\). 
        Then \(\sfR^i f_*\calF\) is locally free and the base change homomorphism
        \[
            \varphi_s^i: \sfR^i f_*\calF \ten_{\calO_S} \rkk(s) \to H^i(X_s,\calF_s)
        \]
        is an isomorphism for all \(s \in S\).
    \end{theorem}

    \begin{theorem}[Miracle Flatness, {\cite[Theorem 23.1]{Mat89}}]\label{thm:miracle_flatness}
        Let \(f:X \to Y\) be a morphism of noetherian schemes. 
        Assume that \(Y\) is regular and \(X\) is Cohen-Macaulay. 
        If all fibers of \(f\) have the same dimension \(d = \dim X - \dim Y\), then \(f\) is flat.
    \end{theorem}

    \begin{proposition}[Geometric form of Nakayama's Lemma]\label{prop: geometric form of Nakayama's lemma}
        Let \(X\) be a variety, $x\in X$ a closed point and $\calF$ a coherent sheaf on $X$.
        If $a_1,\cdots,a_k \in \calF(X)$ generate $\calF|_x = \calF \ten \rkk(x)$, then there is an open subset $U \subset X$ such that $a_i|_U$ generate $\calF(U)$. 
    \end{proposition}

    \begin{proposition}\label{prop:relative_morphism_to_projective_bundle}
        Let \(S\) be a noetherian scheme and \(\calE\) a vector bundle of rank \(r+1\) on \(S\).
        Denote by \(\pi: \bbP_S(\calE) \to S\) the projection.
        Let \(X\) be an \(S\)-scheme via a morphism \(g:X \to S\).
        Then there is a bijection
        \[
            \left\{\begin{array}{l}
                S\text{-morphisms }\\
                X \to \bbP_S(\calE)
            \end{array}\right\}
            \leftrightarrow
            \left\{
                \begin{array}{l}
                    \calL \in \Pic(X) \text{ and surjective}\\
                    \text{homomorphisms } g^*\calE \to \calL
                \end{array}
            \right\}.
        \]
        % \Yang{Need to check.}
    \end{proposition}
    \begin{proof}
        We have a surjection \(\pi^*\calE \to \calO_{\bbP_S(\calE)}(1)\) by the definition of \(\bbP_S(\calE)\).
        If we have a morphism \(f:X \to \bbP_S(\calE)\) over \(S\), then we have a surjective homomorphism \(f^*\pi^*\calE \to f^*\calO_{\bbP_S(\calE)}(1)\).

        Suppose we have a surjective homomorphism \(g^*\calE \surjmap \calL\) where \(\calL\) is a line bundle on \(X\).
        Take an affine cover \(\{U_i\}\) of \(S\) such that \(\calE|_{U_i}\) is trivial.
        On \(U_i\), choose a basis \(e_0^{(i)},\ldots,e_r^{(i)}\) of \(\calE|_{U_i}\).
        Suppose \(\bbP_S(\calE)\) is given by gluing \(\bbP^r_{U_i}\) via \(\varphi_{ij}\) induced by the transition functions of \(\calE\).
        
        The surjection \(g^*\calE|_{U_i} \surjmap \calL|_{X_{U_i}}\) gives a unique morphism \(f_i: X_{U_i} \to \bbP^r_{U_i}\) by \cref{thm:morphism_to_projective_space}.
        On \(X_{U_i \cap U_j}\), \(f_i\) and \(f_j\) agree since we have 
        \[ \begin{tikzcd}
            X_{U_i \cap U_j} \arrow[r, "="] \arrow[d, "f_i"'] & X_{U_i \cap U_j} \arrow[d, "f_j"]  \\
            \bbP_{U_i\cap U_j}(\bigoplus\calO_{U_i \cap U_j} e_k^{(i)}) \arrow[r, "\varphi_{ij}"] & \bbP_{U_i\cap U_j}(\bigoplus\calO_{U_i \cap U_j} e_k^{(j)})
        \end{tikzcd} \]
        and the bottom arrow is identical to the identity map on \(\bbP_{S}(\calE)_{U_i\cap U_j}\).
        Gluing \(f_i\) gives a morphism \(f:X \to \bbP_S(\calE)\) over \(S\).
        In particular, we have \(\calL \cong f^*\calO_{\bbP_S(\calE)}(1)\).
    \end{proof}

    \begin{definition}\label{def:extension}
        An \emph{extension} of a coherent sheaf \(\calF\) by a coherent sheaf \(\calG\) on a scheme \(X\) is an exact sequence of coherent sheaves
        \[ S = (0 \to \calG \to \calE \to \calF \to 0). \]
        Two extensions \(S\) and \(S'\) are \emph{equivalent} if there is a commutative diagram
        \[
            \begin{tikzcd}
                0 \arrow[r] & \calG \arrow[r] \arrow[d, "\id_{\calG}"] & \calE \arrow[r] \arrow[d, "\cong"] & \calF \arrow[r] \arrow[d, "\id_{\calF}"] & 0 \\
                0 \arrow[r] & \calG \arrow[r] & \calE' \arrow[r] & \calF \arrow[r] & 0.
            \end{tikzcd}
        \]
    \end{definition}

    \begin{proposition}\label{prop:extension_and_Ext1}
        Let \(X\) be a scheme and \(\calF,\calG\) be coherent sheaves on \(X\).
        Then there is a one-to-one correspondence between equivalence classes of extensions
        \[ S = (0 \to \calG \to \calE \to \calF \to 0) \]
        and elements of \(\Ext^1_X(\calF,\calG)\) given by 
        \[ S \mapsto \delta(\id_{\calF}) \]
        where \(\delta:\Hom_X(\calF,\calF) \to \Ext^1_X(\calF,\calG)\) is the connecting homomorphism.
    \end{proposition}
    \begin{proof}
        Take an exact sequence
        \[ 0 \to \calG \to \calI \xrightarrow{\varphi} \calC \to 0 \] 
        with \(\calI\) injective.
        Applying \(\Hom_X(\calF,-)\) gives a long exact sequence
        \[ 0 \to \Hom_X(\calF,\calG) \to \Hom_X(\calF,\calI) \to \Hom_X(\calF,\calC) \xrightarrow{\delta} \Ext^1_X(\calF,\calG) \to 0. \]
        For \(a \in \Ext^1_X(\calF,\calG)\), choose a lifting \(\alpha \in \Hom_X(\calF,\calC)\) of \(a\).
        Let \(\calE := \Ker (\calI \oplus \calF \to \calC, (i,f) \mapsto \varphi(i) - \alpha(f))\).

        Let \(\calE \to \calF\) be the projection to the second factor.
        It is surjective since \(\varphi\) is surjective.
        Consider the inclusion \(\calG \to \calI \to \calI \oplus \calF\), which factors through \(\calE\).
        On the other hand, if \(e \in \calE\) maps to \(0\) in \(\calF\), then \(e \in \calI\) and \(\varphi(e) = 0\), whence \(e \in \calG\).
        Hence we have an extension \(S = (0 \to \calG \to \calE \to \calF \to 0)\).

        \Yang{To be continued...}
    \end{proof}