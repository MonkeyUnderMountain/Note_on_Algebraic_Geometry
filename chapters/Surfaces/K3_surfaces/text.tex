\section{K3 surface}

Let \(\kkk\) be an algebraically closed field of arbitrary characteristic.
Unless otherwise specified, all varieties are defined over \(\kkk\).


\subsection{The first properties}

    \begin{definition}\label{def:K3_surface}
        A \emph{K3 surface} is a smooth, projective surface \(X\) with trivial canonical bundle \(K_X\cong \calO_X\) and irregularity \(q(X)=h^1(X,\calO_X)=0\).
    \end{definition}

    \begin{example}\label{ex:quartic_in_P3_is_K3}
        A smooth quartic surface \(X\subseteq \bbP^3\) is a K3 surface.
        Indeed, by the adjunction formula, we have
        \[
            K_X = (K_{\bbP^3} + X)|_X = (-4H + 4H)|_X = 0,
        \]
        where \(H\) is a hyperplane in \(\bbP^3\).
        Moreover, by the exact sequence
        \[
            0 \to \calO_{\bbP^3}(-4) \to \calO_{\bbP^3} \to \calO_X \to 0,
        \]
        we have long exact sequence in cohomology
        \[
            \cdots \to H^1(\bbP^3,\calO_{\bbP^3}) \to H^1(X,\calO_X) \to H^2(\bbP^3,\calO_{\bbP^3}(-4)) \to \cdots.
        \]
        Since \(H^1(\bbP^3,\calO_{\bbP^3})=0\) and \(H^2(\bbP^3,\calO_{\bbP^3}(-4))=0\), we get \(H^1(X,\calO_X)=0\).
        % By the Lefschetz hyperplane theorem, we have \(h^1(X,\calO_X)=h^1(\bbP^3,\calO_{\bbP^3})=0\).
        % \Yang{To be checked.}
    \end{example}

\subsection{Hodge Structure and Moduli of K3 surfaces}


\subsection{Neron-Severi group of K3 surfaces}

    